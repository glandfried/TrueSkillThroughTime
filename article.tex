\documentclass[article]{jss}
\usepackage[utf8]{inputenc}
% \usepackage{draftwatermark}
% \SetWatermarkText{Draft}
% \SetWatermarkScale{2}
\newif\ifen
\newif\ifes
\newcommand{\en}[1]{\ifen#1\fi}
\newcommand{\es}[1]{\ifes#1\fi}
\entrue
\usepackage[utf8]{inputenc}
\usepackage{paracol}
%\newcommand{\jan}{%
%  \en{January}%
%  \es{Enero}
%}

\newcommand\Wider[2][3em]{%
\makebox[\linewidth][c]{%
  \begin{minipage}{\dimexpr\textwidth+#1\relax}
  \raggedright#2
  \end{minipage}%
  }%
}

\usepackage{wrapfig}
\usepackage{caption}
\usepackage{subcaption}
\usepackage{amsmath} %para escribir funci\'on partida , matrices
\usepackage{amsthm} %para numerar definciones y teoremas
\usepackage{amsfonts} % \mathbb{N} -> conjunto de los n\'umeros naturales

\usepackage{bm} % \bm{\alpha} bold greek symbol
\usepackage[makeroom]{cancel} % \cancel{} \bcancel{} etc
\usepackage{wrapfig} % \begin{wrapfigure} Pone figura al lado del texto
\usepackage{mdframed}
\usepackage{algorithm}
\usepackage{comment}
%\usepackage{quoting}
\usepackage{mathtools}
\usepackage{tikz}
\usepackage{csvsimple}
\usepackage{listings}
\renewcommand{\lstlistingname}{Code}% Listing -> Algorithm
\usepackage{hyperref}
\usepackage{todonotes}
\definecolor{dkgreen}{rgb}{0,0.6,0}
\definecolor{mauve}{rgb}{0.58,0,0.82}

\definecolor{julia}{rgb}{1, 0.8, 1}
\definecolor{python}{rgb}{0.8, 0.9, 1}
\definecolor{r}{rgb}{0.8, 1, 0.8}
\definecolor{all}{rgb}{0.85, 0.85, 0.85}



% tikzlibrary.code.tex
%
% Copyright 2010-2011 by Laura Dietz
% Copyright 2012 by Jaakko Luttinen
%
% This file may be distributed and/or modified
%
% 1. under the LaTeX Project Public License and/or
% 2. under the GNU General Public License.
%
% See the files LICENSE_LPPL and LICENSE_GPL for more details.

% Load other libraries

%\newcommand{\vast}{\bBigg@{2.5}}
% newcommand{\Vast}{\bBigg@{14.5}}
% \usepackage{helvet}
% \renewcommand{\familydefault}{\sfdefault}

\usetikzlibrary{shapes}
\usetikzlibrary{fit}
\usetikzlibrary{chains}
\usetikzlibrary{arrows}

% Latent node
\tikzstyle{latent} = [circle,fill=white,draw=black,inner sep=1pt,
minimum size=20pt, font=\fontsize{10}{10}\selectfont, node distance=1]
% Observed node
\tikzstyle{obs} = [latent,fill=gray!25]
% Invisible node
\tikzstyle{invisible} = [latent,minimum size=0pt,color=white, opacity=0, node distance=0]
% Constant node
\tikzstyle{const} = [rectangle, inner sep=0pt, node distance=0.1]
%state
\tikzstyle{estado} = [latent,minimum size=8pt,node distance=0.4]
%action
\tikzstyle{accion} =[latent,circle,minimum size=5pt,fill=black,node distance=0.4]
\tikzstyle{fijo} =[latent,circle,minimum size=5pt,fill=black]


% Factor node
\tikzstyle{factor} = [rectangle, fill=black,minimum size=10pt, draw=black, inner
sep=0pt, node distance=1]
% Deterministic node
\tikzstyle{det} = [latent, rectangle]

% Plate node
\tikzstyle{plate} = [draw, rectangle, rounded corners, fit=#1]
% Invisible wrapper node
\tikzstyle{wrap} = [inner sep=0pt, fit=#1]
% Gate
\tikzstyle{gate} = [draw, rectangle, dashed, fit=#1]

% Caption node
\tikzstyle{caption} = [font=\footnotesize, node distance=0] %
\tikzstyle{plate caption} = [caption, node distance=0, inner sep=0pt,
below left=5pt and 0pt of #1.south east] %
\tikzstyle{factor caption} = [caption] %
\tikzstyle{every label} += [caption] %

\tikzset{>={triangle 45}}

%\pgfdeclarelayer{b}
%\pgfdeclarelayer{f}
%\pgfsetlayers{b,main,f}

% \factoredge [options] {inputs} {factors} {outputs}
\newcommand{\factoredge}[4][]{ %
  % Connect all nodes #2 to all nodes #4 via all factors #3.
  \foreach \f in {#3} { %
    \foreach \x in {#2} { %
      \path (\x) edge[-,#1] (\f) ; %
      %\draw[-,#1] (\x) edge[-] (\f) ; %
    } ;
    \foreach \y in {#4} { %
      \path (\f) edge[->,#1] (\y) ; %
      %\draw[->,#1] (\f) -- (\y) ; %
    } ;
  } ;
}

% \edge [options] {inputs} {outputs}
\newcommand{\edge}[3][]{ %
  % Connect all nodes #2 to all nodes #3.
  \foreach \x in {#2} { %
    \foreach \y in {#3} { %
      \path (\x) edge [->,#1] (\y) ;%
      %\draw[->,#1] (\x) -- (\y) ;%
    } ;
  } ;
}

% \factor [options] {name} {caption} {inputs} {outputs}
\newcommand{\factor}[5][]{ %
  % Draw the factor node. Use alias to allow empty names.
  \node[factor, label={[name=#2-caption]#3}, name=#2, #1,
  alias=#2-alias] {} ; %
  % Connect all inputs to outputs via this factor
  \factoredge {#4} {#2-alias} {#5} ; %
}

% \plate [options] {name} {fitlist} {caption}
\newcommand{\plate}[4][]{ %
  \node[wrap=#3] (#2-wrap) {}; %
  \node[plate caption=#2-wrap] (#2-caption) {#4}; %
  \node[plate=(#2-wrap)(#2-caption), #1] (#2) {}; %
}

% \gate [options] {name} {fitlist} {inputs}
\newcommand{\gate}[4][]{ %
  \node[gate=#3, name=#2, #1, alias=#2-alias] {}; %
  \foreach \x in {#4} { %
    \draw [-*,thick] (\x) -- (#2-alias); %
  } ;%
}

% \vgate {name} {fitlist-left} {caption-left} {fitlist-right}
% {caption-right} {inputs}
\newcommand{\vgate}[6]{ %
  % Wrap the left and right parts
  \node[wrap=#2] (#1-left) {}; %
  \node[wrap=#4] (#1-right) {}; %
  % Draw the gate
  \node[gate=(#1-left)(#1-right)] (#1) {}; %
  % Add captions
  \node[caption, below left=of #1.north ] (#1-left-caption)
  {#3}; %
  \node[caption, below right=of #1.north ] (#1-right-caption)
  {#5}; %
  % Draw middle separation
  \draw [-, dashed] (#1.north) -- (#1.south); %
  % Draw inputs
  \foreach \x in {#6} { %
    \draw [-*,thick] (\x) -- (#1); %
  } ;%
}

% \hgate {name} {fitlist-top} {caption-top} {fitlist-bottom}
% {caption-bottom} {inputs}
\newcommand{\hgate}[6]{ %
  % Wrap the left and right parts
  \node[wrap=#2] (#1-top) {}; %
  \node[wrap=#4] (#1-bottom) {}; %
  % Draw the gate
  \node[gate=(#1-top)(#1-bottom)] (#1) {}; %
  % Add captions
  \node[caption, above right=of #1.west ] (#1-top-caption)
  {#3}; %
  \node[caption, below right=of #1.west ] (#1-bottom-caption)
  {#5}; %
  % Draw middle separation
  \draw [-, dashed] (#1.west) -- (#1.east); %
  % Draw inputs
  \foreach \x in {#6} { %
    \draw [-*,thick] (\x) -- (#1); %
  } ;%
}



\newcommand{\vm}[1]{\mathbf{#1}}
\newcommand{\N}{\mathcal{N}}
\newcommand\hfrac[2]{\genfrac{}{}{0pt}{}{#1}{#2}} %\frac{}{} sin la linea del medio

\usepackage{listings}
\lstset{
  aboveskip=3mm,
  belowskip=3mm,
  showstringspaces=true,
  columns=flexible,
  basicstyle={\footnotesize\ttfamily},
  breaklines=true,
  breakatwhitespace=true,
  tabsize=4,
  showlines=true
}


%% -- LaTeX packages and custom commands ---------------------------------------

%% recommended packages
\usepackage{thumbpdf,lmodern}

%% another package (only for this demo article)
\usepackage{framed}

%% new custom commands
\newcommand{\class}[1]{`\code{#1}'}
\newcommand{\fct}[1]{\code{#1()}}

 
%% -- Article metainformation (author, title, ...) -----------------------------

%% - \author{} with primary affiliation
%% - \Plainauthor{} without affiliations
%% - Separate authors by \And or \AND (in \author) or by comma (in \Plainauthor).
%% - \AND starts a new line, \And does not.
\author{Gustavo Landfried \\Universidad de Buenos Aires
   \And Matias Mazzanti \\Universidad de Buenos Aires
   \And Esteban Mocskos \\Universidad de Buenos Aires}
\Plainauthor{Gustavo Landfried, Matias Mazzanti, Esteban Mocskos }

%% - \title{} in title case
%% - \Plaintitle{} without LaTeX markup (if any)
%% - \Shorttitle{} with LaTeX markup (if any), used as running title
\title{TrueSkill Through Time: the \proglang{Julia}, \proglang{Python} and \proglang{R} packages}
\Plaintitle{TrueSkill Through Time: the Julia, Python and R packages}
\Shorttitle{TrueSkill Through Time: the \proglang{Julia}, \proglang{Python} and \proglang{R} packages (Draft)}

%% - \Address{} of at least one author
%% - May contain multiple affiliations for each author
%%   (in extra lines, separated by \emph{and}\\).
%% - May contain multiple authors for the same affiliation
%%   (in the same first line, separated by comma).
\Address{
  Gustavo Andr\'es Landfried\\
  Departamento de Computaci\'on\\
  Facultad de Ciencias Exactas y Naturales\\
  Universidad de Buenos Aires\\
  Buenos Aires, Argentina\\
  E-mail: \texttt{gustavolandfried@gmail.com}\\

  \vspace{0.3cm}

  Matias Mazzanti\\
  Departamento de F\'isica\\
  Facultad de Ciencias Exactas y Naturales\\
  Universidad de Buenos Aires\\
  Buenos Aires, Argentina\\

  \vspace{0.3cm}

  Esteban Mocskos\\
  Departamento de Computaci\'on\\
  Facultad de Ciencias Exactas y Naturales\\
  Universidad de Buenos Aires\\
  Buenos Aires, Argentina\\
  \emph{and}\\
  Departamento de Computaci\'on\\
  Facultad de Ciencias Exactas y Naturales\\
  Universidad de Buenos Aires\\
  Buenos Aires, Argentina\\
}




























%% - \Abstract{} almost as usual
\Abstract{
 \en{Humans develop complex skills through time.}
 \es{Los humanos desarrollan habilidades complejas a trav\'es del tiempo.}
 %
 \en{Knowing how they change is crucial in many areas, such as educational systems and the video game industry.}
 \es{Saber c\'omo cambian es crucial en muchas \'areas, como los sistemas educativos y la industria de los videojuegos.}
 %
 \en{All widely used skill estimators share the same underlying causal model, but differ in their inferential procedure.}
 \es{Todos los estimadores de habilidad ampliamente utilizados comparten el mismo modelo causal subyacente, pero se diferencian en sus procedimientos metodol\'ogicas.}
 %
 \en{The success of TrueSkill was based on the application of an efficient algorithm to find the best approximation of the exact posterior.}
 \es{El \'exito del TrueSkill se bas\'o en la aplicaci\'on de un algoritmo eficiente para encuentrar la mejor aproximaci\'on del posterior exacto.}
 %
 \en{TrueSkill Through Time is a major enhancement that uses a single generative model for the entire history of activities, rather than a sequence of independent observations, allowing historical information to propagate throughout the system, resulting in better estimates with even less data.}
 \es{TrueSkill Through Time utiliza un \'unico modelo generativo para toda la historia de las actividades, en lugar de una secuencia de observaciones independientes, lo que permite que la informaci\'on hist\'orica se propague por todo el sistema, lo que da lugar a mejores estimaciones con a\'un menos datos.}
 %
 \en{The use of an efficient algorithm, that requires only a few linear iterations over the data, allows scaling to millions of observations in few minutes.}
 \es{El uso de un algoritmo eficiente, que requiere s\'olo unas pocas iteraciones lineales sobre los datos, permite escalar a millones de observaciones en pocos minutos.}
 %
 \en{This paper offer the first packages for \proglang{Julia}, \proglang{Python}, and \proglang{R} together with its scientific documentation.}
 \es{Mediante este art\'iculo ofrecemos los primeros paquetes para \proglang{Julia}, \proglang{Python} y \proglang{R}, junto con su documentaci\'on cient\'ifica.}
}
\Keywords{Learning, skill, inference, gaming, education, sports, \proglang{Julia}, \proglang{Python},  \proglang{R}}
\Plainkeywords{}


% Para fijar que la siguiente cita la incluya como primer autor y et al. Usar con citas de muchos autores.
\shortcites{Koster2020}
\shortcites{Herrmann2007}
\shortcites{Kschischang2001}
\shortcites{Herbrich2007}
\shortcites{Dangauthier2007}
% \shortcites{jaynes2003-bookProbabilityTheory}
\shortcites{VanHorn2003}
\shortcites{maystre2019-pairwise}
\shortcites{Bishop2006}

\begin{document}
%\lstset{language=Python}

\section[Introduction]{Introduction: } \label{sec:intro}

\en{Humans develop complex skills because of an special integration of biological, cognitive and social processes~\citep{Koster2020}.}
\es{Los humanos desarrollan habilidades complejas gracias a una integraci\'on especial de los procesos biol\'ogicos, cognitivos y sociales~\citep{Koster2020}.}
%
\en{An exceptional cognitive ability to imitate, combined with long periods of juvenile dependency and postreproductive life span, allows humans to learn things from others and transmit innovations through generations~\citep{Richerson2020}.}
\es{Una extraordinaria capacidad para imitar, combinada con los largos per\'iodos de aprendizaje juvenil y vida posreproductiva, permite a los humanos aprender de los dem\'as y transmitir las innovaci\'on a trav\'es de la generaciones~\citep{Richerson2020}.}
%
\en{As a population-based process, human adaptation is also affected by demographic characteristics, such as the size and structure of populations~\citep{Derex2020}.}
\es{Al ser un proceso poblacional, la adaptaci\'on humana tambi\'en se ve afectada por caracter\'isticas demogr\'aficas, como el tama\~no y estructura de las poblaciones~\citep{Derex2020}.} 

%

\en{Knowing how individual skills change over time is essential in Sports and Education.}
\es{Conocer c\'omo cambian las habilidades individuales a lo largo del tiempo es esencial para en contextos deportivos y educativos.}
%
\en{Since the skill is a hidden variable, the best we can do is to estimate it based on its direct observable consequences: the outcome of problem solving and competitions.}
\es{Dado que son variables ocultas, lo mejor que podemos hacer es estimarlas a partir de sus consecuencias observables directas: el producto de resoluci\'on de problemas y competencias.}
%
\en{Considering only the frequency of positive results as an indicator of the individuals' ability could lead to wrong approximations, mainly because the outcome also depends on the difficulty of the challenge.}
\es{Considerar s\'olo la frecuencia de resultados positivos como indicador de la habilidad de los individuos puede conducir a aproximaciones erroneas, fundamentalmente porque su valor depende tambi\'en de la dificultad de los desaf\'ios.}
%
\en{For this reason, all widely used skill estimators are based on pairwise comparisons.}
\es{Por esta raz\'on, todos los estimadores de habilidad ampliamente usados se basan en comparaciones por pares.}
%
\en{Since the first generative models, proposed almost a century ago by~\cite{Thurstone1927} and~\cite{Zermelo1929}, it is assumed that the probability of the observed result $r$ depends on the performance $p$ of the agent $i$ and their opponent $j$, expressed as $P(\, r \,|\, p_i, \, p_j \,)$.}
\es{Desde los primeros modelos generativos, propuestos hace casi un siglo por~\cite{Thurstone1927} y~\cite{Zermelo1929}, se supone que la probabilidad de un resultado observado, $r$, depende del rendimiento, $p$, del agente $i$ y de su oponente $j$, expresada como $P(\, r \,|\, p_i, \, p_j \,)$.}
%
\en{The field continued to progress with the work of~\cite{Bradley1952} and~\cite{Mosteller1951a,Mosteller1951b,Mosteller1951c}, leading to the breakthrough that took place with the methodology developed by~\cite{Elo2008} for the US Chess Federation (USCF), which is still used by the International Chess Federation (FIDE).}
\es{El campo sigui\'o progresando con los trabajos de \cite{Bradley1952} y~\cite{Mosteller1951a,Mosteller1951b,Mosteller1951c}, que condujo al gran avance que tuvo lugar con la metodología desarrollada por~\cite{Elo2008} para la Federaci\'on de Ajedrez de los Estados Unidos (USCF), adoptada hasta el d\'ia de hoy por la Federaci\'on Internacional de Ajedrez (FIDE).}

% Parrafo

\en{All currently used skill estimators share some variant of the probabilistic model proposed by Elo~\citep{Glickman1999,Herbrich2007,VanDerLinden2016,Fox2010}.}
\es{Todos los estimadores de habilidad ampliamente utilizados actualmente comparten alguna variante del modelo probabil\'istico propuesto por Elo~\cite{Glickman1999, Herbrich2007, VanDerLinden2016, Fox2010}.}
%
\begin{figure}[h!]
\centering \small
    \tikz{         
    \node[det, fill=black!10] (r) {$r$} ; 
    \node[const, left=of r, xshift=-1.35cm] (r_name) {\small \en{Result}\es{Resultado}:}; 
    \node[const, right=of r] (dr) {\normalsize $ r = (d > 0)$}; 

    \node[latent, above=of r, yshift=-0.45cm] (d) {$d$} ; %
    \node[const, right=of d] (dd) {\normalsize $ d = p_i-p_j$}; 
    \node[const, left=of d, xshift=-1.35cm] (d_name) {\small \en{Difference}\es{Diferencia}:};
    
    \node[latent, above=of d, xshift=-0.8cm, yshift=-0.45cm] (p1) {$p_i$} ; %
    \node[latent, above=of d, xshift=0.8cm, yshift=-0.45cm] (p2) {$p_j$} ; %
    \node[const, left=of p1, xshift=-0.55cm] (p_name) {\small \en{Performance}\es{Rendimiento}:}; 

    \node[accion, above=of p1,yshift=0.3cm] (s1) {} ; %
    \node[const, right=of s1] (ds1) {$s_i$};
    \node[accion, above=of p2,yshift=0.3cm] (s2) {} ; %
    \node[const, right=of s2] (ds2) {$s_j$};
    
    \node[const, right=of p2] (dp2) {\normalsize $p \sim \N(s,\beta^2)$};

    \node[const, left=of s1, xshift=-.85cm] (s_name) {\small \en{Skill}\es{Habilidad}:}; 
    
    \edge {d} {r};
    \edge {p1,p2} {d};
    \edge {s1} {p1};
    \edge {s2} {p2};
    %\node[invisible, right=of p2, xshift=4.35cm] (s-dist) {};
}
     \caption{
     \en{Generative model in which skills cause the observable results mediated by the difference of hidden performances, $d =p_i - p_j$, both random variables around their unknown true skill, $p \sim \N(s,\beta^2)$.}
    \es{Modelo generativo en el que las habilidades causan los resultados observables a trav\'es de la diferencia de rendimientos ocultos, $d=p_i-p_j$, ambas variables aleatorias centradas en la verdadera habilidad, $p \sim \N(s,\beta^2)$}
    %
    \en{The one with the highest performance wins, $r = (d > 0)$.}
    \es{Gana quien haya obtenido mayor rendimiento, $r = (d > 0)$.}
    %
    \en{Observable variables are painted gray, hidden in white, and constants are shown as black dots.}
    \es{Las variables observables se pintan de gris, la ocultas en blanco, y las constantes se muestran como puntos negros.}
    }
    \label{fig:generative_model}
\end{figure}
%
\en{Figure~\ref{fig:generative_model} provides a causal model in which skills generates the observable result.}
\es{La figura~\ref{fig:generative_model} ofrece un modelo causal en la que las habilidades generan el resultado observable.}
%
\en{The agents exhibit different performances at each event, varying around their true skill, $\N(p_i\,|\,s_i,\beta^2)$.}
\es{Los agentes exhiben distintos desempe\~nos en cada evento, que var\'ian alrededor de su verdadera habilidad, $\N(p\,|\,s,\beta^2)$.}
%
\en{The model assumes that the agent with the highest performance wins, $r = (p_i > p_j)$. In other words, whoever obtains a difference of performance greater than 0 wins, $r = (p_i - p_j > 0)$.}
\es{El modelo supone que gana el agente con mayor rendimiento, $r = (p_i > p_j)$, o en otras palabras, gana quien obtenga una diferencia de rendimiento mayor a 0, $r = (p_i - p_j > 0)$.}
\en{The parameter $\beta^2$ is the same for all the agents and acts as the scale of the estimates: skills separated by a $\beta$ keep always the same probability of winning.}
\es{El parámetro $\beta^2$, al ser el mismo para todos los agentes, act\'ua como la escala de las estimaciones: habilidades separada por un $\beta$ mantienen siempre la misma probabilidad de ganar.}

% Parrafo

\en{Using the graphical representation proposed in Fig.~\ref{fig:generative_model}, it is possible to derive the prediction of the result given the previous estimates, $P(\,r\,|\,s_{i_\text{old}},s_{j_\text{old}}\,)$.}
\es{A partir de su representaci\'on gr\'afica se puede derivar entre otras, la predicci\'on del resultado dadas las estimaciones previas, $P(\,r\,|\,s_{i_\text{old}},s_{j_\text{old}}\,)$.}
%
\en{Elo's methodological solution is simple and smart: updates both previous estimates, $s_{i_\text{old}}$ and $s_{j_\text{old}}$, based on the surprise (i.e. the complement of the prediction of the observed result).}
\es{La soluci\'on metodol\'ogica de Elo es simple y astuta: actualizar las estimaciones previas, $s_{i_\text{old}}$ y $s_{j_\text{old}}$, en base a la sorpresa (i.e. el complemento de la predicci\'on del resultado observado).}
%
\begin{equation} \label{eq:elo_delta}
 \Delta = \underbrace{\left(1-P(\,r\,|\,s_{i_\text{old}},s_{j_\text{old}}\,)\right)}_{\text{\en{Surprise}\es{Sorpresa}}}
\end{equation}
%
\en{where the probability arises from instantiating the previous estimates, $s_{i_\text{old}}$ and $s_{j_\text{old}}$, in the generative model (more details in section~\ref{sec:exactSolution}).}
\es{Donde la probabilidad surge de instanciar las estimaciones previas en el modelo generativo (detalles en la secci\'on~\ref{sec:exactSolution}).}
%
\en{The idea is that the magnitude of the surprise $\Delta$ is related to the accuracy of the previous estimates and, therefore, can be used to update them.}
\es{La idea es que la magnitud de la sorpresa $\Delta$ est\'a relacionada con cuan buenas son las estimaciones previas, y por lo tanto puede usarse para actualizarlas.}
%
\en{Unexpected results would indicate that current estimates should be updated to a greater extent than if they had occurred as expected.}
\es{Los resultados inesperados indicar\'ian que las estimaciones actuales no son del todo correctas y deber\'ian actualizarse en mayor medida que si hubieran ocurrido como se esperaba.}
%
\begin{equation}\label{eq:elo_update}
 s_{\text{winner}_\text{new}} = s_{\text{winner}_\text{old}} + \Delta \ \ \ \ \ s_{\text{loser}_\text{new}} = s_{\text{loser}_\text{old}} - \Delta 
\end{equation}
%
\en{where the surprise $\Delta$ acts as a correction factor for both previous estimates.}
\es{Donde la sorpresa $\Delta$ act\'ua como factor de correcci\'on para ambas estimaciones previas.}
%
\en{This procedure can recover the relative scale of the agents, starting from arbitrary initial values.}
\es{Esta soluci\'on puede recuperar la escala relativa de los agentes, partiendo de de valores iniciales arbitrarios.}
%
\en{However, it does have a major weakness.}
\es{Sin embargo, tiene algunas debilidades importantes.}
%
\en{The update rule (Eq.~\ref{eq:elo_update}) is symmetric: what one agent wins is lost by the other.}
\es{La regla de actualizaci\'on (Eq.~\ref{eq:elo_update}) es sim\'etrica, as\'i que lo que un agente gana el otro lo pierde.}
%
\en{Because new agents start with arbitrary skills (the same initial value is used for all individuals), they tend to generate greater surprise values and abruptly modify estimates that had already converged.}
\es{Debido a que a los agentes nuevos comienzan con estimaciones arbitrarias (el mismo valor inicial para cualquier individuo), ellas tienden a generan alta sorpresa y a modificar bruscamente estimaciones que ya hab\'ian convergido.}
%
\en{This weakness emerges because the uncertainties of the agent's estimates are not considered.}
\es{Esta debilidad ocurre porque no se tiene en cuenta la incertidumbre sobre las estimaciones de los agentes.}
%
\en{An ad-hoc solution was proposed to solve this problem: reducing the impact of the surprise based on the number of times the agent has participated previously.}
\es{Para resolver este problema, una soluci\'on ad-hoc fue propuesta: reducir el impacto de la sorpresa en funci\'on de la cantidad de veces que el agente ha participado previamente.}
%
% \en{This is the role played by the K-factor used by FIDE,  $\Delta_i = \Delta \cdot K_i$.}
% \es{Ese es rol que desempe\~na el K-factor usado por la FIDE, $\Delta_i = \Delta \cdot K_i$.}

% Cambio de parrafo

\en{Instead of selecting a single value as an estimate, an enhancement consists of distributing the certainty among all possible skill hypotheses.}
\es{En vez de seleccionar un \'unico valor como estimaci\'on, una alternativa superadora consiste distribuir la certidumbre entre todas las posibles hip\'otesis de habilidad.}
%
\en{This approach known as \emph{Bayesian inference} has proven successful in practice~\citep{Bishop2006} and ensures consistent reasoning when handling uncertainty (plausible beliefs)~\citep{Jaynes2003,VanHorn2003}.}
\es{Este enfoque conocido como inferencia bayesiana ha demostrado ser exitoso en la pr\'actica~\citep{Bishop2006} y garantiza un razonamiento consistente en el manejo de la incertidumbre (o creencias plausibles)~\citep{Jaynes2003,VanHorn2003}.}
%
\en{Any inference, independently of its complexity, can be solved by the rules of probability: the~\ref{eq:sum_rule} and the~\ref{eq:product_rule}.}
\es{Toda inferencia, no importa cuan compleja sea, puede ser resuelta mediante las reglas de la probabilidad: la~\ref{eq:sum_rule} y la~\ref{eq:product_rule}.}
%
\en{The \ref{eq:sum_rule} states that any marginal distribution can be obtained by integrating, or summing up, the joint distribution:}
\es{La \ref{eq:sum_rule} afirma que cualquier distribuci\'on marginal se puede obtener integrando o sumando la distribuci\'on conjunta.}
%
\begin{equation} \label{eq:sum_rule}
 \tag{\en{sum rule}\es{regla de la suma}}
 P(x) = \sum_{y} P(x,y) \ \ \ \ \ \text{or} \ \ \ \ \ p(x) = \int p(x,y) \, dy
\end{equation}
%
\en{where $ P(\cdot)$ and $p(\cdot)$ represent discrete and continuous probabilities distributions respectively.}
\es{Donde $P(\cdot)$ y $p(\cdot)$ representan distribuciones de probabilidad discretas y continuas respectivamente.}
%
\en{Additionally, the \ref{eq:product_rule} states that any joint distribution can be expressed as the product of one-dimensional conditional distributions.}
\es{Adem\'as, la \ref{eq:product_rule} se\~nala que cualquier distribuci\'on conjunta puede ser expresada como el producto de distribuciones condicionales uni-dimensionles.}
%
\begin{equation}\label{eq:product_rule}
\tag{\en{product rule}\es{regla del producto}}
 p(x,y) = p(x|y) p(y)
\end{equation}
%
\en{From these rules, we immediately obtain the~\ref{eq:bayes_theorem}:}
\es{De estas reglas obtenemos inmediatamente el~\ref{eq:bayes_theorem},}
%
\begin{equation}\label{eq:bayes_theorem}
\tag{\en{Bayes' theorem}\es{Teorema de bayes}}
 p(y|x) = \frac{p(x|y)p(y)}{p(x)}
\end{equation}
%
% \en{The inferential use of Bayes' theorem plays a central role in modern statistical learning techniques.}
% \es{El uso inferencial del teorema de bayes juega un rol central en las t\'ecnicas modernas de aprendizaje estad\'isitico.}
%
\en{The \ref{eq:bayes_theorem} allows us to optimally update our beliefs about the hypotheses, given a model and the data.}
\es{El \ref{eq:bayes_theorem} permite actualizar de forma \'optima las creencia sobre las hip\'otesis, dado un modelo y los datos.}
%
\en{In our case, to quantify the relative certainty of the skill hypotheses using the information provided by the observed result and the described causal model, we need to solve:}
\es{En nuestro caso, para cuantificar la certidumbre relativa de las hip\'otesis de habilidades utilizando la informaci\'on que nos ofrece el resultado observado y el modelo causal descrito, necesitamos resolver:}
%
\begin{equation}\label{eq:event_inference} 
 \underbrace{p(\overbrace{\text{\en{Skill}\es{Habilidad}$_i$}}^{\text{\en{Hidden}\es{Oculta}}}|\overbrace{\text{Result\es{ado}}}^{\text{Observ\en{ed}\es{ado}}}, \text{Model\es{o}})}_{\text{Posterior}} = \frac{\overbrace{P(\,\text{Result\es{ado}}\,|\,\text{\en{Skill}\es{Habilidad}$_i$}\,,\text{Model\es{o}})}^{\text{\en{Likelihood}\es{Verosimilitud}}}\overbrace{p(\text{\en{Skill}\es{Habilidad}$_i$})}^{\text{Prior}}}{\underbrace{P(\text{Result\es{ado}}\,|\,\text{Model\es{o}})}_{\text{Evidenc\en{e}\es{ia} o\en{r}\es{ predicci\'on a} prior \en{prediction}}}}
\end{equation}
%
\en{where the only free variable is the skill hypothesis of agent $i$.}
\es{Donde la \'unica variable libre es la hip\'otesis de habilidad del agente $i$.}
%
\en{The prior quantifies the uncertainty about the ability, and the posterior contains the uncertainty that remains after seeing the new data (i.e.~the result of the event).}
\es{El prior cuantifica la incertidumbre sobre la habilidad, y el posterior contiene la incertidumbre que queda luego de ver el nuevo dato (i.e.~el resultado del evento).}
%
\en{The likelihood and the evidence are both probabilities of the observed result, so they can be seen as predictions (since the results are discrete variables, those probabilities are expressed using capital letters).}
\es{La verosimilitud como la evidencia son ambas probabilidades del resultado observado, por lo que pueden ser vistas como predicciones (como los resultados son variables discretas, esas probabilidad se escribe con letras may\'usculas).}
%
\en{Because the evidence is the same for all hypotheses, the only factor that updates our beliefs is the likelihood.}
\es{Debido a que la evidencia es la misma para todas las hip\'otesis, el \'unico factor que actualiza nuestras creencias es la verosimilitud.}

% Parrafo

\en{As an instance, let's consider a winning case ($p_i > p_j$) using a Gaussian prior (i.e. $\N(\,s\,|\,\mu, \sigma^2)$) for each of the skills.}
\es{A modo de ejemplo, consideremos un caso ganador ($p_i > p_j$) usando priors Gaussianos, $\N(\,s\,|\,\mu, \sigma^2)$.}
%
\en{The difference of performances, $d=p_i-p_j$, can also be expressed as a Gaussian distribution, centered on the difference of the prior estimates ($\mu_i -\mu_j$), with a variance that incorporates the uncertainty of both estimates ($\sigma_i$ and $\sigma_j$) and the variance of both performances ($\beta$), $\N( d \, | \, \mu_i -\mu_j \, ,\ 2\beta^2 + \sigma_i^2 + \sigma_j^2 \,)$.}
\es{La diferencia de rendimientos, $d=p_i-p_j$, tambi\'en se puede expresar como una gaussiana centrada en la diferencia de las medias de las estimaciones a priori ($\mu_i - \mu_j$), con una varianza que incorpora la incertidumbre de ambas estimaciones ($\sigma$) y la varianza de ambos rendimientos ($\beta$), $\N(\, d \, | \, \mu_i -\mu_j \, ,\ 2\beta^2 + \sigma_i^2 + \sigma_j^2 \,)$.}
%
\en{As we observed that the agent $i$ won, we know from the causal model that the hidden difference of performances was positive.}
\es{Como observamos que el agente $i$ gan\'o, sabemos por el modelo causal que la diferencia de rendimientos oculta fue en efecto positiva.}
%
\en{Therefore, the prior prediction of the observed result (or evidence) is the cumulative density, $\Phi$, of all positive values of the difference of performances (Eq.~\ref{eq:evidence}).}
\es{Por lo tanto, la predicci\'on a priori del resultado observado (o evidencia) es la densidad acumulada, $\Phi$, de todos los valores positivos de la diferencia de rendimientos (Eq.~\ref{eq:evidence}).}
%
\en{During the rest of this work, the role of the model will be left implicit.}
\es{A partir de ahora el rol del modelo se dejr\'a impl\'icito.}
%
\begin{equation}\label{eq:evidence}
 \overbrace{P(r)}^{\text{Evidenc\en{e}\es{ia}}} = 1-\Phi(0 \, | \overbrace{\mu_i^{\phantom{2}} - \mu_j}^{\hfrac{\text{\en{Expected}\es{Diferencia}}}{\text{\en{difference}\es{esperada}}}} , \, \overbrace{2\beta^2 + \sigma_i^2+ \sigma_j^2}^{\hfrac{\text{\en{Total}\es{Incertidumbre}}}{\text{\en{uncertainty}\es{total}}}})
\end{equation}
%
\en{The evidence is a prediction made with all the prior hypotheses.}
\es{La evidencia es una predicci\'on hecha con todas las hip\'otesis a priori.}
%
\en{Since the evidence is a constant value, the posterior uncertainty of each hypothesis is proportional to the product of their prior uncertainty and their likelihood, as shown in equation~\ref{eq:posterior_win}.}
\es{Como la evidencia es constante, la incertidumbre a posteriori de cada hip\'otesis es proporcional al producto de su incertidumbre a priori y su verosimilitud, como se muestra en la ecuaci\'on~\ref{eq:posterior_win}.}
%
\en{Section~\ref{sec:exactSolution} shows how these expressions can be derived applying the sum and product rule.}
\es{En la secci\'on~\ref{sec:exactSolution} veremos en detalle c\'omo todas estas ecuaciones surge de aplicar las reglas de las suma y el producto sobre el modelo.}
%
\begin{equation}\label{eq:posterior_win}
\underbrace{p(\,s_i\, | \, r \, )}_{\text{Posterior}} \propto \underbrace{1-\Phi(0 \, |  s_i - \mu_j , \, 2\beta^2 + \sigma_j^2)}_{\text{\en{Likelihood}\es{Verosimilitud}} \ P(r|s_i)} \,  \underbrace{\N(s_i \, | \, \mu_i,\, \sigma_i^2)}_{\text{Prior} \ p(s_i)} 
\end{equation}
%
% \begin{equation}\label{eq:posterior_win}\tag{\text{mejor esta?}}
% \textcolor{black!60}{
% \underbrace{p(\,s_i\, | \, r \, )}_{\text{Posterior}} = \frac{ \overbrace{1-\Phi(0 \, | s_i^{\textcolor{white}{2}} - \mu_j , \, 2\beta^2 + \sigma_j^2)}^{\text{\en{Likelihood}\es{Verosimilitud}}} \,  \overbrace{\N(s_i \, | \, \mu_i,\, \sigma_i^2)}^{\text{Prior}} }{\underbrace{1-\Phi(0 \, | \mu_i - \mu_j , \, 2\beta^2 + \sigma_i^2+ \sigma_j^2)}_{\text{Evidenc\en{e}\es{ia}}}}
% }
% \end{equation}
%
\en{The normalized posterior can be found dividing the right hand by the evidence $P(r)$.}
\es{Donde el el posterior normalizado se obtiene dividiendo el lado derecho por la evidencia, $P(r)$.}
%
\en{It is interesting to note the similarities and differences between likelihood and evidence.}
\es{Es interesante notar las similitudes y diferencias entre el likelihood y el evidencia.}
%
\en{The likelihood quantifies the same cumulative density as the evidence but centered at the difference between the hypothesis we are evaluating $s_i$ and the opponent's mean estimate $\mu_j$, with a variance that includes all uncertainties except the one of $s_i$.}
\es{La verosimilitud cuantifica la misma densidad acumulada que la evidencia, pero centrada ahora en la diferencia entre la hip\'otesis que estamos evaluando $s_i$ y la estimaci\'on media del oponente $\mu_j$ con una varianza que incluye todas las incertidumbres salvo la de la propia hip\'otesis $s_i$.}
%
\en{In other words, the likelihood is just the prior prediction of the observed result assuming true the skill hypothesis we are evaluating.}
\es{En otras palabras, la verosimilitud no es m\'as que la predicci\'on a priori del resultado observado suponiendo verdadera la hip\'otesis de habilidad que estamos evaluando.}
%, made with all the skill opponent's hypotheses,
% \en{For each hypothesis $s_i$, the filtered density is exactly proportional to the magnitude of the surprise, defined as the complement of the likelihood.}
% \es{Para cada hip\'otesis $s_i$, la masa filtrada es exactamente proporcional a la magnitud de la sorpresa, definida como el complemento de la verosimilitud.}
%
\en{Figure~\ref{fig:posterior_win} shows, in graphical terms, the updating procedure executed in equation~\ref{eq:posterior_win}.}
\es{La figura~\ref{fig:posterior_win} muestra, en t\'erminos gr\'aficos, el procedimiento de actualizaci\'on que se realiza en la ecuaci\'on~\ref{eq:posterior_win}.}
%
\begin{figure}[H]
    \centering
    \en{\includegraphics[page={1},width=.6\linewidth]{figures/posterior_win}}
    \es{\includegraphics[page={2},width=.6\linewidth]{figures/posterior_win}}
    \caption{
    %
    \en{Belief update for the winning case.}
    \es{Actualizaci\'on de creencias para el caso ganador.}
    %
    \en{The proportinal posterior is obtained as the product of the prior (Gaussian) and the likelihood (cumulative Gaussian).}
    \es{El posterior proporcional se obtiene como el producto de la distribuci\'on a priori (distribuci\'on gaussiana) y la verosimilitud (distribuci\'on gaussiana acumulada).}
    %
    \en{The evidence is the integral of the proportional posterior.}
    \es{La evidencia es la integral del posterior proporcional.}
    %
    \en{The distributions are not neccesary on the same scale: the prior integrates 1, while the likelihood goes from 0 to 1.}
    \es{Las distribuciones no est\'an necesariamente en la misma escala: la distribuci\'on a priori integra 1, mientras que la verosimilitud va de 0 a 1.}
    }
    \label{fig:posterior_win}
\end{figure}
%
\en{The surprise, defined as the complement of the likelihood, works as a filter for the prior.}
\es{La sorpresa, definida como el complemento de la verosimilitud, funciona como un filtro para el prior.}
%
\en{The posterior is just the prior's density that is not filtered by the likelihood.}
\es{El posterior no es m\'as que la densidad del prior no filtrada por la verosimilitud.}
%    
\en{At the region of very high skill hypotheses, where the winning result would have generated almost no surprise ($\lim_{s_i \to \infty}P(r|s_i) = 1$), the posterior receives all the prior's density.}
\es{En la regi\'on de hip\'otesis de muy alta habilidad, donde el resultado ganador no nos hubiera generado casi ninguna sorpresa ($\lim_{s_i \to \infty}P(r|s_i) = 1$), el posterior recibe casi toda la masa del prior.}
%
\en{At the region of very low skill hypotheses, where the winning result would have generated huge surprise ($\lim_{s_i \to -\infty}P(r|s_i) = 0$), the posterior receive no density from the prior.}
\es{En cambio, en la regi\'on de hip\'otesis de muy baja habilidad, donde el resultado habr\'ia generado mucha sorpresa ($\lim_{s_i \to -\infty}P(r|s_i) = 0$), el posterior no recibe casi nada de la masa del prior.}

% Parrafo

\en{It is important to stress that the posterior, although similar, is not a Gaussian distribution, preventing us from using equation~\ref{eq:posterior_win} iteratively.}
\es{Es importante remarcar que la posterior, aunque se parezca, no es una distribuci\'on gaussiana, lo que nos impedir\'a usar la ecuaci\'on~\ref{eq:posterior_win} iterativamente.}
%
\en{But due to the shape of the exact posterior, a Gaussian distribution could be used as a good approximation, allowing us to avoid the computational cost of the sampling methodologies.}
\es{Por la forma del posterior exacto, una gaussiana puede ser usada como una buena aproximaci\'on, permiti\'endonos evitar el costo computacional de las metodolog\'ias de sampleo.}
%
\en{The main contribution of the Glicko system~\citep{glikman_gliko_2} was the development of an efficient method to approximate the exact posterior using a Gaussian distribution.}
\es{El principal aporte del sistema Glicko~\citep{glikman_gliko_2} fue el desarroll\'o de un m\'etodo eficiente para aproximar la posterior exacta con una distribuci\'on Gasussiana.}
%
\en{However, this method does not guarantee the quality of the approximation used.}
\es{Sin embargo, este m\'etodo no garantiza que la distribuci\'on gaussiana seleccionada sea la que mejor aproxima.}
%
\en{The success of the TrueSkill solution~\citep{Herbrich2007} is based on the application of an efficient method for computing the Gaussian distribution that guarantees the approximation of the posterior (section~\ref{sec:approximate_posterior}).}
\es{El \'exito de la soluci\'on TrueSkill~\citep{Herbrich2007} se basa en la aplicaci\'on de m\'etodo eficiente para calcular la gaussiana que mejor aproxima a la posterior exacta (secci\'on~\ref{sec:approximate_posterior}).}
%
\begin{equation} \label{eq:approx} 
 \widehat{p}(s_i| r, s_j) = \underset{\mu, \sigma}{\text{ arg min }} \ \ \text{KL}(\, p(s_i| r, s_j) \, || \,  \N(s_i|\mu, \sigma^2) \, )
\end{equation}
%
\en{which minimize the Kullback-Leibler divergence between the true and the approximate distribution. This kind of methods allows performing Bayesian inference in situations that otherwise would be impossible.}
\es{Esta clase de m\'etodos, que minizan la divergencia Kullback-Leibler entre la distribuci\'on verdadera y aproximada, permiten realizar inferencia bayesiana en situaciones que de otra forma no ser\'ia posibles.}
%
\en{In our case, in which we need to know how skills change over time, this technique allows us to efficiently apply equation~\ref{eq:posterior_win} iteratively over a sequence of observations.}
\es{En nuestro caso, en el que necesitamos conocer c\'omo cambian las habilidades a trav\'es del tiempo, est\'a t\'ecnica nos permite aplicar eficientemente la ecuaci\'on~\ref{eq:posterior_win} iterativamente sobre una secuencia de observaciones.}

%

\en{The approach adopted by TrueSkill to treat the dynamical process, known as \emph{filtering}, uses the posterior as the prior for the next event.}
\es{El enfoque adoptado por TrueSkill para tratar el proceso din\'amico, conocido como \emph{filtering}, usa el posterior como prior del siguiente evento.}
%
\en{Then, the posterior at any given time is,}
\es{Luego, el posterior en un determinado momento es,}
%
\begin{equation}\label{eq:filter} %\tag{\text{filtering}}
 \widehat{\text{Posterior}}_t \propto \widehat{\text{Likelihood}}_t  \overbrace{\widehat{\text{Likelihood}}_{t-1} \dots \widehat{\text{Likelihood}}_{2} \underbrace{\widehat{\text{Likelihood}}_{1} \text{Prior}_1}_{\widehat{\text{Posterior}}_{1} \text{ \en{as}\es{como} } \text{Prior}_{2}} }^{\widehat{\text{Posterior}}_{t-1} \text{ \en{as}\es{como} } \text{Prior}_{t}} %= \text{Prior}_1 \prod_{i=1}^t \text{Likelihood}_i 
\end{equation}
%
\en{where {\footnotesize $\widehat{\text{Posterior}}_i$} and {\footnotesize $\widehat{\text{Likelihood}}_i$} represents the approximations induced by the equation~\ref{eq:approx} at the $i$-th event.}
\es{Donde {\footnotesize $\widehat{\text{Posterior}}_i$} y {\footnotesize $\widehat{\text{Likelihood}}_i$} representan la aproximaciones inducidas por la ecuaci\'on~\ref{eq:approx} en el $i$-\'esimo evento.}
%
\en{If we consider the likelihood as a filter of the prior, each posterior can be seen as an accumulation of all previous filters.}
\es{Si consideramos la verosimilitud como un filtro del prior, cada posterior puede ser visto como una acumulaci\'on de todos los filtros anterios.}
%
\en{In this way, information propagates from past to future estimates.}
\es{De esta forma, la informaci\'on propaga del estimaciones pasadas hacia futuras.}
%
%\en{Learning is a dynamical process.}
%\es{El aprendizaje es un proceso din\'amico.}
%
\en{Since skills are dynamic variables it is important to incorporate a degree of uncertainty $\gamma$ at each step.}
\es{Debido a que las habilidades son variables din\'amicas es importante agregar alguna incertidumbre $\gamma$ en cada paso.}
%
\begin{equation}\label{eq:dynamic_factor}
 \widehat{p}(s_{i_t}) = \N(s_{i_t} | \mu_{i_{t-1}}, \sigma_{i_{t-1}}^2 + \gamma^2 )
 \end{equation}
 %
\en{Because the filtering approach does not arise from any probabilistic model, it suffers from some problems related to the fact that the information propagates in only one direction through the system.}
\es{Debido a que el enfoque de filtrado no se surge de ning\'un modelo probabil\'istico, sufre de un serie de problemas, todos relacionados con el hecho de que la informaci\'on se propaga en una sola direcci\'on a trav\'es del sistema.}
%
\en{The most obvious is that the beginning of any sequence of estimates always has high uncertainty.}
\es{El m\'as obvio es que el inicio de toda secuencia de estimaciones siempre tiene alta incertidumbre.}
%
\en{But it also suffers from ``temporary'' and ``spatial'' decoupling.}
\es{Pero tambi\'en sufre de desacoplamientos ``temporales'' y ``espaciales''.}
%
\en{Although the relative differences between current estimates within well-connected communities are correct, estimates separated in time and between poorly connected communities are often incorrect.}
%
\es{Aunque la diferencias relativas entre estimaciones contempor\'aneas al interior de comunidades bien conectadas sean correctas, las estimaciones separadas en el tiempo y entre comunidades poco conectadas suelen ser incorrectas.}
%

% Parrafo

\en{To obtain good initial estimates and ensure temporal and spatial comparability we need a causal model of the temporal process that links all historical activities.}
\es{A fin de obtener buenas estimaciones iniciales y garantizar comparbilidad temporal y espacial, necesitamos un modelo causal del proceso temporal que vincule todas las actividades hist\'oricas.}
%
\en{In any video game this typically involves evaluating a complex network made up of millions of events.}
\es{En cualquier video juego esto típicamente implica evaluar una red compleja compuesta por millones de eventos.}
%
% \en{For this reason, figure~\ref{fig:smoothing} shows only the closest neighbors of a sequence of estimates.}
% \es{Por esta raz\'on, la figura~\ref{fig:smoothing} muestra s\'olo los vecinos m\'as pr\'oximos de una secuencia de estimaciones.}
% %
\en{This approach, known as \emph{smoothing}, is the one implemented by TrueSkill Through Time~\citep{Dangauthier2007}.}
\es{Este enfoque, conocido como \emph{smoothing}, es el implementado por TrueSkill Through Time~\citep{Dangauthier2007}.}
%
% \begin{figure}[h!]
%   \centering
%   \scalebox{.8}{
%     \tikz{ %
%       \node[latent] (s0) {$s_{i_0}$} ;
%       
%       \node[latent,right=of s0,xshift=-0.15cm] (s1) {$s_{i_1}$} ;
%       \node[latent, below=of s1,yshift=0.3cm] (p1) {$p_{i_1}$};
%       \node[const, below=of p1, yshift=-0.6cm] (t1) {\LARGE $\hfrac{}{\dots}$} ;
% 
%       \node[latent, right=of s1,xshift=-0.15cm] (s2) {$s_{i_2}$} ;
%       \node[latent, below=of s2,yshift=0.3cm] (p2) {$p_{i_2}$};
%       \node[const, below=of p2, yshift=-0.6cm] (t2) {\LARGE $\hfrac{}{\dots}$} ;
%       
%       \node[const, right=of s2, xshift=0.6cm] (s3) {$\dots$} ;
% 
%       \node[latent, right=of s3,xshift=-0.3cm] (sn) {$s_{i_n}$} ;
%       \node[latent, below=of sn,yshift=0.3cm] (pn) {$p_{i_n}$};
%       \node[const, below=of pn, yshift=-0.6cm] (tn) {\LARGE $\hfrac{}{\dots}$} ;
%       
%       \edge {s0} {s1};
%       \edge {s1} {p1,s2};
%       \edge {s2} {p2,s3};
%       \edge {s3} {sn};
%       \edge {sn} {pn};
%       \edge {p1} {t1};
%       \edge {p2} {t2};
%       \edge {pn} {tn};
%       
%       }  
%   }
%   \caption{
%   \en{Schematic representation of a single causal model for the entire history of events.}
%   \es{Representaci\'on esquem\'atica de un \'unico modelo causal para toda la historia de eventos.}
%   %
%   \en{Only the neighbors of a sequence of estimates are displayed.}
%   \es{S\'olo se muestra los vecinos de una secuencia de estimaciones.}
%   }
%   \label{fig:smoothing}
% \end{figure}
%
\en{By applying the rules of probability over this temporal model, historical information naturally propagate throughout the system, solving the problems of the filtering approach.}
\es{Al aplicar las reglas de la probabilidad sobre este modelo temporal, la informaci\'on hist\'orica propaga naturalmente hacia todo el sistema, resolviendo los problemas del enfoque de filtro.}
%
\en{Excluding the dynamic component, $\gamma = 0$, the prior of an agent $i$ at the $t$-eth event is just the product of all their likelihoods, except that of the $t$-eth event.}
\es{Exluyendo el aspecto din\'amico, $\gamma = 0$, el prior de un agente $i$ en el $t$-\'esimo evento es el producto del todas sus verosimilitudes, salvo la del $t$-\'esimo evento.}
%
\begin{equation}\label{eq:smooth_prior}
 \text{Prior}_{i_t} = \text{Prior}_{i_0} \underbrace{\prod_{k = 1}^{t-1} \text{\en{Likelihood}\es{Verosimilitud}}_{i_k}}_{\text{\en{Past information}\es{Informaci\'on pasada}}} \underbrace{\prod_{k = t + 1}^{T_i} \text{\en{Likelihood}\es{Verosimilitud}}_{i_k}}_{\text{\en{Future information}\es{Informaci\'on futura}}}
\end{equation}
%
\en{where $T_i$ is the total number of events in which the $i$ agent participated, with {\small Prior$_{i_0}$} the initial prior of agent $i$.}
\es{Donde $T_i$ es la cantidad total de eventos del agente $i$, con {\small Prior$_{i_0}$} el prior inicial del agente $i$.}
%
\en{This procedure establishes a mutual dependency between likelihoods, forcing us to implement an iterative algorithm to solve the inference (details in section~\ref{sec:throguthTime}).}
\es{Esto produce una mutua dependencia entre verosimilitudes, oblig\'andonos a implementar un algoritmo iterativo para resolver la inferencia (detalles en la secci\'on~\ref{sec:throguthTime}).}
%
\en{In the first pass, we use the past likelihoods to compute the priors of each event.}
\es{En la primera pasada, utilizamos las verosimilitudes pasadas para calcular las priores de cada evento.}
%
\en{Once all the likelihoods have been obtained, we can perform a new iteration using the recently computed likelihoods.}
\es{Una vez calculadas todas las verosimilitudes, podemos realizar una nueva pasada usando las \'ultimas verosimilitudes disponibles.}
%
%\en{The first likelihoods will necessarily be incorrect because the priors had been partially defined.}
%\es{Las primeras verosimilitudes necesariamente ser\'an incorrectas debido a que los priors habían sido definidos parcialmente.}
%
%\en{But so will be the likelihoods calculated in the second pass, because its outcome will depend partially on the likelihoods calculated in the previous pass.}
%\es{Pero tambi\'en lo ser\'an las verosimilitudes calculadas en la segunda pasada, debido a que su resutado depender\'a parcialmente de las verosimilitudes calculadas en la pasada anterior.}
%
\en{This procedure converges with a few linear iterations over the data.}
\es{Este procedimiento converge con unas pocas iteraciones lineales sobre los datos.}

% Parrafo

\en{TrueSkill Through Time outperforms TrueSkill and others filtering models.}
\es{TrueSkill Through Time supera a TrueSkill y otros modelos de filtrado.}
%
\en{In Figure~\ref{fig:smooth_example} we show the estimation's behavior in a data set with two agents and two events.}
\es{En la figura~\ref{fig:smooth_example} se muestra el comportamiento de las estimaciones en conjunto de datos de dos agentes y dos eventos.}
\begin{figure}[h!]
  \centering
  \scalebox{.9}{
    \tikz{ %
      \node[latent] (s10) {$s_{a_0}$} ;
      %
      \node[latent,  below=of s10,yshift=-1cm] (s11) {$s_{a_1}$} ;
      
      \node[latent, right=of s11, xshift=3cm] (p11) {$p_{a_1}$} ;
      %
      \node[latent, below=of s11,yshift=-1cm] (s12) {$s_{a_2}$} ;
      \node[latent, right=of s12, xshift=3cm] (p12) {$p_{a_2}$} ;
      
      \node[const, right=of p11,xshift=0.5cm] (r1) {$\bm{>}$} ;
      \node[const, above=of r1, yshift=0.3cm] (nr1) {\footnotesize \ \  Observed result} ;
      \node[const, right=of p12,xshift=0.5cm] (r2) {$\bm{<}$} ;
      \node[const, above=of r2, yshift=0.3cm] (nr2) {\footnotesize \ \ Observed result} ;
      
      \node[latent, left=of s10, xshift=13.4cm] (s20) {$s_{b_0}$} ;
      \node[latent, below=of s20,yshift=-1cm] (s21) {$s_{b_1}$} ;
      \node[latent, left=of s21, xshift=-3cm] (p21) {$p_{b_1}$} ;
      
      \node[latent, below=of s21, yshift=-1cm] (s22) {$s_{b_2}$} ;
      \node[latent, left=of s22, xshift=-3cm] (p22) {$p_{b_2}$} ;
      
      
      \edge {s10} {s11};
      \edge {s11} {s12};
      \edge {s20} {s21};
      \edge {s21} {s22};
      \edge {s11} {p11};
      \edge {s12} {p12};
      \edge {s21} {p21};
      \edge {s22} {p22};
      
      \node[const, right=of s10, yshift=0.6cm ] (wp10) {\includegraphics[page={13},width=.125\linewidth]{figures/smoothing}} ;
      \node[const, left=of s20, yshift=0.6cm ] (wp20) {\includegraphics[page={13},width=.125\linewidth]{figures/smoothing}} ;
      
      
      \node[const, left=of s11, yshift=0.6cm ] (post11) {\includegraphics[page={1},width=.125\linewidth]{figures/smoothing}} ;
      \node[const, right=of s11, yshift=0.6cm ] (wp11) {\includegraphics[page={2},width=.125\linewidth]{figures/smoothing}} ;
      \node[const, left=of p11, yshift=0.6cm ] (lh11) {\includegraphics[page={3},width=.125\linewidth]{figures/smoothing}} ;
      
      \node[const, left=of s12, yshift=0.6cm ] (post12) {\includegraphics[page={4},width=.125\linewidth]{figures/smoothing}} ;
      \node[const, right=of s12, yshift=0.6cm ] (wp12) {\includegraphics[page={5},width=.125\linewidth]{figures/smoothing}} ;
      \node[const, left=of p12, yshift=0.6cm ] (lh12) {\includegraphics[page={6},width=.125\linewidth]{figures/smoothing}} ;
      
      
      \node[const, right=of s21, yshift=0.6cm ] (post21) {\includegraphics[page={7},width=.125\linewidth]{figures/smoothing}} ;
      \node[const, left=of s21, yshift=0.6cm ] (wp21) {\includegraphics[page={8},width=.125\linewidth]{figures/smoothing}} ;
      \node[const, right=of p21, yshift=0.6cm ] (lh21) {\includegraphics[page={9},width=.125\linewidth]{figures/smoothing}} ;
      
      
      \node[const, right=of s22, yshift=0.6cm ] (post22) {\includegraphics[page={10},width=.125\linewidth]{figures/smoothing}} ;
      \node[const, left=of s22, yshift=0.6cm ] (wp22) {\includegraphics[page={11},width=.125\linewidth]{figures/smoothing}} ;
      \node[const, right=of p22, yshift=0.6cm ] (lh22) {\includegraphics[page={12},width=.125\linewidth]{figures/smoothing}} ;
      
      \node[const, above=of post11] (npost11) {\scriptsize Posterior} ;
      \node[const, above=of wp11] (nwp11) {\scriptsize Prior} ;
      \node[const, above=of lh11] (nlh11) {\scriptsize \en{Likelihood}\es{Verosimilitud}} ;
      \node[const, above=of post21] (npost21) {\scriptsize Posterior} ;
      \node[const, above=of wp21] (nwp21) {\scriptsize Prior} ;
      \node[const, above=of lh21] (nlh21) {\scriptsize \en{Likelihood}\es{Verosimilitud}} ;
      
      \node[const, above=of post12] (npost12) {\scriptsize Posterior} ;
      \node[const, above=of wp12] (nwp12) {\scriptsize Prior} ;
      \node[const, above=of lh12] (nlh12) {\scriptsize \en{Likelihood}\es{Verosimilitud}} ;
      \node[const, above=of post22] (npost22) {\scriptsize Posterior} ;
      \node[const, above=of wp22] (nwp22) {\scriptsize Prior} ;
      \node[const, above=of lh22] (nlh22) {\scriptsize \en{Likelihood}\es{Verosimilitud}} ;
      
      \node[const, above=of wp10,yshift=-0.35cm] (nwp10) {\scriptsize Prior} ;
      \node[const, above=of wp20,yshift=-0.35cm] (nwp20) {\scriptsize Prior} ;
      
      }  
  }
  \caption{
  \en{The convergence of TrueSkill Through Time in a history consisting of two events and two agents.}
  \es{Convergencia de TrueSkill Through Time en una historia con s\'olo dos eventos y dos agentes.}
  %
  \en{The first game is won by player $a$, $p_{a_1} > p_{b_1}$, and the second one is won by player $b$, $p_{a_2} < p_{b_2}$.}
  \es{La primera partida la gana el jugador $a$, $p_{a_1} > p_{b_1}$, y la segunda la gana el jugador $b$, $p_{a_2} < p_{b_2}$.}
  %
  \en{The brightness of the curves indicates the order of convergence: the darkest are the latest estimates.}
  \es{La luminosidad de las curvas indican el orden en la convergencia.}
  %
  \en{The first one (the clearest) corresponds to the TrueSkill solution.}
  \es{La primera (la m\'as clara) se correponde a la soluci\'on TrueSkill.}
  %
  \en{The last one (the darkest) corresponds to the TrueSkill Through Time solution.}
  \es{La \'ultima (la m\'as oscura) se correponde a la soluci\'on TrueSkill Through Time.}
  %
  \en{Once convergence is reached, the posterior is centered at 0 for both players, indicating that they have the same skill.}
  \es{Alcanzada la convergencia, el posterior se centra en 0 para ambos jugadores, indicando que tienen misma habilidad.}
  }
  \label{fig:smooth_example}
\end{figure}
%
\en{TrueSkill Through Time recovers the actual differences between hidden skills even at the beginning of the sequence.}
\es{TrueSkill Through Time recupera las verdaderas diferencias entre habilidades ocultas incluso al principio de la secuencia.}
%  
\en{Similar methodologies to TrueSkill Through Time have been developed by \cite{coulom2008-wholeHistoryRating} and \cite{maystre2019-pairwise}.}
\es{Metodologías similares a TrueSkill Through Time han sido desarrollads por \cite{coulom2008-wholeHistoryRating} y \cite{maystre2019-pairwise}.}
%
%\en{This methodology has not been available until now in any of the major programming languages.}
%\es{Esta metodolog\'ia no hab\'ia estado hasta ahora disponible en ninguno de los principales lenguajes de programaci\'on.}
%
\en{Jointly with this work, we make available the first TrueSkill Through Time packages for \proglang{Julia}, \proglang{Python}, and \proglang{R}, together with its scientific documentation.}
\es{Con este art\'iculo ponemos a disposici\'on los primeros paquetes de TrueSkill Through Time para \proglang{Julia}, \proglang{Python} y \proglang{R}, junto con su documentaci\'on cient\'ifica.}


% \cite{A Guide to state-space modeling of ecological time series}


%En este punto es razonable preguntarse si los priors de los \emph{state-space} model son priors honestos en el sentido de que desconocen la informacion del dato  para el que son priors.
% 
%Si bien en la ecuaci\'on~\ref{eq:smooth_prior} vimos que el prior efectivamente no se multiplica directamente por el likelihood presente, ¿Ser\'a que la informaci\'on de ese likelihood esta siendo incorporada de manera oculta por el algoritmo iterativo?
%
%Si esto fuera as\'i, entonces en cada paso de la convergencia deber\'ia ocurrir que la sorpresa disminuye.
%
%Para responder a este pregunta y para entender el funcionamiento general de los state-space models mostramos en la Figura~\ref{fig:smooth_example} una solcuci\'on de forma gr\'afica.
%










































































































\newpage


%% -- Illustrations ------------------------------------------------------------

%% - Virtually all JSS manuscripts list source code along with the generated
%%   output. The style files provide dedicated environments for this.
%% - In R, the environments {Sinput} and {Soutput} - as produced by Sweave() or
%%   or knitr using the render_sweave() hook - are used (without the need to
%%   load Sweave.sty).
%% - Equivalently, {CodeInput} and {CodeOutput} can be used.
%% - The code input should use "the usual" command prompt in the respective
%%   software system.
%% - For R code, the prompt "R> " should be used with "+  " as the
%%   continuation prompt.
%% - Comments within the code chunks should be avoided - these should be made
%%   within the regular LaTeX text.

\section{Illustrations} \label{sec:illustrations}

%
\begin{comment}
\begin{CodeChunk}
\begin{CodeInput}
R> data("quine", package = "MASS")
\end{CodeInput}
\end{CodeChunk}

\begin{leftbar}
For code input and output, the style files provide dedicated environments.
Either the ``agnostic'' \verb|{CodeInput}| and \verb|{CodeOutput}| can be used
or, equivalently, the environments \verb|{Sinput}| and \verb|{Soutput}| as
produced by \fct{Sweave} or \pkg{knitr} when using the \code{render_sweave()}
hook. Please make sure that all code is properly spaced, e.g., using
\code{y = a + b * x} and \emph{not} \code{y=a+b*x}. Moreover, code input should
use ``the usual'' command prompt in the respective software system. For
\proglang{R} code, the prompt \code{"R> "} should be used with \code{"+  "} as
the continuation prompt. Generally, comments within the code chunks should be
avoided -- and made in the regular {\LaTeX} text instead. Finally, empty lines
before and after code input/output should be avoided (see above).
\end{leftbar}
\end{comment}

\en{In this section we show how to use \proglang{Julia}, \proglang{Python} and \proglang{R} packages to solve a single event, various sequences of events, and the real history of the Association of Tennis Professionals (ATP).}
\es{En esta secci\'on mostramos c\'omo usar los paquetes de \proglang{Julia}, \proglang{Python} y \proglang{R} para resolver un evento, doversas secuencias de eventos, y la historia real de la Asociación de Tenis Profesional (ATP).}
%
\en{We present both TrueSkill and TrueSkill Through Time solutions, and the steps to obtain the posteriors, the learning curves, and the prior prediction of the observed data (i.e. evidence).}
\es{Presentamos la soluciones TrueSkill y TrueSkill Through Time, y los pasos para obtener los posteriors, las curvas de aprendizaje, y la predicci\'on a priori del dato observado (i.e. evidencia).}
%
\en{Since the tool we present was developed with three programming languages, we will identify the different syntaxes using the following layout:}
\es{Debido a que la herramienta que presentamos fue desarrollada con tres lenguajes de programaci\'on, identificaremos las diferentes sintaxis usando el siguiente formato:}
%
\begin{lstlisting}[backgroundcolor=\color{all}, belowskip=-0.77 \baselineskip]
Syntax common to Julia, Python and R
\end{lstlisting}
\begin{paracol}{3}
\begin{lstlisting}[backgroundcolor=\color{julia}]
Julia syntax
\end{lstlisting}
  \switchcolumn
\begin{lstlisting}[backgroundcolor=\color{python}]
Python syntax
\end{lstlisting}
   \switchcolumn
\begin{lstlisting}[backgroundcolor=\color{r}]
R syntax
\end{lstlisting}  
\end{paracol}
%
\en{where the full line is used when the syntax of the three languages match, and the columns are used when the languages differ: \proglang{Julia} on the left, \proglang{Python} in the middle, and \proglang{R} on the right.}
\es{donde la l\'inea completa la usamos cuando la sintaxis de los tres lenguajes coinciden, y las columnas las usamos cuando los lenguajes difieren: \proglang{Julia} a la izquierda, \proglang{Python} al centro, y \proglang{R} a la derecha.}

\subsection{Single event} \label{sec:singleEvent}

\en{The class \texttt{Game} is used to model events.}
\es{La clase \texttt{Game} la utilizamos para modelar eventos.}
%
\en{The features of the players are defined within class \texttt{Player}: the prior Gaussian distribution characterized by the mean (\texttt{mu}) and the standard deviation (\texttt{sigma}), the standard deviation of the performance (\texttt{beta}), and the dynamic factor of the skill (\texttt{gamma}).}
\es{Las caracterísiticas de los jugadores se definen al interior de la clase \texttt{Player}: la distribuci\'on gaussiana a priori caracterizada por la media (\texttt{mu}) y el desv\'io est\'andar (\texttt{sigma}); el desv\'io est\'andar de los rendimientos (\texttt{beta}); y el factor din\'amico de la habilidad (\texttt{gamma}).}
%
\en{In the following code we define the variables that we will use later, assigning the default values of the packages.}
\es{In the following code we define the variables that we will use later, assigning the default values of the packages.}
%
\begin{lstlisting}[backgroundcolor=\color{all},label=lst:parameters, caption={\en{Package parameters and their default values}\es{Parámetros de los paquetes y sus valores por defecto}},aboveskip=0.1cm]
mu = 0.0; sigma = 6.0; beta = 1.0; gamma = 0.03; p_draw = 0.0
\end{lstlisting}  
%
\en{The initial value of \texttt{mu}, common to all players, can be freely chosen because it is the difference in skills which really matters and not its absolute value.}
\es{El valor inicial de \texttt{mu}, común a todos los jugadores, puede elegirse a gusto debido que es la diferencia de habilidades lo que realmente importa y no su valor absoluto.}
%
\en{The prior's standard deviation \texttt{sigma} must be sufficiently large to include all possible skill hypotheses.}
\es{El desv\'io est\'andar del prior \texttt{sigma} debe ser suficientemente grande para incluir todas las posibles hip\'otesis de habilidad.}
%
\en{The value of \texttt{beta} ($\beta$) is perhaps the most important because it works as the scale of the estimates.}
\es{El valor de \texttt{beta} quizás sea el más importante debido a que funciona como la escala de las estimaciones.}
%
\en{A difference of one $\beta$ between two actual skills, $ s_i - s_j = \beta $, is equivalent to 76\% probability of winning.}
\es{Una diferencia de habilidad real de un $\beta$, $s_i - s_j = \beta$, equivale a 76\% de probabilidad de ganar.}
%
\en{Since it is the unit of measurement, we choose \texttt{beta=1.0}.}
\es{Como es la unidad de medida, elegimos \texttt{beta=1.0}.}
%
\en{The dynamic factor \texttt{gamma} is generally a small fraction of \texttt{beta}.}
\es{El factor dinámico \texttt{gamma} es en general una peque\~na fracci\'on de \texttt{beta}.}
%
\en{And the probability of a draw in the games (\texttt{p\_draw}) is usually initialized with the observed frequency of draws.}
\es{Y la probabilidad de empate de las partidas (\texttt{p\_draw}) se suele inicializar con la frecuencia observada de empates.}
%
\en{With these default values we create four identical players.}
\es{Con estos valores por defecto creamos cuatro jugadores idénticos.}
\begin{lstlisting}[backgroundcolor=\color{all},label=lst:player, caption={\en{Players initialization}\es{Inicializaci\'on de los jugadores}},aboveskip=0.1cm]
p1 = Player(Gaussian(mu, sigma), beta, gamma); p2 = Player(); p3 = Player(); p4 = Player()
\end{lstlisting}   
%
\en{The first player was created by making the constructor parameters explicit, while the rest were initialized with the default values.}
\es{El primer jugador fue creado haciendo explícitos los parámetros del constructor, mientras que el resto fueron inicializados con los valores por defecto.}
%
\en{The \texttt{Gaussian} class is used to model the standard operations of Gaussian distributions including multiplication, summation, division, and substraction (details at section \ref{sec:Gasussian}).}
\es{La clase \texttt{Gaussian} se usa para modelar las operaciones estandar de las distribuciones gaussianas, incluyendo multiplicación, suma, divisi\'on y resta (detalles en la secci\'on \ref{sec:Gaussian}).}
%
\en{In the next step we create a game with two teams of two players.}
\es{En el siguiete paso creamos una partida con dos equipos de dos jugadores.}
%
\en{When dealing with teams, the observed result depends on the sum of the performances of each team (see details at section \ref{sec:2vs2}).}
\es{En presencia de equipos, el rendimiento observado depende de la suma de los rendimientos de cada equipo (detalles en la secci\'on \ref{sec:2vs2}).}
%
\begin{lstlisting}[backgroundcolor=\color{white},label=lst:game, caption={\en{Teams and game initialization}\es{Equipos e inicializaci\'on del juego}}, belowskip=-1.0 \baselineskip, aboveskip=0.1cm]
\end{lstlisting}
\begin{paracol}{3}
\begin{lstlisting}[backgroundcolor=\color{julia}, belowskip=-0.77 \baselineskip]
team_a = [ p1, p2 ]
team_b = [ p3, p4 ]
teams = [team_a, team_b]
\end{lstlisting}
  \switchcolumn
\begin{lstlisting}[backgroundcolor=\color{python}, belowskip=-0.77 \baselineskip]
team_a = [ p1, p2 ]
team_b = [ p3, p4 ]
teams = [team_a, team_b]
\end{lstlisting}
   \switchcolumn
\begin{lstlisting}[backgroundcolor=\color{r}, belowskip=-0.77 \baselineskip]
team_a = c(p1, p2)
team_b = c(p3, p4)
teams = list(team_a, team_b)
\end{lstlisting}  
\end{paracol}
\begin{lstlisting}[backgroundcolor=\color{all}]
g = Game(teams)
\end{lstlisting}
%
\en{where the result of the game is implicitly defined by the order of the teams in the list: the teams appearing first in the list (lower index) beat those appearing later (higher index)).}
\es{donde el resultado de la partida queda definido implícitamente por el orden de los equipos en la lista \texttt{temas}: los equipos que aparecen primero en la lista (menor índice) le ganan a los que aparecen después (mayor índice)).}
%
\en{This is the simplest example of use.}
\es{Este es el ejemplo de uso más simple.}
%
\en{Later on we will see how to explicitly specify the result.}
\es{Más adelante veremos como especificar explícitamente el resultado.}
%
\en{During the initialization, the class \texttt{Game} computes the prior prediction of the observed result (\texttt{evidence}) and the approximate likelihood of each player (\texttt{likelihoods}).}
\es{Duarante la inicialización, la clase \texttt{Game} calcula la predicción a priori del resultado observado (\texttt{evidence}) y las verosimilitud aproximada de cada jugador (\texttt{likelihoods}).}
%
\begin{lstlisting}[backgroundcolor=\color{white},label=lst:evidence_likelihoods, caption={\en{Evidence and likelihoods queries}\es{Comsulta de la evidencia y las verosimilitudes}}, belowskip=-1.0 \baselineskip, aboveskip=0.1cm]
\end{lstlisting}
\begin{paracol}{3}
\begin{lstlisting}[backgroundcolor=\color{julia}, belowskip=-0.77 \baselineskip]
lhs = g.likelihoods
ev = g.evidence
ev = round(ev, digits=3)
\end{lstlisting}
  \switchcolumn
\begin{lstlisting}[backgroundcolor=\color{python}, belowskip=-0.77 \baselineskip]
lhs = g.likelihoods
ev = g.evidence
ev = round(ev, 3)
\end{lstlisting}
   \switchcolumn
\begin{lstlisting}[backgroundcolor=\color{r}, belowskip=-0.77 \baselineskip]
lhs = g$likelihoods
ev = g$evidence
ev = round(ev, 3)
\end{lstlisting}
\end{paracol}
\begin{lstlisting}[backgroundcolor=\color{all}]
print(ev)
> 0.5
\end{lstlisting}
%
\en{In this case, the evidence is $0.5$ indicating that both teams had the same probability of winning given the prior estimates.}
\es{En este caso, la evidencia es de $0,5$ indicando que ambos equipos tenian la misma probabilidad de ganar dadas las estimaciones a priori.}
%
\en{Posteriors can be found by manually multiplying the likelihoods and priors, or we can call the method \texttt{posteriors()} of class \texttt{Game} to compute them.}
\es{Los posteriors se pueden obtener multiplicando manualmente las verosimilitudes y los priors, o podemos llamar al método \texttt{posteriors()} de la clase \texttt{Game} para que los compute.}
%
\en{The likelihoods and posteriors are returned keeping the order in which players and teams were loaded during the initialization of the class \texttt{Game}.}
\es{Las verosimilitudes y posteriors se devuelven manteniendo el orden en el que los jugadores y equipos fueron cargados durante la inicialización de la clase \texttt{Game}.}
%
\begin{lstlisting}[backgroundcolor=\color{white},label=lst:game_posterior, caption={\en{Posteriors query}\es{Consulta de los posteriors}}, belowskip=-1.0 \baselineskip, aboveskip=0.1cm]
\end{lstlisting}
\begin{paracol}{3}
\begin{lstlisting}[backgroundcolor=\color{julia}, belowskip=-0.77 \baselineskip]
pos = posteriors(g)
print(pos[1][1])
\end{lstlisting}
  \switchcolumn
\begin{lstlisting}[backgroundcolor=\color{python}, belowskip=-0.77 \baselineskip]
pos = g.posteriors()
print(pos[0][0])
\end{lstlisting}
   \switchcolumn
\begin{lstlisting}[backgroundcolor=\color{r}, belowskip=-0.77 \baselineskip]
pos = g$posteriors()
print(pos[[1]][[1]])
\end{lstlisting}  
\end{paracol}
\begin{lstlisting}[backgroundcolor=\color{all}, belowskip=-0.77 \baselineskip]
> Gaussian(mu=2.361, sigma=5.516)
\end{lstlisting}
\begin{paracol}{3}
\begin{lstlisting}[backgroundcolor=\color{julia}, belowskip=-0.77 \baselineskip]
print(lhs[1][1] * p1.prior)
\end{lstlisting}
  \switchcolumn
\begin{lstlisting}[backgroundcolor=\color{python}, belowskip=-0.77 \baselineskip]
print(lhs[0][0] * p1.prior)
\end{lstlisting}
   \switchcolumn
\begin{lstlisting}[backgroundcolor=\color{r}, belowskip=-0.77 \baselineskip]
print(lhs[[1]][[1]]*p1$prior)
\end{lstlisting}  
\end{paracol}
\begin{lstlisting}[backgroundcolor=\color{all}]
> Gaussian(mu=2.361, sigma=5.516)
\end{lstlisting}
%
\en{where the printed posterior corresponds to the first player of the first team.}
\es{donde el posterior impreso corresponde al primer jugadar del primer equipo.}
%
\en{Due to the winning result, the player's estimate now has a larger mean and a smaller uncertainty.}
\es{Debido al resultado ganador, la estimación del jugador tiene ahora una media más grande y una incertidumbre más chica.}
%
\en{The product of Gaussians, the likelihood times the prior, generates the same normalized posterior.}
\es{El producto de gaussianas, la verosimilitud por el prior, genera el mismo posterior normalizado.}

% Parrafo

\en{We now analyze a more complex example in which the same four players participate in a multi-team game.}
\es{Ahora analizamos un ejemplo más complejo en el que los mismos cuatro jugadores participan en un juego de varios equipos.}
%
\en{The players are organized in three teams of different size: two teams with only one player, and the other with two players.}
\es{Los jugadores se organizan en tres equipos de diferente tamaño: dos equipos con un solo jugador, y el otro con dos jugadore.}
%
\en{The result has a single winning team and a tie between the other two losing teams.}
\es{El resultado tiene un único equipo ganador y un empate entre los otros dos equipos perdedores.}
%
\en{Unlike the previous example, we need to use a draw probability greater than zero.}
\es{A diferencia del ejemplo anterior, ahora necesitamos usar una probabilidad de empate mayor a cero.}
%
\begin{lstlisting}[backgroundcolor=\color{white},label=lst:game, caption={\en{Games with multiple teams of different sizes and the possibility of ties}\es{Juegos con multiples equipos de diferente tamaño y posibilidad de empates}}, belowskip=-1.0 \baselineskip, aboveskip=0.1cm]
\end{lstlisting}
\begin{paracol}{3}
\begin{lstlisting}[backgroundcolor=\color{julia}, belowskip=-0.77 \baselineskip]
ta = [p1]
tb = [p2, p3]
tc = [p4]
teams = [ta, tb, tc]
result = [1., 0., 0.]
\end{lstlisting}
  \switchcolumn
\begin{lstlisting}[backgroundcolor=\color{python}, belowskip=-0.77 \baselineskip]
ta = [p1]
tb = [p2, p3]
tc = [p4]
teams = [ta, tb, tc]
result = [1, 0, 0]
\end{lstlisting}
   \switchcolumn
\begin{lstlisting}[backgroundcolor=\color{r}, belowskip=-0.77 \baselineskip]
ta = c(p1)
tb = c(p2, p3)
tc = c(p4)
teams = list(ta, tb, tc)
result = c(1, 0, 0)
\end{lstlisting}  
\end{paracol}
\begin{lstlisting}[backgroundcolor=\color{all}]
g2 = Game(teams, result, p_draw=0.25)
\end{lstlisting}
%
\en{where \texttt{teams} contains the players distributed in different teams, while \texttt{result} now indicates the score obtained by each team.}
\es{donde \texttt{teams} contiene a los jugadores distribuidos en diferentes equipos, mientras que \texttt{result} indica ahora la puntuación obtenida por cada equipo.}
%
\en{The team with the highest score is the winner and the teams with the same score are tied.}
\es{El equipo con la mayor puntuación es el ganador y los equipos con misma puntuación están empatados.}
%
\en{In this way we can specify any outcome, including global draws.}
\es{De este modo podemos especificar cualquier resultado, incluidos empates globales.}
%
\en{The evidence and the posteriors can be queried in the same way as before.}
\es{La evidencia y el posterior se obtienen de la misma forma que hemos visto.}

\subsection{Sequence of events}

\en{The class \texttt{History} is used to compute the posteriors and evidence of a sequence of events.}
\es{La clase \texttt{History} se usa para computar los posteriors y la evidencia  the secuencias de eventos.}
%
\en{In the first example, we instantiate the class \texttt{History} with three players (\texttt{\small "a", "b", "c"}) and three games.}
\es{En el primer ejemplo, inicializamos una la clase \texttt{History} con tres jugadores (\texttt{"a","b","c"}) y tres partidas.}
%
\en{In the first game \texttt{"a"} beats \texttt{"b"}, in the second game \texttt{"b"} beats \texttt{"c"}, and in the third game \texttt{"c"} beats \texttt{"a"}.}
\es{En la primera partida \texttt{"a"} le gana a \texttt{"b"}, en la segunda \texttt{"b"} le gana a \texttt{"c"} y en la tercera \texttt{"c"} le gana a \texttt{"a"}.}
%
\en{In this case, all agents win one game and lose the other.}
\es{En este caso, todos los agentes ganan una partida y pierden la otra.}
%
\begin{lstlisting}[backgroundcolor=\color{white},label=lst:history, caption={\en{History initialized with a sequence of three events }\es{Historia inicializada con una secuencia de tres eventos}}, belowskip=-1.0 \baselineskip, aboveskip=0.1cm]
\end{lstlisting}
\begin{paracol}{3}
\begin{lstlisting}[backgroundcolor=\color{julia},belowskip=-0.77 \baselineskip]
c1 = [["a"],["b"]]
c2 = [["b"],["c"]]
c3 = [["c"],["a"]]
composition = [c1, c2, c3]
\end{lstlisting}
  \switchcolumn
\begin{lstlisting}[backgroundcolor=\color{python},belowskip=-0.77 \baselineskip]
c1 = [["a"],["b"]]
c2 = [["b"],["c"]]
c3 = [["c"],["a"]]
composition = [c1, c2, c3]
\end{lstlisting}
   \switchcolumn
\begin{lstlisting}[backgroundcolor=\color{r},belowskip=-0.77 \baselineskip]
c1 = list(c("a"),c("b"))
c2 = list(c("b"),c("c"))
c3 = list(c("c"),c("a"))
composition = list(c1,c2,c3)
\end{lstlisting}
\end{paracol}
\begin{lstlisting}[backgroundcolor=\color{all}]
h = History(composition, gamma=0.0)
\end{lstlisting}
%
\en{where the variable \texttt{c1} indicates the composition of the first game using the names (i.e. identifiers) of the agents, the variable \texttt{composition} is a list containing the game compositions, and the parameter \texttt{gamma=0.0} indicate that agent's skill does not change between games.}
\es{donde la variable \texttt{c1} indica la composici\'on de la primer partida usando los nombres (i.e. identificadores) de los agentes, la variable \texttt{composition} es una lista que contiene la composición de la tres partidas y el par\'ametro \texttt{gamma=0.0} indica que la habilidad de los agetes no cambia entre partidas.}
%
\en{Again the results are implicitly defined by the order in which the game compositions were initialized: the teams appearing first in the list beat those appearing later.}
\es{Otra vez el resultado de la partida queda definido implícitamente por el orden en el que las composiciones fueron inicializadas: los equipos que aparecen primero en la lista le ganan a los que aparecen después.}
%
\en{The rest of the parameters are initialized with the default values seen in code~\ref{lst:parameters}.}
\es{El resto de los parámetros se inicializa con los valores por defecto vistos en el código \ref{lst:parameters}.}

% Parrafo

\en{When initialized, the \texttt{History} class immediately triggers the computation of TrueSkill estimates: it creates a player for each name, initializes a game for each event, and between events applies the dynamic factor to the players' estimates.}
\es{Al inicializarse, la clase \texttt{History} inmediatamente activa el computo de las estimaciones TrueSkill: crea un jugador por nombre, inicializa una partida por cada evento, y entre eventos aplica el factor dinámico a las estimaciones de los jugadores.}
%
\en{In this example, where all agents beat each other and their skills do not change over time, it would be reasonable for all estimates to be the same after observing the data.}
\es{En este ejemplo, en el que todos los agentes se ganan mutuamente y sus habilidades no cambian en el tiempo, sería razonable que todas las estimaciones sean iguales luego de observar los datos.}
%
\en{To access them we can call the method \texttt{learning\_curves()} of the class \texttt{History}, which returns a dictionary that is indexed by the names of the agents.}
\es{Para acceder acceder a ellas podemos llamar al m\'etodo \texttt{learning\_curves()} de la clase \texttt{HIstory}, que devuelve un diccionario indexado por los nombres de los agentes.}
%
\en{Individual learning curves are lists of tuples: the first element of each tuple indicates the time, and the second the estimate corresponding to that time.}
\es{Las curvas de aprendizaje individuales son listas de tuplas: el primer elemento de cada tupla indica el tiempo, y el segundo la estimaci\'on correpondiente a ese tiempo.}
%
\begin{lstlisting}[backgroundcolor=\color{white},label=lst:trueskill, caption={\en{Learning curves query}\es{Consulta de curvas de aprendizaje}}, belowskip=-1.0 \baselineskip, aboveskip=0.1cm]
\end{lstlisting}
\begin{paracol}{3}
\begin{lstlisting}[backgroundcolor=\color{julia}, belowskip=-0.77 \baselineskip]
lc = learning_curves(h)
print(lc["a"])
\end{lstlisting}
  \switchcolumn
\begin{lstlisting}[backgroundcolor=\color{python}, belowskip=-0.77 \baselineskip]
lc = h.learning_curves()
print(lc["a"])
\end{lstlisting}
   \switchcolumn
\begin{lstlisting}[backgroundcolor=\color{r}, belowskip=-0.77 \baselineskip]
lc = h$learning_curves()
lc_print(lc[["a"]])
\end{lstlisting}
\end{paracol}
\begin{lstlisting}[backgroundcolor=\color{all}, belowskip=-0.77 \baselineskip]
> [(1, Gaussian(mu=3.339, sigma=4.985)), (3, Gaussian(mu=-2.688, sigma=3.779))]
\end{lstlisting}
\begin{paracol}{3}
\begin{lstlisting}[backgroundcolor=\color{julia}, belowskip=-0.77 \baselineskip]
print(lc["b"])
\end{lstlisting}
  \switchcolumn
\begin{lstlisting}[backgroundcolor=\color{python}, belowskip=-0.77 \baselineskip]
print(lc["b"])
\end{lstlisting}
   \switchcolumn
\begin{lstlisting}[backgroundcolor=\color{r}, belowskip=-0.77 \baselineskip]
lc_print(lc[["b"]])
\end{lstlisting}
\end{paracol}
\begin{lstlisting}[backgroundcolor=\color{all}]
> [(1, Gaussian(mu=-3.339, sigma=4.985)), (2, Gaussian(mu=0.059, sigma=4.218))]
\end{lstlisting}
%
\en{The learning curves of both player \texttt{"a"} and \texttt{"b"} contain only two tuples, one per game.}
\es{Las curvas de aprendizaje de ambos jugador \texttt{"a"} y \texttt{"b"} contienen solamente dos tuplas, una por partida.}
%
\en{Although it would be reasonable to have the same estimate for all agents, the posteriors reported by TrueSkill differ greatly between players and vary greatly for the same player.}
\es{A pesar de que lo razonable sería tener la misma estimaci\'on para todos los agentes, los posteriors reportados por TrueSkill difieren mucho entre jugadores y varían mucho para uno mismo jugador.}
%
\en{The first tuple of both corresponds to time 1, the game in which \texttt{"a"} beats \texttt{"b"}.}
\es{La primera tupla de ambos se corresponde con el tiempo 1, la partida en la que \texttt{"a"} le gana a \texttt{"b"}.}
%
\en{Both estimates have the same uncertainty and the same mean absolute value, being positive for the winning player and negative for the loser.}
\es{Ambas estimaciones tienen misma incertidumbre y misma valor absoluto como media, positivo para el jugador ganador y negativo para el perdedor.}
%
\en{The second tuple for each player has less uncertainty and the mean is closer to zero.}
\es{La segunda tupla de cada jugador tiene menor incertidumbre y la media se encuentra más cerca de cero.}
%
\en{However, the estimates did not turn out to be equal as they should be in cases where no player is better than another.}
\es{Sin embargo las estimaciones no resultaron ser iguales como debería ser en los casos donde ningún jugador es mejor que otro.}
%
\en{The TrueSkill Through Time model remedy all these problems.}
\es{El modelo TrueSkill Through Time corrige todos estos problemas.}
%
\en{To compute it we need to converge history calling the method \texttt{convergence()}.}
\es{Para computarlo necesitamos converger la historia llamando al método \texttt{convergence()}.}
%
\begin{lstlisting}[backgroundcolor=\color{white},label=lst:ttt, caption={\en{History convergence}\es{Convergencia de la historia}}, belowskip=-1.0 \baselineskip, aboveskip=0.1cm]
\end{lstlisting}
\begin{paracol}{3}
\begin{lstlisting}[backgroundcolor=\color{julia}, belowskip=-0.77 \baselineskip]
convergence(h)
lc = learning_curves(h)
print(lc["a"])
\end{lstlisting}
  \switchcolumn
\begin{lstlisting}[backgroundcolor=\color{python}, belowskip=-0.77 \baselineskip]
h.convergence()
lc = h.learning_curves()
print(lc["a"])
\end{lstlisting}
   \switchcolumn
\begin{lstlisting}[backgroundcolor=\color{r}, belowskip=-0.77 \baselineskip]
h$convergence()
lc = h.learning_curves()
lc_print(lc[["a"]])
\end{lstlisting}
\end{paracol}
\begin{lstlisting}[backgroundcolor=\color{all}, belowskip=-0.77 \baselineskip]
> [(1, Gaussian(mu=0.0, sigma=2.395)), (3, Gaussian(mu=-0.0, sigma=2.395))]
\end{lstlisting}
\begin{paracol}{3}
\begin{lstlisting}[backgroundcolor=\color{julia}, belowskip=-0.77 \baselineskip]
print(lc["b"])
\end{lstlisting}
  \switchcolumn
\begin{lstlisting}[backgroundcolor=\color{python}, belowskip=-0.77 \baselineskip]
print(lc["a"])
\end{lstlisting}
   \switchcolumn
\begin{lstlisting}[backgroundcolor=\color{r}, belowskip=-0.77 \baselineskip]
lc_print(lc[["a"]])
\end{lstlisting}
\end{paracol}
\begin{lstlisting}[backgroundcolor=\color{all}]
> [(1, Gaussian(mu=-0.0, sigma=2.395)), (3, Gaussian(mu=0.0, sigma=2.395))]
\end{lstlisting}

%
\en{It can be seen that the TrueSkill Through Time estimates are not only the same between players, but also have less uncertainty.}
\es{Se puede ver que las estimaciones de TrueSkill Through Time no s\'olo son las mismas entre jugadores, sino que tienen tambi\'en menos incertidumbre.}
%
\en{Note also that the estimates do not change over time according to the assumption \texttt{gamma=0.0}.}
\es{Notar también que las estimaciones de los jugadores no cambian en el tiempo de acuerdo con el supuesto \texttt{gamma=0.0}.}

\subsection{The history of the Association of Tennis Professionals (ATP)}

\todo[inline]{Crear un archivo csv con lo básico, buenos nombres de columnas}
\todo[inline]{Verificar que el nombre sirva como ID}
\todo[inline]{Mostrar el ejemplo con un único lenguaje de programación.}
% 
% \en{The parameter \texttt{epsilon} indicates the convergence condition, the maximum change in all posteriors before and after an iteration.}
% \es{El parámetro \texttt{epsilon} indica la condici\'on de convergencia, el cambio máximo en todos los posteriors antes y despu\'es de una iteraci\'on.}
% %
% \en{Finally, the number of iterations \texttt{iter} functions as a cut-off condition in case convergence has not been reached.}
% \es{Por \'ultimo, la cantidad de iteraciones \texttt{iter} funciona como condici\'on de corte en caso de no haber alcanzado la convergencia.}

% \en{Immediately after the history is initialized, a summary of the data is printed.}
% \es{Inmediatamente despu\'es de que se inicializa la historia, se imprime un resumen de los datos.}
%
% \en{In this case, 3 events, 3 batches and 3 agents are reported.}
% \es{En este caso se reportan 3 eventos, 3 batches y 3 agentes.}
%
% \en{Batches are used to group events that happened at the same time.}
% \es{Los batches sirven para agrupar eventos que ocurrieron al mismo tiempo.}
%
% \en{We will see an example with batches later on.}
% \es{Ya veremos un ejemplo con batches más adelante.}
%
% \en{As we do not indicate the time of each game, the system places the games in different batches.}
% \es{Como no le indicamos el tiempo de cada partida, el sistema ubica las partida en batches distintos.}


\subsection{Example 4: Performance Analysis}


%\todo[inline]{Aca tendremos dos figuras analizando la base de Go: 1) TT original, TT nuestro, TT-R y TT-Julia ; y otra figura con TTT: Python, Julia y R}
























































































































%% -- Manuscript ---------------------------------------------------------------
%% - When describing longer chunks of code that are _not_ meant for execution
%%   (e.g., a function synopsis or list of arguments), the environment {Code}
%%   is recommended. Alternatively, a plain {verbatim} can also be used.
%%   (For executed code see the next section.)

\section{Models and software} 

\en{Driven by the growth in computing power, researchers have developed over the past five decades thousands of successful machine learning algorithms tailored to specific sets of problems.}
\es{Impulsado por la crecimiento en la capacidad de c\'omputo ocurrida, en las \'ultimas cinco d\'ecadas los investigadores han desarrollado miles de algoritmos exitosos de aprendizaje de autom\'atico, hechos a medida de conjuntos de problemas espec\'ificos.}
%
\en{In contrast, probabilistic inference offers a generally applicable framework~\citep{Bishop2013}.}
\es{Por el contrario, la inferencia probabil\'istica ofrece un marco de aplicaci\'on general~\citep{Bishop2013}.}
%
\en{The assumptions made by standard machine learning algorithms are generally difficult to remember and often difficult to perceive.}
\es{Las suposiciones hechas por los algoritmos est\'andar de aprendizaje autom\'atico son generalmente dif\'iciles de recordar y a menudo dif\'iciles de percibir.}
%
\en{Similarly, the algorithms used vary from application to application, forcing us to learn a variety of methods.}
\es{Del mismo modo, los algoritmos empleadas var\'ian de aplicaci\'on en aplicaci\'on, oblig\'andonos a aprender una gran variedad de m\'etodos.}
%
\en{Instead of having to transform the problem to fit some standard algorithm, probabilistic inference provides the building blocks to solve any specific problem.}
\es{En lugar de tener que transformar el problema para que encaje en alg\'un algoritmo est\'andar, la inferencia probabil\'istica ofrece los bloques de construcci\'on que permiten resolver cualquier problema espec\'ifico.}
%
\en{All the assumptions are made explicit through a generative (or causal) model, which can be intuitively specified in a graphical way.}
\es{Todos los supuestos se hacen expl\'icitos a trav\'es de un modelo generativo (o causal), que puede ser especificado intuitivamente de forma gr\'afica.}
%
\en{Inference can always be solved by applying the rules of probability (the sum and product rules) over the joint distribution represented by the model.}
\es{Y toda inferencia se hace aplicando las reglas de la probabilidad, de la suma y del producto, sobre el modelo representado como una distribuci\'on conjunta.}

% Parrafo

\en{In this section we will explain all the details of the probabilistic framework in the context of the TrueSkill Through Time problem.}
\es{En esta secci\'on explicaremos todos los detalles del marco probabil\'isitico en el contexto del problema TrueSkill Through Time.}
%
\en{In the section~\ref{sec:sumProductAlgorithm} we will introduce the \emph{sum-product algorithm}, which allows to efficiently compute any marginal distribution from a joint distributions.}
\es{En la secci\'on~\ref{sec:sumProductAlgorithm} introduciremos el \emph{sum-product algorithm}, que permite computar eficientemente cualquier distribuciones marginal a partir de una distribuci\'on conjunta.}
%
\en{In the section~\ref{sec:propiedades} we list the properties that we will need to derive the marginal distributions of interest.}
\es{En la secci\'on~\ref{sec:propiedades} enumeramos las propiedades que necesitaremos para derivar las distribuciones marginales de inter\'es.}
%
% Afortunadamente, gracias a la distribuci\'on de probabilidad elegida para representar el modelo causal~\ref{modelo_elo}, la distribuci\'on de creencias a posteriori de la habilidad de los agentes y la predicci\'on a priori del resultado observado (aka evidencia) tienen soluci\'on anal\'itica.
%
\en{In the section \ref{sec:Gasussian}, \ref{sec:rating} and \ref{sec:game} we introduce the implementation details of the classes \texttt{Gaussian}, \texttt{Rating} and \texttt{Game}.}
\es{En las secciones \ref{sec:Gasussian}, \ref{sec:rating} y \ref{sec:game} introducimos los detalles de implementaci\'on de las clases \texttt{Gaussian}, \texttt{Rating} y \texttt{Game}.}
%
\en{In the sections \ref{sec:exactSolution}, \ref{sec:empate}, \ref{sec:approximate_posterior}, and \ref{sec:iterative_posterior} we show respectively how to solve the prior prediction and the exact posterior of an event, we modified the model to include ties, we explain how to approximate the posterior, and we give the general multi-team solution.}
\es{En las secciones \ref{sec:exactSolution}, \ref{sec:empate}, \ref{sec:approximate_posterior}, y \ref{sec:iterative_posterior} mostramos respectivamente c\'omo resolver la predicci\'on a priori y el posterior exacto de un evento, modificamos el modelo para que incluya empates, explicamos c\'omo aproximar el posterior y damos la soluci\'on general multi-equipos.}
%
FALTA HISTORY
% %
% En la secci\'on~\ref{history} mostamos los pasos matem\'aticos requeridos para resolver el modelo TTT.
% %
% En la subsecci\'on~\ref{estructuras} introducimos las estructuras de datos utlizadas para generar la hsitoria.
% %
% En la subsecci\'on~\ref{trueskill} realizamos la inicializaci\'on de la historia, la que genera un resultado equivalente a trueskill.
% %
% En la subseci\'on~\ref{TTT} mostramos el algoritmo utilizado para converger las estmaciones.
% %
% 

\subsection{Sum-product algorithm} \label{sec:sumProductAlgorithm}

\en{The \emph{sum-product algorithm}~\citep{Kschischang2001} is a general procedure that takes advantage of the structure of the joint probability distribution imposed by the causal model to efficiently apply the rules of probability, the \ref{eq:sum_rule} and the \ref{eq:product_rule}.}
\es{El \emph{sum-product algorithm}~\citep{Kschischang2001} es un procedimiento general que aprovecha la estructura de la disstribuci\'on de probabilidad conjunta que impone el modelo causal para aplicar eficientemente las reglas de la probabilidad, la~\ref{eq:sum_rule} y la~\ref{eq:product_rule}.}
%
\en{Represented as joint distributions, the models can be factored into the product of simple conditional probabilities.}
\es{Representados como distribuciones conjuntas, los modelos pueden factorizarse en el producto de probabilidades condicionales simples.}
%
\en{Our generative model (Figure~\ref{fig:generative_model}) can be factored as,}
\es{Nuestro modelo generativo (figura~\ref{fig:generative_model}) puede factorizarse como,}
%
\begin{equation} \label{eq:factorization}
 p(\bm{s},\bm{p},d,r) = p(s_1)p(s_2)p(p_1|s_1)p(p_2|s_2)p(d|\bm{p})P(r|d)
\end{equation}
%
\en{In the Figure~\ref{fig:factor_graph}, we show its factorization graphically.}
\es{En la figura~\ref{fig:factor_graph} mostramos gr\'aficamente la factorizaci\'on.}
%
\en{These types of representations are known as \emph{factor graph}.}
\es{A este tipo de representaciones se las conoce como \emph{factor graph}.}
%
\en{The \emph{factor graph} are bipartite graphs, consisting of variable nodes $v$ and function nodes (or factors) $f$.}
\es{Los \emph{factor graph} son gr\'afos bipartitos, compuestos por nodos variables $v$ y nodos funciones (o factores) $f$.}
%
\en{The edge between variables and functions represent the mathematical relationship ``the variable $v$ is an argument of the function $f$''.}
\es{Los ejes entre variables y funciones representan la relaci\'on matem\'atica ``la variable $v$ es argumento de la funci\'on $f$''.}
%
\begin{figure}[h!]
\centering \small
    \tikz{         
%         \node[const, above=of fr] (nfr) {$f_r$}; %
% 	\node[const, above=of nfr] (dfr) {\large $\mathbb{I}(d >0)$}; %
    
    
    \node[factor] (fr) {} ; 
    %\node[const, left=of fr, xshift=-1.35cm] (r_name) {\small \en{Result}\es{Resultado}:}; 
    \node[const, left=of fr] (nfr) {\normalsize $P(r|d)$}; 
    \node[const, right=of fr] (dfr) {\normalsize \hspace{2.4cm} $P(r|d)=\mathbb{I}(d>0)$}; 

    \node[latent, above=of fr, yshift=-0.6cm] (d) {$d$} ; %
    \node[const, left=of d, xshift=-1.35cm] (d_name) {\small \en{Difference}\es{Diferencia}:};
    
    
    \node[factor, above=of d,yshift=-0.6cm] (fd) {} ; 
    \node[const, left=of fd] (nfd) {\normalsize $p(d|\bm{p})$}; 
    \node[const, right=of fd] (dfd) {\normalsize \hspace{2.4cm} $p(d|\bm{p}) =\mathbb{I}(d=p_1-p_2) \ o \ \delta() $}; 
    
    
    \node[latent, above=of fd, xshift=-0.8cm, yshift=-0.6cm] (p1) {$p_1$} ; %
    \node[latent, above=of fd, xshift=0.8cm, yshift=-0.6cm] (p2) {$p_2$} ; %
    \node[const, left=of p1, xshift=-0.55cm] (p_name) {\small \en{Performance}\es{Rendimiento}:}; 

    \node[factor, above=of p1 ,yshift=-0.6cm] (fp1) {} ; 
    \node[factor, above=of p2 ,yshift=-0.6cm] (fp2) {} ; 
    
    \node[latent, above=of fp1,yshift=-0.6cm] (s1) {$s_1$} ; %
    \node[latent, above=of fp2,yshift=-0.6cm] (s2) {$s_2$} ; %
    
    \node[factor, above=of s1 ,yshift=-0.6cm] (fs1) {} ; 
    \node[factor, above=of s2 ,yshift=-0.6cm] (fs2) {} ; 
    
    
    \node[const, left=of fp1] (nfp1) {\normalsize $p(p_1|s_1)$};
    \node[const, right=of fp2] (nfp2) {\normalsize $p(p_2|s_2)$};
    \node[const, right=of fp2] (dfp2) {\normalsize \hspace{1.6cm} $p(p_i|s_i)=\N(p_i|s_i,\beta^2)$};

    \node[const, left=of s1, xshift=-.85cm] (s_name) {\small \en{Skill}\es{Habilidad}:}; 
    
    \node[const, left=of fs1] (nfs1) {\normalsize $p(s_1)$};
    \node[const, right=of fs2] (nfs2) {\normalsize $p(s_2)$};
    \node[const, right=of fs2] (dfs) {\normalsize \hspace{1.6cm} $p(s_i) = \N(s_i|\mu_i,\sigma_i^2)$};

    
    \edge[-] {d} {fr};
    \edge[-] {p1,p2,d} {fd};
    \edge[-] {fp1} {p1,s1};
    \edge[-] {fp2} {p2,s2};
    \edge[-] {fs1} {s1};
    \edge[-] {fs2} {s2};
    %\node[invisible, right=of p2, xshift=4.35cm] (s-dist) {};
}
     \caption{
     \en{Factorizaci\'on gr\'afica del modelo generativo (Fig.~\ref{fig:generative_model}).}
     %
     \en{Los cuadrados negros representan las funciones, los c\'irculos blancos representan las variable, y los ejes entre ellos representan la relaci\'on matem\'atica ``la variable es argumento de la funci\'on''.}
     %
     }
    \label{fig:factor_graph}
\label{modelo}
\end{figure} 
%
\en{The structure encodes the minimum number of steps required to calculate any marginal probability distribution.}
\es{La estructura codifica la m\'inima cantidad de pasos que se requieren para calcular cualquier distribuci\'on de probabilidad marginal.}
%
\en{In our case we want to compute two marginals, the posterior of the skills $p(s_i|r)$ and the a prior probability of the result $p(r)$.}
\es{En nuestro caso querermos computar dos marginales, el posterior de las habilidades $p(s_i|r)$ y la probabilidad a priori del resultado $p(r)$.}

% Parrafo

\en{The \emph{sum-product algorithm} is a general way of breaking down the rules of probability as messages that are sent locally between the variables of the \emph{factor graph}.}
\es{El \emph{sum-product algorithm} es una forma general de descomponer las reglas de la probabilidad como mensajes que se env\'ian localmente las variables del \emph{factor graph}.}
%
\en{There are two types of messages: the messages that variable nodes send to their functions neighbors ($m_{v \rightarrow f}(v)$); and the messages that function nodes send to their variable neighbors ($m_{f \rightarrow v}(v)$).}
\es{Hay dos tipos de mensajes: los mensajes que envian los nodos variables a sus funciones vecinas ($m_{v \rightarrow f}(v)$); y los mensajes que envian los nodos funciones a sus variables vecinas ($m_{f \rightarrow v}(v)$).}
%
\en{The messages sent by the variable nodes encode a portion of the product rule.}
\es{Los mensajes que env\'ian los nodos variables codifican una porci\'on de la regla del producto.}
%
\begin{equation}\label{eq:m_v_f} \tag{\text{product step}}
m_{v \rightarrow f}(v) = \prod_{h \in n(v) \setminus \{f\} } m_{h \rightarrow v}(v)
\end{equation}
%
\en{Where $n(v)$ represents the set of node neighbors of $v$.}
\es{Donde $n(v)$ representa el conjunto de vecinos del nodo $v$.}
%
\en{In short, the messages sent by a $v$ variable is simply the product of the messages that $v$ received from the rest of their neighbors $h \in n(v)$ except $f$.}
\es{En pocas palabras, los mensajes que env\'ia una variables $v$ es simplemente la multiplicaci\'on de los mensajes que recibi\'o del resto de sus vecinos $h \in n(v)$ salvo $f$.}
%
\en{And the messages sent by the function nodes encode a portion of the sum rule.}
\es{Y los mensajes que env\'ian los nodos funciones codifican una parte de la regla de la suma.}
%
\begin{equation}\label{eq:m_f_v}  \tag{\text{sum step}}
m_{f \rightarrow v}(v) = \int \cdots \int \Big( f(\bm{h},v) \prod_{h \in n(f) \setminus \{v\} } m_{h \rightarrow f}(h) \Big) \,  d\bm{h}
\end{equation}
%
\en{Where $\bm{h} = n(f)\setminus \{v\}$ is the set of all of neighbors of $f$ except $v$, and $f(\bm{h},v)$ represents the function $f$, evaluated in all its arguments.}
\es{Donde $\bm{h} = n(f)\setminus \{v\}$ es el conjunto de todos los vecinos de $f$ salvo $v$, y $f(\bm{h},v)$ represeta la funci\'on $f$, evaluada en todos sus argumentos.}
%
\en{In short, the messages sent by a function $f$ to a neighboring variable $v$ is simply the integration over $\bm{h}$ of the product of itself and all the messages that $f$ receives from the rest of its neighbors $\bm{h}$ except $v$.}
\es{En pocas palabras, los mensajes que enviado por una funci\'on $f$ a una variable vecina $v$ es simplemente la integraci\'on sobre $\bm{h}$ del producto de si mismo con todos los mensajes que $f$ recibe del resto de sus vecinos $\bm{h}$ salvo $v$.}
%
\en{The composition of messages generates a partial computation.}
\es{La composici\'on de mensajes va generando un computo parcial.}
%
\en{Finally, the marginal probability distribution of a $v$ variable is simply the product of the messages that $v$ receives from its neighbors.}
\es{Finalmente, la distribuci\'on de probabilidad marginal de una variable $v$ es simplemente la multiplicaci\'on de los mensajes que $v$ recibe de sus vecinos.}
%
\begin{equation}\label{eq:marginal}
g_i(x_i) = \prod_{h \in n(x_i)} m_{h \rightarrow x_i}
\end{equation}



\subsection{\en{Mathematical properties and notation}\es{Propiedades matem\'aticas y notaci\'on}}\label{sec:propiedades}

\en{The efficiency of TrueSkill Through Time is due to the fact that margins, whether exact or approximate, are solved analytically.}
\es{La eficiencia de TrueSkill Through Time se debe a que las marginales, sean exactas o aproximadas, se resuelven de forma anal\'itica.}
%
\en{In this section we list the properties that we will use to derive the exact messages that arise from the sum-product algorithm.}
\es{En esta secci\'on enumeramos las propiedades que usaremos para derivar los mensajes exactos que surgen del \emph{sum-product algorithm}.}
%
\en{Although these properties are widely known, we attach their full demonstrations in the supplemental material.}
\es{Aunque estas propiedades son ampliamente conocidas, adjuntamos sus demostraciones completas en material suplementario.}

% Parrafo

\en{Let $\N$ be the Gaussian probability distribution, $\Phi$ the cumulative Gaussian distribution, $\mathbb{I}$ the indicator function.}
\es{Sea $\N$ la ditribuci\'on de probabilidad gaussiana, $\Phi$ la acumulada de una distribuc\'on gaussiana, $\mathbb{I}$ la funci\'on indicadora.}
%
\begin{equation}\label{eq:propiedad_1} \tag{\text{prop 1}}
\N(x|\mu_1,\sigma_1^2)\N(x|\mu_2,\sigma_2^2) \overset{\ref{multiplicacion_normales}}{=} \N(\mu_1|\mu_2,\sigma_1^2+\sigma_2^2) \N(x|\mu_{*},\sigma_{*}^2)
\end{equation}
%
con $\mu_{*} = \frac{\mu_1}{\sigma_1^2} + \frac{\mu_2}{\sigma_2^2}$ y $\sigma_{*}^2 = \left(\frac{1}{\sigma_1^2} + \frac{1}{\sigma_2^2} \right)^{-1}$ 

\begin{equation}\label{eq:propiedad_2} \tag{\text{prop 2}}
\N(x|\mu_1,\sigma_1^2)/\N(x|\mu_2,\sigma_2^2) \overset{\ref{sec:division_normales}}{=} \N(\mu_1|\mu_2,\sigma_1^2+\sigma_2^2) \N(x|\mu_{\div},\sigma_{\div}^2)
\end{equation}
%
con $\mu_{\div} = \frac{\mu_1}{\sigma_1^2} - \frac{\mu_2}{\sigma_2^2}$ y $\sigma_{\div}^2 = \left(\frac{1}{\sigma_1^2} - \frac{1}{\sigma_2^2} \right)^{-1}$ 

\begin{equation}\label{eq:integral_con_indicadora} \tag{\text{prop 3}}
\begin{split}
 \iint  \mathbb{I}(x=h(y,z)) f(x) g(y)\, dx\, dy = \int f(h(y,z)) g(y) dy
 \end{split}
\end{equation}

\begin{equation}\label{eq:simetria} \tag{\text{prop 4}}
 \N(x|\mu,\sigma^2) = \N(\mu|x,\sigma^2) = \N(-\mu|-x,\sigma^2) = \N(-x|-\mu,\sigma^2) 
\end{equation}

\begin{equation}\label{eq:estandarizar} \tag{\text{prop 5}}
  \N(x|\mu,\sigma^2) = \N((x-\mu)/\sigma | 0, 1)
\end{equation}

\begin{equation}\label{eq:phi_norm} \tag{\text{prop 6}}
 \frac{\partial}{\partial x} \Phi(x|\mu,\sigma^2) = \N(x|\mu,\sigma^2)
\end{equation}

\begin{equation}\label{eq:phi_simetria} \tag{\text{prop 7}}
\Phi(0|\mu,\sigma^2) = 1-\Phi(0|-\mu,\sigma^2)
\end{equation}

\subsection{Gaussian}\label{sec:Gasussian}

\en{The \texttt{Gaussian} class does most of the computation of the packages.}
\es{La clase \texttt{Gaussian} realiza la mayor parte del c\'omputo en todos los paquetes.}
%
\en{It is represented by two parameters, the mean and the standard deviation.}
\es{Se representa mediante dos par\'ametros, la media y el desv\'io estandar.}
%
\begin{lstlisting}[backgroundcolor=\color{white},label=lst:N1_N2, caption=\relax, belowskip=-1.0 \baselineskip, aboveskip=-0 \baselineskip]
\end{lstlisting}
\begin{lstlisting}[backgroundcolor=\color{all}, belowskip=-0.77 \baselineskip]
N1 = Gaussian(mu = 1.0, sigma = 1.0); N2 = Gaussian(1.0, 2.0)  
\end{lstlisting}
\begin{paracol}{3}
\begin{lstlisting}[backgroundcolor=\color{julia}]
m1 = N1.mu
v1 = N1.sigma^2
v2 = N2.sigma^2
\end{lstlisting}
\switchcolumn
\begin{lstlisting}[backgroundcolor=\color{python}]
m1 = N1.mu
v1 = N1.sigma**2
v2 = N2.sigma**2
\end{lstlisting}
\switchcolumn
\begin{lstlisting}[backgroundcolor=\color{r}]
m1 = N1$mu
v1 = N1$sigma^2
v2 = N2$sigma^2
\end{lstlisting}
\end{paracol}
%
\en{The class overwrites the operators addition (\texttt{+}), subtraction (\texttt{-}), product (\texttt{*}) and division (\texttt{/}) with the main properties required to compute the marginal distributions in the TrueSkill Through Time model.}
\es{La clase sobreescribe los operadores suma (\texttt{+}), resta (\texttt{-}), producto (\texttt{*}) y divisi\'on (\texttt{/}) con las principales propiedades requeridas para computar las distribuciones marginales en el modelo TrueSkill Through Time.}
%
\begin{equation} \tag{\texttt{N1 * N2}}
 \N(x|\mu_1,\sigma_1^2)\N(x|\mu_2,\sigma_2^2) \propto \N(x|\mu_{*},\sigma_{*}^2)
\end{equation}
%
\begin{equation} \tag{\texttt{N1 / N2}}
 \N(x|\mu_1,\sigma_1^2)/\N(x|\mu_2,\sigma_2^2) \propto \N(x|\mu_{\div},\sigma_{\div}^2)
\end{equation} 
%
\vspace{-0.3cm}
%
\begin{equation} \tag{\texttt{N1 + N2}} \label{eq:suma_normales}
\begin{split}
\iint \mathbb{I}(t = x + y) \N(x|\mu_1, \sigma_1^2)\N(y|\mu_2, \sigma_2^2) dxdy = \N(t|\mu_1+\mu_2,\sigma_1^2 + \sigma_2^2)
\end{split}
\end{equation}
%
\vspace{-0.5cm}
%
\begin{equation} \tag{\texttt{N1 - N2}} \label{eq:resta_normales}
\begin{split}
\iint \mathbb{I}(t = x - y) \N(x|\mu_1, \sigma_1^2)\N(y|\mu_2, \sigma_2^2) dxdy = \N(t|\mu_1 - \mu_2,\sigma_1^2 + \sigma_2^2)
\end{split}
\end{equation}
%
\en{For its part, the following property will be computed ``by hand''.}
\es{Por su parte, la siguiente propiedad la computaremos ``a mano''.}
%
\begin{equation} \label{eq:normal_de_normal} \tag{\texttt{Gaussian(m1, sqrt(v1+v2))}}
\int \N(y| x, \sigma_y^2) \N(x| \mu_x, \sigma_x^2) dx = \N(y|\mu_x,\sigma_y^2 + \sigma_x^2)   
\end{equation}
%
\en{Note that the value of \texttt{m1},\texttt{v1},\texttt{v2} has been previously defined.}
\es{Notar que el valor de \texttt{m1},\texttt{v1},\texttt{v2} a sido definido previamente.}


\subsection{\en{Graphical factorization}\es{Factorizaci\'on gr\'afica}}\label{sec:2vs2}

\en{In the figure~\ref{fig:modelo_trueskill_2vs2}, we show its graphical factorization.}
\es{En la figura~\ref{fig:modelo_trueskill_2vs2} mostramos su factorizaci\'on gr\'afica.} 
%
\begin{figure}[h!]
  \centering
  \scalebox{.9}{
  \tikz{
      
        \node[factor] (fr) {} ;
        \node[const, right=of fr] (nfr) {$f_{r}$}; %
	
	\node[latent, above=of fr, yshift=-0.4cm] (d) {$d$} ; %
        \node[factor, above=of d, yshift=-0.4cm] (fd) {} ;
        \node[const, above=of fd] (nfd) {$f_{d}$}; %
	
        
        \node[latent, left=of fd,xshift=0.4cm] (ta) {$t_a$} ; %
        \node[factor, left=of ta,xshift=0.4cm] (fta) {} ;
        \node[const, above=of fta] (nfta) {$f_{t_a}$}; %
        
        \node[latent, left=of fta,yshift=1cm,xshift=0.4cm] (p1) {$p_1$} ; %
        \node[factor, left=of p1,xshift=0.4cm] (fp1) {} ;
        \node[const, above=of fp1] (nfp1) {$f_{p_1}$}; %
        
        \node[latent, left=of fp1,xshift=0.4cm] (s1) {$s_1$} ; %
        \node[factor, left=of s1,xshift=0.4cm] (fs1) {} ;
	\node[const, above=of fs1] (nfs1) {$f_{s_1}$}; %
     
        \node[latent, left=of fta,yshift=-1cm,xshift=0.4cm] (p2) {$p_2$} ; %
        \node[factor, left=of p2,xshift=0.4cm] (fp2) {} ;
        \node[const, above=of fp2] (nfp2) {$f_{p_2}$}; %
        
        \node[latent, left=of fp2,xshift=0.4cm] (s2) {$s_2$} ; %
        \node[factor, left=of s2,xshift=0.4cm] (fs2) {} ;
	\node[const, above=of fs2] (nfs2) {$f_{s_2}$}; %
        
            
        \node[latent, right=of fd,xshift=-0.4cm] (tb) {$t_b$} ; %
        \node[factor, right=of tb,xshift=-0.4cm] (ftb) {} ;
        \node[const, above=of ftb] (nftb) {$f_{t_b}$}; %
        
        \node[latent, right=of ftb,yshift=1cm,xshift=-0.4cm] (p3) {$p_3$} ; %
        \node[factor, right=of p3,xshift=-0.4cm] (fp3) {} ;
        \node[const, above=of fp3] (nfp3) {$f_{p_3}$}; %
        
        \node[latent, right=of fp3,xshift=-0.4cm] (s3) {$s_3$} ; %
        \node[factor, right=of s3,xshift=-0.4cm] (fs3) {} ;
	\node[const, above=of fs3] (nfs3) {$f_{s_3}$}; %
     
        \node[latent, right=of ftb,yshift=-1cm,xshift=-0.5cm] (p4) {$p_4$} ; %
        \node[factor, right=of p4,xshift=-0.4cm] (fp4) {} ;
        \node[const, above=of fp4] (nfp4) {$f_{p_4}$}; %
        
        \node[latent, right=of fp4,xshift=-0.4cm] (s4) {$s_4$} ; %
        \node[factor, right=of s4,xshift=-0.4cm] (fs4) {} ;
	\node[const, above=of fs4] (nfs4) {$f_{s_4}$}; %
     
        \edge[-] {fr} {d};
	\edge[-] {d} {fd};
	
        \edge[-] {fd} {ta};
        \edge[-] {ta} {fta};
        \edge[-] {fta} {p1};
        \edge[-] {p1} {fp1};
        \edge[-] {fp1} {s1};
        \edge[-] {s1} {fs1};
        \edge[-] {fta} {p2};
        \edge[-] {p2} {fp2};
        \edge[-] {fp2} {s2};
        \edge[-] {s2} {fs2};
        	
	\edge[-] {fd} {tb};
        \edge[-] {tb} {ftb};
        \edge[-] {ftb} {p3};
        \edge[-] {p3} {fp3};
        \edge[-] {fp3} {s3};
        \edge[-] {s3} {fs3};
        \edge[-] {ftb} {p4};
        \edge[-] {p4} {fp4};
        \edge[-] {fp4} {s4};
        \edge[-] {s4} {fs4};
        
	
	\node[const, below=of fr,xshift=7cm,yshift=-0.3cm] (dfr) { $f_r = \mathbb{I}(d>0)$}; %
	\node[const, left=of dfr,xshift=-0.5cm] (dfd) {$f_d = \mathbb{I}(d=t_a - t_b)$}; %
	\node[const, left=of dfd,xshift=-0.5cm] (dft) {$f_{t_e} = \mathbb{I}(t_e = \sum_{i \in A_e} p_i)$}; %
        \node[const, left=of dft,xshift=-0.5cm] (dfp) {$f_{p_i} = \N(p_i|s_i,\beta^2)$}; %
        \node[const, left=of dfp,xshift=-0.5cm] (dfs) {$f_{s_i} = \N(s_i|\mu_i,\sigma^2)$}; %
   }
   }
  \caption{
  \en{Graphical factorization of a 2 vs 2 game.}
  \es{Factorizaci\'on gr\'afica de una partida 2 vs 2.}
  }
  \label{fig:modelo_trueskill_2vs2}
\end{figure}

\subsection{\en{Exact posterior}\es{Posterior exacta}} \label{sec:exactSolution}

\en{Every game with two teams has an exact solution.}
\es{Todo partida con dos equipos tiene soluci\'on exacta.}
%
\en{The exact likelihoods can be computed  \proglang{Julia}, \proglang{Python} y \proglang{R} by the method \texttt{exact\_likelihoods()}.}
\es{y los likelihoods exactos pueden computarse en \proglang{Julia}, \proglang{Python} y \proglang{R} mediante el m\'etodo \texttt{exact\_likelihoods()}.}
%
\begin{lstlisting}[backgroundcolor=\color{white},label=lst:elhs, caption=\relax, belowskip=-1.0 \baselineskip, aboveskip=-0 \baselineskip]
\end{lstlisting}
\begin{paracol}{3}
\begin{lstlisting}[backgroundcolor=\color{julia}]
elhs = exact_likelihoods(g)
\end{lstlisting}
  \switchcolumn
\begin{lstlisting}[backgroundcolor=\color{python}]
elhs = g.exact_likelihoods()
\end{lstlisting}
   \switchcolumn
\begin{lstlisting}[backgroundcolor=\color{r}]
elhs = g$exact_likelihoods()
\end{lstlisting}  
\end{paracol}
%
\en{In this section we will show the steps to compute the evidence and an exact likelihood of a game with two teams of two players.}
\es{En esta secci\'on vamos a mostrar los pasos para calcular la evidencia  y un likelihood exacto de una partida con dos equipos de dos jugadores.}
%
\en{We will start first with the ``descending'' messages, from the priors to the result, until compute the evidence.}
\es{Empezaremos primero con los mensajes ``descendentes'', desde los priors hasta el resultado, para calcular la evidencia.}
%
\en{Then we will continue with the ``ascending'' messages to compute the posterior of an agent from the observed result and the prior of the rest of the agents.}
\es{Luego seguiremos con los mensajes ``ascendentes'' que permiten calcular el posterior de un agente a partir del resultado observado y el prior del resto de los agentes.}
%
%\en{To compute the messages we will only use the sum-product algorithm and the properties mentioned above.}
%\es{Para computar los mensajes utilizaremos solamente el \emph{sum-product algorithm} y las propiedades arriba mencionadas.}

%

\en{Because of the rules of the sum-product algorithm and the factorization of the model, the messages from the skill factors $f_{s_i}$ to their variable $s_i$ are just the prior.}
\es{Debido a las reglas del algoritmo de suma-producto y a la factorizaci\'on del modelo, los mensajes de los factores de habilidad $f_{s_i}$ a su variable $s_i$ no son otra cosa m\'as que el prior.}
%
\begin{equation}\label{eq:m_fs_s} \tag{\texttt{prior}}
 m_{f_{s_i} \rightarrow s_i}(s_i) = \N(s_i| \mu_i, \sigma_i^2)
\end{equation}
%
\en{We have already defined the variable \texttt{prior} in the code \ref{lst:prior}.}
\es{Ya hemos definimos la variable \texttt{prior} en el c\'odigo~\ref{lst:prior}.}
%
\en{The next message, that the variable $s_i$ sends to the performance factor $f_{p_i}$ will also be the prior.}
\es{El siguiente mensaje, que la variable $s_i$ le env\'ia al factor rendimiento $f_{p_i}$ tambi\'en ser\'a el prior.}
%
\en{Remember that the messages that the variables send are just the product of the messages that the variable receives from the rest of its neighboring factors.}
\es{Recordar que los mensajes que env\'ian las variables es tan solo la multiplicaci\'on de los mensajes que esa variable recibe del resto de los factores.}
%
\en{Since they are trivial, we will not make these types of messages explicit.}
\es{Por ser de trivial soluci\'on, no haremos expl\'icitos este tipo de mensajes.}
%
\en{Let's see then the message that the performance factors $f_{p_i}$ send to their variable $p_i$.}
\es{Veamos entonces el mensaje que env\'ian los factores rendimiento $f_{p_i}$ a su variable $p_i$.}
%
\begin{equation}\label{eq:m_fp_p} \tag{\texttt{performance()}}
m_{f_{p_i} \rightarrow p_i}(p_i) = \int \N(p_i| s_i, \beta^2) \N(s_i| \mu_i, \sigma_i^2) ds_i = \N(p_i|\mu_i,\beta^2 + \sigma_i^2)
\end{equation}
%
%\en{Computacionalmente, podemos calcular esto de la siguiente manera.}
\begin{lstlisting}[backgroundcolor=\color{white},label=lst:performance, caption=\relax, belowskip=-1.0 \baselineskip, aboveskip=-0.5 \baselineskip]
\end{lstlisting}
\begin{paracol}{3}
\begin{lstlisting}[backgroundcolor=\color{julia}]
p1 = performance(r1)
  ...
p4 = performance(r4)
\end{lstlisting}
  \switchcolumn
\begin{lstlisting}[backgroundcolor=\color{python}]
p1 = r1.performance()
  ...
p4 = r4.performance()
\end{lstlisting}
   \switchcolumn
\begin{lstlisting}[backgroundcolor=\color{r}]
p1 = r1$performance()
  ...
p4 = r4$performance()
\end{lstlisting}  
\end{paracol}
%
\en{Then, the message sent by the team factors $f_{t_e}$ to the team variable $t_e$ are,}
\es{Luego, el mensaje que env\'ian los factores equipos $f_{t_e}$ a la variable equipo $t_e$ son,}
%
\begin{equation} \label{eq:m_ft_t} \tag{\texttt{ta = p1 + p2}}
\begin{split}
 m_{f_{t_e} \rightarrow t_e}(t_e) &= \iint \mathbb{I}(t_e = p_i + p_j) \N(p_i|\mu_i,\beta^2 + \sigma_i^2)\N(p_j|\mu_j,\beta^2 + \sigma_j^2) dp_idp_j  \\ &=  \N(t_e|\mu_i+\mu_j,2\beta^2 + \sigma_i^2 + \sigma_j^2)
\end{split}
\end{equation}
%
\en{In general (demo at annex section~\ref{suma_normales_induccion})}
\es{En general (demostraci\'on en secci\'on anexa~\ref{suma_normales_induccion})}.
%
\begin{equation*}
 m_{f_{t_e} \rightarrow t_e}(t_e) =  \N \Big( t_e | \underbrace{\sum_{i\in A_e } \mu_i}_{\hfrac{\text{Habilidad}}{\text{de equipo}} \ \mu_e}, \underbrace{\sum_{i \in A_e} \beta^2 + \sigma_i^2}_{\hfrac{\text{Varianza}}{\text{de equipo}} \ \sigma_e^2} \Big) = \N(t_e | \mu_e, \sigma_e^2)
\end{equation*}
%
\en{The message sent by the difference factor $f_{d_1}$ to the difference variable $d_1$ are,}
\es{El mensaje que env\'ia el factor diferencia $f_{d_1}$ a la variable diferencia $d_1$ es,}
%
\begin{equation}\label{eq:m_fd_d} \tag{\texttt{d = ta - tb}}
 \begin{split} 
  m_{f_{d_1} \rightarrow d_1}(d_1) & = \iint \mathbb{I}(d_1 = t_a - t_b) \N(t_a| \mu_a, \sigma_a^2)  \N(t_b| \mu_b, \sigma_b^2)  dt_adt_b \\[0.25cm]
  & = \N\big( d_1 | \underbrace{\mu_a - \mu_b}_{\hfrac{\text{Differencia}}{\text{esperada:} \ \delta} }, \underbrace{\sigma_a^2 +\sigma_b^2}_{\hfrac{\text{incertidumbre}}{\text{total:} \ \vartheta^2}}  \big) = \N(d_1 | \delta, \, \vartheta^2)
 \end{split}
\end{equation}
\begin{lstlisting}[backgroundcolor=\color{white},label=lst:difference, caption=\relax, belowskip=-1.0 \baselineskip, aboveskip=-0.5 \baselineskip]
\end{lstlisting}
\begin{lstlisting}[backgroundcolor=\color{all}, belowskip=-0.77 \baselineskip]
ta = p1 + p2; tb = p3 + p4
d = ta - tb
\end{lstlisting}
\begin{paracol}{3}
\begin{lstlisting}[backgroundcolor=\color{julia}]
delta = d.mu
theta = d.sigma
ev = cdf(d, 0.0)
evidencia = ev
\end{lstlisting}  
 \switchcolumn
\begin{lstlisting}[backgroundcolor=\color{python}]
delta = d.mu
theta = d.sigma
ev = cdf(0,delta,theta) 
evidencia = ev
\end{lstlisting} 
 \switchcolumn
\begin{lstlisting}[backgroundcolor=\color{r}]
delta = d$mu
theta = d$sigma
ev = cdf(0,delta,theta) 
evidencia = ev
\end{lstlisting}   
\end{paracol}
%
\begin{equation}\label{eq:m_fr_r} \tag{\texttt{evidencia}}
\begin{split}
 m_{f_{r_1} \rightarrow r_1}(r_1) = \int \mathbb{I}(d_1 > 0) \N(d_1 | \delta, \vartheta^2)  dd_1 = \Phi(\frac{\delta}{\vartheta})
\end{split}
\end{equation}

% Parrafo

\en{Let us now examine the ascending messages.}
\es{Examinemos ahora los mensajes ascendentes.}
%
\begin{equation}\label{eq:m_fd_ta} \tag{\texttt{exact\_lhood\_ta}}
\begin{split}
m_{f_{d_1} \rightarrow t_a}(t_a) & = \iint \mathbb{I}(d_1 = t_a - t_b) \mathbb{I}(d_1 > 0) \N(t_b | \mu_b , \sigma_b^2 ) \, dd_1\,dt_b \\
& = \int \mathbb{I}( t_a > t_b)  \N(t_b | \mu_b , \sigma_b^2 ) \,dt_b  \\
& = 1 - \Phi (t_a| -\mu_b, \sigma_b^2) = \Phi (t_a| \mu_b, \sigma_b^2)
\end{split}
\end{equation}
%
\begin{equation}\label{eq:m_fta_p_inicial} \tag{\texttt{exact\_lhood\_p1}}
\begin{split}
m_{f_{t_a} \rightarrow p_1}(p_1)  & = \iint \mathbb{I}( t_a = p_1 + p_2) \, N(p_2| \mu_2, \beta^2 + \sigma_2^2 ) \, \Phi (t_a| \mu_b , \sigma_b^2 ) \, dt_a dp_2 \\
& = \int  \, \N(p_2| \mu_2, \beta^2 + \sigma_2^2 ) \, \Phi (p_1 + p_2| \mu_b , \sigma_b^2 ) \, dp_2 \\
& = 1 - \Phi( p_1 | \underbrace{\mu_2 - \mu_b}_{\delta-\mu_1}, \underbrace{\beta^2 + \sigma_2^2 + \sigma_b^2}_{\vartheta^2 - (\sigma_1^2 + \beta^2)}) 
\end{split}
\end{equation}
%
\begin{equation}\label{eq:m_fp_s1} \tag{\texttt{exact\_lhood\_s1}}
\begin{split}
m_{f_{p_1} \rightarrow s_1}(s_1) & = \int N(p_1| s_1, \beta^2) \, \Phi(p_1| \mu_1-\delta, \vartheta^2 - (\sigma_1^2 + \beta^2)) \, dp_1 \\[0.1cm]
& = 1 - \Phi(0 | \underbrace{(s_1 + \mu_2) - (\mu_3 + \mu_4)}_{\hfrac{\text{\en{Expected difference}\es{Diferencia esperada}}}{\text{\en{parameterized in }\es{parametrizada en }$s_1$}}: \ \delta_1(s_1) } \ , \underbrace{\ \ \ \ \vartheta^2 - \sigma_1^2 \ \ \ \ }_{\hfrac{\text{\en{Total uncertainty}\es{Incertidumbre total}}}{\text{\en{except for }\es{salvo la de }$s_1$}}: \ \vartheta_1^2})\end{split}
\end{equation}
%
\en{This last message, $m_{f_{p_1} \rightarrow s_1}(s_1)$, is the exact likelihood and computes the prior probability of winning if the player's true skill was $s_1$.}
\es{Este \'ultimo mensaje, $m_{f_{p_1} \rightarrow s_1}(s_1)$, es el likelihood exacto y computa la probabilidad a priori de ganar si la verdadera habilidad del jugador fuera $s_1$.}
%
\en{Assuming that the skill is known, we must replace the average estimate $\mu_1$ by the hypothesis $s_1$ in the expected difference $\delta$, and remove its own uncertainty from the total uncertainty $\vartheta^2$.}
\es{Al suponer conocida la habilidad, debemos remplazar su estimaci\'on media $\mu_1$ por la hip\'otesis $s_1$ en la diferencia esperada $\delta$, y remover su propia incertidumbre de la incertidumbre total $\vartheta^2$.}
%
\begin{lstlisting}[backgroundcolor=\color{white},label=lst:delta_1, caption=\relax, belowskip=-1.0 \baselineskip, aboveskip=-0 \baselineskip]
\end{lstlisting}
\begin{paracol}{3}
\begin{lstlisting}[backgroundcolor=\color{julia}]
theta1 = theta^2 - sigma1^2
theta1 = sqrt(theta1)
function exact_lhood(s1)
  delta1 = delta - mu1 + s1 
  N= Gaussian(delta1,theta1)
  return 1-cdf(N,0.0)
end
\end{lstlisting}  
 \switchcolumn
\begin{lstlisting}[backgroundcolor=\color{python}]
theta1 = theta**2 - sigma1**2
theta1 = sqrt(theta1)
def exact_lhood(s1):
    delta1 = delta - mu1 + s1 
    res=1-cdf(0,delta1,theta1)
    return res
    
\end{lstlisting} 
 \switchcolumn
\begin{lstlisting}[backgroundcolor=\color{r}]
theta1 = theta^2 - sigma1^2
theta1 = sqrt(theta1)
exact_lhood = function(s1){
  delta1 = delta - mu1 + s1 
  res=1-cdf(0,delta1,theta1)
  return(res)
}
\end{lstlisting}   
\end{paracol}

\subsection{\en{Elementary draw model}\es{Modelo b\'asico de empates}} \label{sec:empate}

\en{The draw model asummes that a tie occur when the difference in performance does not exceed a certain margin, $|t_a > t_b| \leq \varepsilon$.}
\es{El modelo supone que ocurre un empate cuando la diferencia de rendimientos no supera un cierto margen, $|t_a > t_b| \leq \varepsilon$.}
%
\en{In the figure~\ref{fig:draw_a} you can see in graphical terms the probabilities of the three possible outcomes.}
\es{En la figura~\ref{fig:draw_a} se puede ver en t\'erminos gr\'aficos las probabilidades de los tres resultados posibles.}
%
\en{The elementary model requires determining the value of the margin.}
\es{Este modelo b\'asico requiere determinar el valor del margen.}
%
\en{The original paper~\citep{Herbrich2007} proposed to use the empirical frequency of ties.}
\es{El art\'iculo original~\citep{Herbrich2007} propon\'ia usar la frecuencia emp\'irica de empates.}
%
\en{However, this value depends on the actual skill difference, which we just don't know.}
\es{Sin embargo, este valor depende de la diferencia de habilidad real, que justamente no conocemos.}
%
\en{In section~\ref{sec:ttt-d} we present the Bayesian solution to the darw model.}
\es{En la secci\'on~\ref{sec:ttt-d} presentamos la soluci\'on bayesiana al modelo de empates.}
%
\en{In any case, assuming that we can define the ``probability of a draw between teams with same skill'', it is important to note that the margin also depends on the number of players.}
\es{En cualquier caso, suponiendo que podemos definir esa la ``probabilidad de empate entre equipos con misma habilidad'', es importante tener en cuenta que el margen tambi\'en depende de la cantidad de jugadores.}
%
\en{In the figure~\ref{fig:draw_b} you can see that to keep the tie area constant it is necessary to adapt the margin according to the uncertainty.}
\es{En la figura~\ref{fig:draw_b} se puede ver que para mantener el \'area de empates constante, es necesario adaptar el margen dependiendo de la incertidumbre.}
%
%\en{This is because the actual distribution of performance differences depends on how many players are in the game.}
%\es{Esto es as\'i porque la distribuci\'on de diferencias de rendimientos real depende de cu\'antos jugadores hay en la partida.}
%
\begin{figure}[h!]
\centering
\begin{subfigure}[t]{0.48\textwidth}
 \includegraphics[width=1\textwidth]{figures/draw.pdf} 
 \caption{Las distintas \'areas}
 \label{fig:draw_a}
\end{subfigure}
\begin{subfigure}[t]{0.48\textwidth}
  \includegraphics[page=2,width=1\textwidth]{figures/draw.pdf}
  \caption{Área de empate constante}
 \label{fig:draw_b}
\end{subfigure}
  \caption{Modelo de empate.}
  \label{fig:draw}
\end{figure}
%
\en{Since the observed results are independent of our beliefs, the only source of uncertainty comes from the variance of individual perfomances $\beta$.}
\es{Como los resultados observados son independientes de nuestras creencias, la \'unica fuente de incertidumbre proviene de varianza de los rendimientos $\beta$.}
%
\en{This is how we can define an equation that links the margin with the probability of a tie.}
\es{As\'i es que podemos definir una ecuaci\'on que vincula el margen con la probabilidades de empate.}
%
\begin{equation}
 \text{Draw probability} = \Phi(\frac{\varepsilon}{\sqrt{n_1+n_2}\beta}) - \Phi(\frac{-\varepsilon}{\sqrt{n_1+n_2}\beta})
\end{equation}
%
\begin{lstlisting}[backgroundcolor=\color
{white},label=lst:draw, caption=\relax, belowskip=-1.0 \baselineskip, aboveskip=-0 \baselineskip]
\end{lstlisting}
\begin{paracol}{3}
\begin{lstlisting}[backgroundcolor=\color{julia},belowskip=-0.77 \baselineskip]
na = length(team_a)
nb = length(team_b)
sd = sqrt(na + nb)*BETA
\end{lstlisting}
\switchcolumn
\begin{lstlisting}[backgroundcolor=\color{python},belowskip=-0.77 \baselineskip]
na = len(team_a)
nb = len(team_b)
sd = math.sqrt(na + nb)*BETA
\end{lstlisting}
\switchcolumn
\begin{lstlisting}[backgroundcolor=\color{r},belowskip=-0.77 \baselineskip]
na = length(team_a)
nb = length(team_b)
sd = sqrt(na + nb)*BETA
\end{lstlisting}
\end{paracol}
\begin{lstlisting}[backgroundcolor=\color{all}]
p_draw = 0.25
margin = compute_margin(p_draw, sd)
\end{lstlisting}  
%
%
% \en{In any case, an alternative is to use the classic optimization strategy: choose the value of $\varepsilon$ that maximizes some cost function.}
% \es{En cualquier caso, una alternativa es usar la estrategia cl\'asica de optimizaci\'on: elegir el valor de $\varepsilon$ que maximiza alguna funci\'on de costo.}
% %
% \begin{equation*}
% \varepsilon = \underset{\varepsilon}{\text{arg max}} \ p(\text{Data}|\text{Model}_{\varepsilon})
% \end{equation*}
% %
\subsection{\en{Optimal approximation of the exact posterior}\es{Aproximaci\'on \'optima del posterior exacto}} \label{sec:approximate_posterior}

\en{In the section~\ref{sec:exactSolution} we have seen how to find the exact posterior.}
\es{En la secci\'on~\ref{sec:exactSolution} hemos visto que el posterior exacta, si bien se parece a un Gaussiano, no lo es, .}
%
% \en{ although similar, is not a Gaussian distribution, preventing us from using the equation~\ref{eq:posterior_win}  iteratively.}
% \es{Es importante remarcar que la posterior, aunque se parezca, no es una distribuci\'on gaussiana, lo que nos impedir\'a usar la ecuaci\'on~\ref{eq:posterior_win} iterativamente.}
% %
% 
% 
% \en{But due the shape of the exact posterior, a Gaussian seem to be a good enough approximation, allowing us to avoid the computational cost of the sampling methodologies.}
% \es{Por la forma del posterior exacto, una gaussiana parece una aproximaci\'on suficientemente buena, que nos permitir\'ia evitar el costo computacional de la metodolog\'ias de sampleo.}
% %
% 
% \en{Dado que la forma exacta se parece a una gaussiana (figura~\ref{fig:posterior_win}), usar una gaussiana para representar el posterior no parecer\'ia ser tan grave en t\'erminos de aproximaci\'on y en cambio nos permitir\'ia realizar estimaciones sucesivas.}
% 
% % Parrafo
%\en{The methods for approximating an untractable distribution by means of a distribution belonging to a known family are known in Bayesian inference as ``variational''.}
%\es{Los m\'etodos para aproximar una distribuci\'on intratable mediante una distribuci\'on perteneciente a una familia conocida se conocen en inferencia bayesiana como ``variacionales''.}
%
\en{In this section we will show how to find the Gaussian distribution that best approximates the exact posterior, considering the possibility of ties.}
\es{En esta secci\'on mostraremos c\'omo encontrar la distribuci\'on gaussiana que mejor aproxima al posterior exacto, considerando la posibilidad de empates.}
%
\en{The packages solve it with the following two lines of code.}
\es{Los paquetes lo resuelven con las siguientes dos l\'ineas de c\'odigo.}
%
\begin{lstlisting}[backgroundcolor=\color
{white},label=lst:post_2vs2, caption=\relax, belowskip=-1.0 \baselineskip, aboveskip=-0 \baselineskip]
\end{lstlisting}
\begin{lstlisting}[backgroundcolor=\color{all},belowskip=-0.77 \baselineskip]
g = Game(teams, result, p_draw)
\end{lstlisting}  
\begin{paracol}{3}
\begin{lstlisting}[backgroundcolor=\color{julia}]
post = posteriors(g)
\end{lstlisting}
\switchcolumn
\begin{lstlisting}[backgroundcolor=\color{python}]
post = g.posteriors()
\end{lstlisting}
\switchcolumn
\begin{lstlisting}[backgroundcolor=\color{r}]
post = g$posteriors()
\end{lstlisting}
\end{paracol}
%
\en{Remember that \texttt{teams}, \texttt{results}, and \texttt{p\_draw}, were defined in Code \ref{lst:game} and \ref{lst:draw}.}
\es{Recuerden que \texttt{teams}, \texttt{results}, y \texttt{p\_draw}, fueron definidas en los c\'odigos \ref{lst:game} y \ref{lst:draw}.}

% Parrafo

\en{The need to approximate the posterior occurs because the probability distribution of the difference is a truncated Gasussian (Eq.~\ref{eq:p_d}).}
\es{La necesidad de aproximar el posterior ocurre debido a que la distribuci\'on de probabilidad de la diferencia es una Gasussian truncada (Eq.~\ref{eq:p_d}).}
%
\begin{equation}\label{eq:p_d}
p(d) =
\begin{cases}
\N(d|\delta,\vartheta^2) \mathbb{I}(-\varepsilon < d < \varepsilon) & \text{tie} \\
\N(d|\delta,\vartheta^2) \mathbb{I}(d > \varepsilon) & \text{not tie}
\end{cases}
\end{equation}
%
\en{For Gaussian distributions, moment matching is known to minimize the Kullback-Leibler divergence~\citep{Herbrich2007}.}
\es{Se sabe que las distribuciones gaussianas que minimizan la divergencia Kullback-Libler son las que tiene mismo momentos~\citep{Herbrich2007}.}
%
\en{The expectation and variance of a truncated Gaussian $\N(x|\mu,\sigma^2)$ in a $[a,b]$ interval are,}
\es{La esperanza y la varianza de una gaussiana truncada $\N(x|\mu,\sigma^2)$ en un intervalo $[a,b]$ son,}
%
\begin{equation}\label{eq:mean_aprox_double}
 E(X| a < X < b) = \mu + \sigma \frac{\N(\alpha) - \N(\beta) }{\Phi(\beta) - \Phi(\alpha) }
\end{equation}
%
\begin{equation}\label{eq:variance_aprox_double}
 V(X| a < X < b) = \sigma^2 \Bigg( 1 + \bigg(\frac{\alpha N(\alpha) - \beta N(\beta) }{\Phi(\beta) - \Phi(\alpha) }\bigg) - \bigg(\frac{N(\alpha) - N(\beta) }{\Phi(\beta) - \Phi(\alpha) }\bigg)^2 \Bigg)
\end{equation}
%
\en{where $\beta = \frac{b-\mu}{\sigma}$ and $\alpha = \frac{a-\mu}{\sigma}$.}
\es{donde $\beta = \frac{b-\mu}{\sigma}$ y $\alpha = \frac{a-\mu}{\sigma}$.}
%
\en{With a single-sided truncation, these functions can be simplified as,}
\es{Con un \'unico truncamiento, estas funciones se pueden simplificar como,}
%
\begin{equation*}
 E(X| a < X )   =  \mu + \sigma \frac{\N(\alpha)}{1 - \Phi(\alpha) } \ \ , \ \ V(X| a < X )  = \sigma^2 \Bigg( 1 + \bigg(\frac{\alpha \N(\alpha)}{1 - \Phi(\alpha) }\bigg) - \bigg(\frac{\N(\alpha)}{1 - \Phi(\alpha) }\bigg)^2 \Bigg) 
\end{equation*}
%
\en{Then, the Gaussian that best approximates $p(d_1)$ in a winning case is}
\es{Luego, la gaussiana que mejor aproxima a $p(d_1)$ en un caso ganador es}
%
\begin{equation}\label{eq:p*_d} \tag{\texttt{approx()}}
 \widehat{p}(d) = \N(d | \widehat{\delta}, \widehat{\vartheta}^2) =
 \begin{cases*}
 \N\Big(d \,  | \, E(d | -\varepsilon < d < \varepsilon ) , \,  V(d | -\varepsilon < d < \varepsilon ) \, \Big) & \text{tie} \\
\N\Big(d \,  | \, E(d | d > -\varepsilon ) , \,  V(d | d > -\varepsilon ) \, \Big) & \text{not tie}
  \end{cases*}
\end{equation}
%
\begin{lstlisting}[backgroundcolor=\color
{white},label=lst:pd_approx, caption=\relax, belowskip=-1.0 \baselineskip, aboveskip=-0 \baselineskip]
\end{lstlisting}
\begin{paracol}{3}
\begin{lstlisting}[backgroundcolor=\color{julia},belowskip=-0.77 \baselineskip]
tie, not_tie = true, false
\end{lstlisting}
\switchcolumn
\begin{lstlisting}[backgroundcolor=\color{python},belowskip=-0.77 \baselineskip]
tie, not_tie = True, False
\end{lstlisting}
\switchcolumn
\begin{lstlisting}[backgroundcolor=\color{r},belowskip=-0.77 \baselineskip]
tie = T; not_tie = F 
\end{lstlisting}
\end{paracol}
\begin{lstlisting}[backgroundcolor=\color{all}]
pd_approx = approx(d, margin, not_tie)
\end{lstlisting}
%
\en{Given $\widehat{p}(d_1)$, we can compute the approximate ascending message.}
\es{Dada $\widehat{p}(d_1)$, podemos calcular el mensaje ascendentes aproximado.}
%
\begin{equation}\label{eq:m^_d_fd} \tag{\texttt{approx\_lhood\_d1}}
\begin{split}
 m_{d_1 \rightarrow f_{d_1}}(d_1)   = \frac{p(d_1)}{m_{f_{d_1} \rightarrow d_1}(d_1)} 
 & \approx \frac{\widehat{p}(d_1)}{m_{f_{d_1} \rightarrow d_1}(d_1)}  \\
& = \frac{\N(d_1 \,  | \,\widehat{\delta} , \, \widehat{\vartheta}^{\,2} )}{\N(d_1 | \delta, \vartheta^2)} 
\propto N(d_1,\delta_{\div},\vartheta_{\div}^2 )
\end{split}
\end{equation}
%
\en{with}\es{con} $\delta_{\div} = \frac{\widehat{\delta}}{\widehat{\vartheta}^2} - \frac{\delta}{\vartheta^2}$ \en{and}\es{y} $\vartheta_{\div}^2 = (\frac{1}{\widehat{\vartheta}^2} - \frac{1}{\vartheta^2})^{-1}$ 
%
\begin{lstlisting}[backgroundcolor=\color
{white},label=lst:d_div, caption=\relax, belowskip=-1.0 \baselineskip, aboveskip=-0 \baselineskip]
\end{lstlisting}
\begin{lstlisting}[backgroundcolor=\color{all},belowskip=-0.77 \baselineskip]
approx_lhood_d1 = pd_approx/d
\end{lstlisting}  
\begin{paracol}{3}
\begin{lstlisting}[backgroundcolor=\color{julia}]
d_div = approx_lhood_d1.mu
v_div = approx_lhood_d1.sigma
\end{lstlisting}
\switchcolumn
\begin{lstlisting}[backgroundcolor=\color{python}]
d_div = lhood_d1_approx.mu
v_div = lhood_d1_approx.sigma
\end{lstlisting}
\switchcolumn
\begin{lstlisting}[backgroundcolor=\color{r}]
d_div = lhood_d1_approx$mu
v_div = lhood_d1_approx$sigma
\end{lstlisting}
\end{paracol}
%
\begin{equation}\label{eq:^m_fd_ta} \tag{\texttt{lhood\_ta\_approx}}
\begin{split}
\widehat{m}_{f_{d_1} \rightarrow t_a}(t_a) & =  \iint \mathbb{I}(d_1 = t_a - t_b) \N(d_1 | \delta_{\div}, \vartheta_{\div}^2) \N(t_b | \mu_b , \sigma_b^2 )  \, d{d_1} d_{t_b} \\
& = \int  \N( t_a-t_b | \delta_{\div}, \vartheta_{\div}^2) \N(t_b | \mu_b , \sigma_b^2 )  \,  d_{t_b} \\
& = \N(t_a \, | \, \mu_b + \delta_{\div} \, , \, \vartheta_{\div}^2 + \sigma_b^2) \\
\end{split}
\end{equation}
%
\begin{equation}\label{eq:^m_fta_p} \tag{\texttt{lhood\_p1\_approx}}
\begin{split}
\widehat{m}_{f_{t_a} \rightarrow p_1}(p_1) &= \iint \mathbb{I}(t_a = p_1 + p_2) \N(t_a \, | \, \mu_b + \delta_{\div} \, , \, \vartheta_{\div}^2 + \sigma_b^2) \N(p_2 | \mu_2 , \sigma_2^2 + \beta^2)  \, d{t_a} d_{p_2} \\
& = \int \N(p_1 + p_2 \, | \, \mu_b + \delta_{\div} \, , \, \vartheta_{\div}^2 + \sigma_b^2) \N(p_2 | \mu_2 , \sigma_2^2+ \beta^2 )   \, d_{p_2} \\
& = \N( p_1 \,|\,  \underbrace{\mu_b - \mu_2}_{\mu_1-\delta} + \delta_{\div}  \,,\,\vartheta_{\div}^2 + \underbrace{\sigma_b^2 + \sigma_2^2 + \beta^2}_{\vartheta^2 - (\sigma_1^2 + \beta^2)})  \\
\end{split}
\end{equation}
%
\begin{equation}\label{eq:^m_fp_s} \tag{\texttt{lhood\_s1\_approx}}
\begin{split}
\widehat{m}_{f_{p_1} \rightarrow s_1}(s_1) & = \int \N(p_1|s_1,\beta^2) \N(p_1| \mu_1 - \delta + \delta_{\div}, \vartheta_{\div}^2 + \vartheta^2 - \sigma_1^2 - \beta^2)dp_1 \\
& = \N(s_1| \mu_1 - \delta + \delta_{\div}, \vartheta_{\div}^2 + \vartheta^2 - \sigma_1^2)
\end{split}
\end{equation}
%
\begin{equation}\label{eq:^p_s} \tag{\texttt{posterior\_s1\_approx}}
 \widehat{p}(s_1) = \N(s_1|\mu_1, \sigma_1^2) \N(s_1| \mu_1 - \delta + \delta_{\div}, \vartheta_{\div}^2 + \vartheta^2 - \sigma_1^2)
\end{equation}
%
\begin{lstlisting}[backgroundcolor=\color
{white},label=lst:lhood_s1_approx, caption=\relax, belowskip=-1.0 \baselineskip, aboveskip=-0 \baselineskip]
\end{lstlisting}
\begin{paracol}{3}
\begin{lstlisting}[backgroundcolor=\color{julia},belowskip=-0.77 \baselineskip]
mu1 = prior1.mu
v2, v_div_2 = v^2, v_div^2
sigma1_2 = prior1.sigma^2
\end{lstlisting}
\switchcolumn
\begin{lstlisting}[backgroundcolor=\color{python},belowskip=-0.77 \baselineskip]
mu1 = prior1.mu
v2, v_div_2 = v**2, v_div**2 
sigma1_2 = prior1.sigma**2
\end{lstlisting}
\switchcolumn
\begin{lstlisting}[backgroundcolor=\color{r},belowskip=-0.77 \baselineskip]
mu1 = prior1$mu
v2 = v^2; v_div_2 = v_div^2
sigma1_2 = prior1$sigma^2
\end{lstlisting}
\end{paracol}
\begin{lstlisting}[backgroundcolor=\color{all}]
lhood_s1_approx = Gaussian(mu = (mu1 - d + d_div), sigma = sqrt(v_div_2 + v2 - sigma1_2) )
posterior_s1_approx = prior1 * lhood_s1_approx
\end{lstlisting}

\subsection{\en{Multiple teams}\es{Varios equipos}} \label{sec:iterative_posterior} 
%
\en{In this section we will see how to find the best approximation to the exact posterior in cases where we have more than two teams.}
\es{En esta secci\'on veremos como encontrar la mejor aproximaci\'on al posterior exacto en casos en los que tenemos m\'as de dos equipos.}
%
\en{Let's assume that $n$ agents, organized in $k$ teams $\{1, \dots, k\}$, participate in an event.}
\es{Supongamos que $n$ agentes, organizados en $k$ equipos $\{1, \dots, k\}$, participan de un evento.}
%
\en{The team assignment is represented by a partition of the set of agents, $A$, into $k$ non-overlapping subsets, $A_i$.}
\es{La asignaci\'on de equipos se representa como una partici\'on del conjunto de agentes, $A$, en $k$ subconjuntos disjuntos, $A_i$.}
%
\en{Due to the transitivity of the result, it is enough to evaluate $k-1$ distribution of differences $d_i$ between consecutive teams in the ranking.}
\es{Gracias a la transitividad de los resultados, es suficiente con evaluar $k-1$ distribuciones de diferencia $d_i$ entre equipos consecutivos en el ranking.}
%
\en{For that purpose we define the order of teams that arises from the observed result, $ o:= (o_1, \ dots, o_k) $, where $ o_1 $ indicates the winning team, and in general $ o_i = e $ indicates that the team $ e $ was located at position $ i $.}
\es{Para ese prop\'osito definimos el orden de equipos que surge del resultado observdo, $o := (o_1, \dots, o_k)$, donde $o_1$ indica el equipo ganador, y en general $o_i = e$ indica que el equipo $e$ qued\'o ubicado en la posici\'on $i$.}
%
\begin{figure}[t!]
  \centering
  \scalebox{.9}{
  \tikz{ %
        \node[factor] (fr) {} ;
        \node[const, above=of fr] (nfr) {$f_r$}; %
	\node[const, above=of nfr] (dfr) {\large $\mathbb{I}(d_i>0)$}; %
        \node[latent, left=of fr] (d) {$d_j$} ; %
        \node[factor, left=of d] (fd) {} ;
        \node[const, above=of fd] (nfd) {$f_d$}; %
        \node[const, above=of nfd] (dfd) {\large $\mathbb{I}(d_i=t_{o_i} - t_{o_{i+1}})$}; %
        
        \node[latent, left=of fd,xshift=-0.9cm] (t) {$t_e$} ; %
        \node[factor, left=of t] (ft) {} ;
        \node[const, above=of ft] (nft) {$f_t$}; %
        \node[const, above=of nft,xshift=0.5cm] (dft) {\large $\mathbb{I}(t_e = \sum_{i \in A_e} p_i)$}; %

        \node[latent, left=of ft] (p) {$p_i$} ; %
        \node[factor, left=of p] (fp) {} ;
        \node[const, above=of fp] (nfp) {$f_p$}; %
        \node[const, above=of nfp] (dfp) {\large $N(p_i|s_i,\beta^2)$}; %

        \node[latent, left=of fp] (s) {$s_i$} ; %
        \node[factor, left=of s] (fs) {} ;
        \node[const, above=of fs] (nfs) {$f_s$}; %
        \node[const, above=of nfs] (dfs) {\large $N(s_i|\mu_i,\sigma^2)$}; %

        \edge[-] {d} {fr};
	\edge[-] {fd} {d};
        \edge[-] {fd} {t};
        \edge[-] {t} {ft};
        \edge[-] {ft} {p};
        \edge[-] {p} {fp};
        \edge[-] {fp} {s};
        \edge[-] {s} {fs};

        \plate {personas} {(p)(s)(fs)(nfs)(dfp)(dfs)} {$i \in A_e$}; %
        \node[invisible, below=of ft, yshift=-0.6cm] (inv_below_e) {};
	\node[invisible, above=of ft, yshift=1.1cm] (inv_above_e) {};
	\plate {equipos} {(personas) (t)(ft)(dft) (inv_above_e) (inv_below_e)} {$  \text{\en{Let $A$ be a partition of agents }\es{Sea $A$ una partici\'on de agentes } }$  \hspace{3cm} $0 < e \leq |A|$}; %
	\node[invisible, below=of fr, yshift=-0.6cm] (inv_below) {};
	\node[invisible, above=of fr, yshift=1.1cm] (inv_above) {};
	\plate {comparaciones} {(fd) (dfd) (d) (fr) (dfr) (inv_below) (inv_above)} {$o:=$\en{ observed order}\es{ orden observado} \hspace{1cm} $1 \leq i < |A|$};
    }  
    }
  \caption{
  \en{Factorization of the generic multi-team model.}
  \es{Factorizaci\'on del modelo de multiequipos gen\'erico.}
  %
  \en{Plates indicate replication.}
  \es{Las placas indican replicaci\'on.}
  %
  }
  \label{fig:factorGraph_trueskill}
\end{figure}
%
\en{In Figure ~\ref{fig:factorGraph_trueskill} we show the factorization of the general TrueSkill model.}
\es{En la figura~\ref{fig:factorGraph_trueskill} mostramos la factorizaci\'on del modelo general de TrueSkill.}
%
\en{Let's see how to solve a game with 3 teams.}
\es{Veamos como se resuelve un caso de 3 equipos.}
%
\en{For the users of the package there is no difference regarding the previous case.}
\es{Para los usuarios del paquete no hay ninguna diferencia respecto del caso anterior.}
%
\vspace{-0.6cm}
\begin{figure}[H]
\begin{lstlisting}[backgroundcolor=\color
{white},label=lst:multi_team_game, caption=\relax, belowskip=-1.5 \baselineskip, aboveskip=-0 \baselineskip]
\end{lstlisting}
\begin{subfigure}[t]{0.32\textwidth}
\begin{lstlisting}[backgroundcolor=\color{julia}]
team_a = [ r1 ]
team_b = [ r2, r3 ]
team_b = [ r4 ]
teams= [team_a,team_b,team_c]
result = [1,0,0]
g = Game(teams,result,p_draw)
p = posteriors(g)
\end{lstlisting} 
\end{subfigure}
\begin{subfigure}[t]{0.32\textwidth}
\begin{lstlisting}[backgroundcolor=\color{python}]
team_a = [ r1 ]
team_b = [ r2, r3 ]
team_b = [ r4 ]
teams= [team_a,team_b,team_c]
result = [1,0,0]
g = Game(teams,result,p_draw)
p = g.posteriors()
\end{lstlisting}
\end{subfigure}
\begin{subfigure}[t]{0.32\textwidth}
\begin{lstlisting}[backgroundcolor=\color{r}]
teams = list()
teams$team_a = c(r1)
teams$team_b = c(r2, r3)
teams$team_c = c(r4)
result = c(1,0,0)
g = Game(teams,result,p_draw)
p = g$posteriors()
\end{lstlisting}   
\end{subfigure}
\end{figure}
\vspace{-0.6cm}
%
\en{In these cases it is impossible to perform a one-shot inference because of the mutual dependency between the distribution of difference, $p(d_i)$.}
\es{En estos caso es imposible realizar la inference de una sola pasada por la dependencia mutua entre las distribuciones de diferencia, $p(d_i)$.}
%
\en{The basic idea is to update repeatedly forward and backward all messages in the shortest path between any two marginals $p(d_i)$ until convergence, as shown in Figure~\ref{fig:ep_ts}.}
\es{La idea b\'asica es actualizar repetidamente hacia adelante y hacia atr\'as todos los mensajes en el camino m\'as corto entre dos marginales $p(d_i)$ hasta la convergencia, como se muestra en la figura~\ref{fig:ep_ts}.}
%
\begin{figure}[H]
  \centering
  \scalebox{.9}{
\tikz{ %        
        \node[factor, xshift=-5cm] (fta) {} ;
        \node[const, right=of fta] (nfta) {$f_{t_a}$}; %
        \node[latent, below=of fta,yshift=-0.5cm] (ta) {$t_a$} ; %
        
        \node[factor] (ftb) {} ;
        \node[const, right=of ftb] (nftb) {$f_{t_b}$}; %
        \node[latent, below=of ftb,yshift=-0.5cm] (tb) {$t_b$} ; %
        
        \node[factor, xshift=5cm] (ftc) {} ;
        \node[const, right=of ftc] (nftc) {$f_{t_c}$}; %        
        \node[latent, below=of ftc,yshift=-0.5cm] (tc) {$t_c$} ; %
        
        \node[factor, below=of tb, xshift=-3cm] (fd1) {} ;
        %\node[const, left=of fd1] (nfd1) {$f_{d_0}$}; %        
        \node[latent, below=of fd1,yshift=-1cm] (d1) {$d_1$} ; %
        \node[factor, below=of d1,yshift=-1cm] (fr1) {} ;
        
        \node[factor, below=of tb, xshift=3cm] (fd2) {} ;
        %\node[const, above=of fd2] (nfd2) {$f_{d_{1}}$}; %        
        \node[latent, below=of fd2,yshift=-1cm] (d2) {$d_2$} ; %
        \node[factor, below=of d2,yshift=-1cm] (fr2) {} ;
        
        \edge[-] {ta} {fta,fd1}
        \edge[-] {tb} {ftb,fd1,fd2}
        \edge[-] {tc} {ftc,fd2}
        \edge[-] {d1} {fd1,fr1}
        \edge[-] {d2} {fd2,fr2}
        
        \path[draw, -latex, fill=black!50,sloped] (fd1) edge[bend left,draw=black!50] node[midway,above,color=black!75] {\scriptsize  \texttt{lhood\_lose\_tb}} (tb);
        
        \path[draw, -latex, fill=black!50,sloped] (tb) edge[bend left,draw=black!50] node[midway,below,color=black!75] {\scriptsize \texttt{ \ posterior\_lose\_tb}} (fd1);
        
        \path[draw, -latex, fill=black!50,sloped] (fd2) edge[bend right,draw=black!50] node[midway,above,color=black!75] {\scriptsize \texttt{lhood\_win\_tb}} (tb);
        
        \path[draw, -latex, fill=black!50,sloped] (tb) edge[bend right,draw=black!50] node[midway,below,color=black!75] {\scriptsize \texttt{posterior\_win\_tb}} (fd2);
        
        \path[draw, -latex, fill=black!50,sloped] (fta) edge[bend left,draw=black!50] node[midway,above,color=black!75] {\scriptsize \texttt{prior\_ta}} (ta);
        
        \path[draw, -latex, fill=black!50,sloped] (fr1) edge[bend left,draw=black!50] node[midway,above,color=black!75, rotate=180] {\scriptsize \texttt{lhood\_d1}} (d1);
        
        \path[draw, -latex, fill=black!50,sloped] (d1) edge[bend left,draw=black!50] node[midway,above,color=black!75] {\scriptsize \texttt{lhood\_d1\_approx}} (fd1);
        
        \path[draw, -latex, fill=black!50,sloped] (fd1) edge[bend left,draw=black!50] node[midway,above,color=black!75] {\scriptsize \texttt{prior\_d1}} (d1);
        
        \path[draw, -latex, fill=black!50,sloped] (tc) edge[bend left,draw=black!50] node[midway,above,color=black!75] {\scriptsize \texttt{lhood\_tc\_approx}} (ftc);
        
        
        %\path[draw, -latex, fill=black!50,sloped] (fr2) edge[bend left,draw=black!50] node[midway,above,color=black!75, rotate=180] {\scriptsize \textbf{5:} \emph{likelihood}$(d_{0})$} (d2);
        
        %\path[draw, -latex, fill=black!50,sloped] (fd2) edge[bend left,draw=black!50] node[midway,above,color=black!75] {\scriptsize \textbf{4:} \emph{prior}$(d_{0})$} (d2);
        
        
} 
}
\caption{
 \en{Factorization of a game with 3 teams.}
 \es{Factorizaci\'on de una partida con 3 equipos.}
 %
 \en{We only show factors from the teams to the results.}
 \es{Mostramos s\'olo factores desde los equipos hasta los resultados.}
 %
 \en{The names will be used to explain the iterative procedure known as loopy belief propagation.}
 \es{Los nombres se usar\'an para explicar el procedimiento iterativo conocido como \emph{loopy belief propagatiion}.}
}
\label{fig:ep_ts}
\end{figure}
%
\en{First we define the Prior of the teams.}
\es{Primero definimos el prior de los equipos.}
%
\begin{lstlisting}[backgroundcolor=\color
{white},label=lst:team_prior, caption=\relax, belowskip=-1.0 \baselineskip, aboveskip=-0 \baselineskip]
\end{lstlisting}
\begin{lstlisting}[backgroundcolor=\color{all}]
prior_ta= performance(team_a); prior_tb= performance(team_b); prior_tc= performance(team_c)
\end{lstlisting}
%
\en{Messages that are not yet defined are initialized with a neutral form, such as a Gaussian distribution with infinite variance.}
\es{Los mensajes que a\'un no est\'an definidos se inicializan con una forma neutra, como una distribuci\'on gaussiana con varianza infinita.}
%
\begin{lstlisting}[backgroundcolor=\color
{white},label=lst:init_lhood, caption=\relax, belowskip=-1.0 \baselineskip, aboveskip=-0 \baselineskip]
\end{lstlisting}
\begin{paracol}{3}
\begin{lstlisting}[backgroundcolor=\color{julia},belowskip=-0.77 \baselineskip]
N_inf = Gaussian(0., Inf)
\end{lstlisting}
\switchcolumn
\begin{lstlisting}[backgroundcolor=\color{python},belowskip=-0.77 \baselineskip]
N_inf = Gaussian(0, inf)
\end{lstlisting}
\switchcolumn
\begin{lstlisting}[backgroundcolor=\color{r},belowskip=-0.77 \baselineskip]
N_inf = Gaussian(0, Inf)
\end{lstlisting}
\end{paracol}
\begin{lstlisting}[backgroundcolor=\color{all}]
lhood_lose_ta = N_inf; lhood_win_ta = N_inf
lhood_lose_tb = N_inf; lhood_win_tb = N_inf
lhood_lose_tc = N_inf; lhood_win_tc = N_inf
\end{lstlisting}
%
\en{We compute the margins for each comparison, $d_i$.}
\es{Calculamos el los margenes para cada comparaci\'on $d_1$.}
%
\begin{lstlisting}[backgroundcolor=\color
{white},label=lst:margin, caption=\relax, belowskip=-1.0 \baselineskip, aboveskip=-0 \baselineskip]
\end{lstlisting}
\begin{lstlisting}[backgroundcolor=\color{all}]
margin = compute_margin(p_draw, sqrt(3)*BETA)
\end{lstlisting}
%
\en{Let's start the iterative process by computing the messages of one distributions of difference, for example $d_1$.}
\es{Empezmos el proceso iterativo calculando una de las distribuciones de diferencia, por ejemplo $d_1$.}
%
\begin{lstlisting}[backgroundcolor=\color
{white},label=lst:d1, caption=\relax, belowskip=-1.0 \baselineskip, aboveskip=-0 \baselineskip]
\end{lstlisting}
\begin{lstlisting}[backgroundcolor=\color{all}]
posterior_win_ta = prior_ta * lhood_lose_ta 
posterior_lose_tb = prior_tb * lhood_win_tb
prior_d1 = posterior_win_ta - posterior_lose_tb
lhood_d1_approx = approx(prior_d1, margin, not_tie) / prior_d1 
\end{lstlisting}
%
\en{The message \texttt{lhood\_d1\_approx} allows us to update the distribution of performance $t_b$.}
\es{El mensahe \texttt{lhood\_d1\_approx} nos permite actualizar la distribuci\'on de rendimietos $t_b$.}
%
\begin{lstlisting}[backgroundcolor=\color
{white},label=lst:tb_lose, caption=\relax, belowskip=-1.0 \baselineskip, aboveskip=-0 \baselineskip]
\end{lstlisting}
\begin{lstlisting}[backgroundcolor=\color{all}]
lhood_lose_tb = posterior_win_ta - lhood_d1_approx 
\end{lstlisting}
%
\en{Then we compute the the messages of the next distribution of differences $d_2$.}
\es{Luego computamos los mensajes de la siguiente distirbuci\'on de diferencia $d_2$.}
%
\begin{lstlisting}[backgroundcolor=\color
{white},label=lst:d2, caption=\relax, belowskip=-1.0 \baselineskip, aboveskip=-0 \baselineskip]
\end{lstlisting}
\begin{lstlisting}[backgroundcolor=\color{all}]
posterior_win_tb = prior_tb * lhood_lose_tb
posterior_lose_tc = prior_tc * lhood_win_tc 
prior_d2 = posterior_win_tb - posterior_lose_tc
lhood_d2_approx = approx(prior_d2, margin, tie) / prior_d2 
\end{lstlisting}
%
\en{The message \texttt{lhood\_d2\_approx} allows us to update the distribution of performance $t_b$.}
\es{El mensaje \texttt{lhood\_d2\_approx} nos permite actualizar la distribuci\'on de rendimiento $t_b$.}
%
\begin{lstlisting}[backgroundcolor=\color
{white},label=lst:tb_win, caption=\relax, belowskip=-1.0 \baselineskip, aboveskip=-0 \baselineskip]
\end{lstlisting}
\begin{lstlisting}[backgroundcolor=\color{all}]
lhood_win_tb = posterior_lose_tc + lhood_d2_approx
\end{lstlisting}
%
\en{Then, loop code \ref{lst:d1}, \ref{lst:tb_lose}, \ref{lst:d2}, \ref{lst:tb_win} until convergence.}
\es{Entonces, repite los c\'odigos \ref{lst:d1}, \ref{lst:tb_lose}, \ref{lst:d2}, \ref{lst:tb_win} hasta alcanzar convergencia.}
%
\en{Once the iteration is finished, we send the last ascending message to the teams on the sides.}
\es{Finalizada la iteraci\'on, enviamos los \'ultimos mensajes ascendentes a los equipos de los bordes.}
%
\begin{lstlisting}[backgroundcolor=\color
{white},label=lst:te_side, caption=\relax, belowskip=-1.0 \baselineskip, aboveskip=-0 \baselineskip]
\end{lstlisting}
\begin{lstlisting}[backgroundcolor=\color{all}]
lhood_win_ta = posterior_lose_tb + lhood_d1_approx
lhood_lose_tc = posterior_win_tb - lhood_d2_approx
\end{lstlisting}
%
\en{We are now able to compute the likelihood of each team.}
\es{Ahora estamos en condiciones de computar los likelihood de cada equipo.}
%
\begin{lstlisting}[backgroundcolor=\color
{white},label=lst:lhood_te_approx, caption=\relax, belowskip=-1.0 \baselineskip, aboveskip=-0 \baselineskip]
\end{lstlisting}
\begin{lstlisting}[backgroundcolor=\color{all}]
lhood_ta_approx = lhood_lose_ta * lhood_win_ta 
lhood_tb_approx = lhood_lose_tb * lhood_win_tb
lhood_tc_approx = lhood_lose_tc * lhood_win_tc
\end{lstlisting}
%
\en{As an example, we compute the posterior of player 2.}
\es{A modo de ejemplo, computamos el posterior del jugador 2.}
%
\begin{lstlisting}[backgroundcolor=\color
{white},label=lst:posterior_s2_approx, caption=\relax, belowskip=-1.0 \baselineskip, aboveskip=-0 \baselineskip]
\end{lstlisting}
\begin{paracol}{3}
\begin{lstlisting}[backgroundcolor=\color{julia},belowskip=-0.77 \baselineskip]
prior2 = r2.N
ex = exclude(prior_tb,prior2)
\end{lstlisting}
\switchcolumn
\begin{lstlisting}[backgroundcolor=\color{python},belowskip=-0.77 \baselineskip]
prior2 = r2.N
ex = prior_tb.exclude(prior2)
\end{lstlisting}
\switchcolumn
\begin{lstlisting}[backgroundcolor=\color{r},belowskip=-0.77 \baselineskip]
prior2 = r2$N
ex = prior_tb$exclude(prior2)
\end{lstlisting}
\end{paracol}
\begin{lstlisting}[backgroundcolor=\color{all}]
team_b_without_s2 = ex
lhood_s2_approx = lhood_ta_approx - team_b_without_s2 
posterior_s2_approx = prior2 * lhood_s2_approx 
\end{lstlisting}

\subsection{History} \label{sec:throguthTime}

\en{In this section we will see how to find the best approximation to the exact posterior in a history of events in which $n$ players compete over a period of $T$ time steps or temporal batches (e.g. day, week, month, year).}
\es{En esta secci\'on vamos a ver c\'omo se obtiene la mejor aproximaci\'on al posterior exacto en una historia de eventos en la que $n$ agentes compiten durante $T$ pasos temporales o \emph{batches} (e.g. d\'ia, semana, mes, a\~no).}
% 
\en{Within each time step $t$ an agent $i$ can participate in $K$ events, $K_{i,t}$.}
\es{Al interior de cada paso temporal $t$ un agente $i$ puede participar en $K$ eventos, $K_{i_t}$}
%
\begin{figure}[h!]
  \centering
  \scalebox{.9}{
\tikz{ %
        \node[latent] (s0) {$s_{i_{t-1}}$} ; %

        \node[factor, right=of s0,xshift=1cm ] (fs1) {} ;
        \node[const, above=of fs1] (nfs1) {$f_{s_{i_{t}}}$}; %

        \node[latent, right=of fs1, xshift=1.25cm] (s1) {$s_{i_t}$} ; %

        \node[factor, right=of s1, xshift=1.25cm ] (fs2) {} ;
        \node[const, above=of fs2] (nfs2) {$f_{s_{i_{t+1}}}$}; %

        \node[latent, right=of fs2,xshift=1cm] (s2) {$s_{i_{t+1}}$} ; %

        \node[factor, below=of s1,xshift=-1.4cm,yshift=-1cm] (fp0) {} ;
        \node[const, right=of fp0] (nfp0) {$f_{p_{i_t}(1)}$}; %

        \node[factor, color=white, below=of s1] (fp1) {} ;
        \node[const, below=of fp1, yshift=0.2cm] (nfp1) {$\dots$}; %

        \node[factor, below=of s1,xshift=1.4cm,yshift=-1cm] (fp2) {} ;
        \node[const, left=of fp2] (nfp2) {$f_{p_{i_t}(k)}$}; %

        \node[latent, below=of fp0] (p0) {\footnotesize$p_{i_t}(1)$} ; %
        %\node[latent, below=of fp1] (p1) {\footnotesize$p_i^{t}(2)$} ; %
        \node[latent, below=of fp2] (p2) {\footnotesize$p_{i_t}(k)$} ; %

        %\draw[bend right=90] (fs1) arc (s1) node[midway,above]{label};
        %\draw[bend left,->]  (fs1) to node [auto] {Link} (s1);
        \edge[-] {s1} {fp0,fp1,fp2};
        \edge[-] {fp0} {p0};
        %\edge[-] {fp1} {p1};
        \edge[-] {fp2} {p2};
        \edge[-] {fs1} {s0,s1};
        \edge[-] {fs2} {s1,s2};
        %\edge[bend right] {s0} {fs1};
        \path[draw, -latex, fill=black!50] (s0) edge[bend right,draw=black!50] node[midway,below,color=black!75] {\scriptsize \emph{posterior}$(t-1)$} (fs1);
        \path[draw, -latex, fill=black!50] (fs1) edge[bend left,draw=black!50] node[midway,above,color=black!75] {\scriptsize \emph{prior}$(t)$} (s1);
        \path[draw, -latex, fill=black!50] (s2) edge[bend left,draw=black!50] node[midway,below,color=black!75] {\scriptsize \emph{\ \ inversePosterior}$(t+1)$} (fs2);
        \path[draw, -latex, fill=black!50] (fs2) edge[bend right,draw=black!50] node[midway,above,color=black!75] {\scriptsize \emph{inversePrior}$(t)$} (s1);
        \path[draw, -latex, fill=black!50,sloped] (fp0) edge[bend left,draw=black!50] node[midway,above,color=black!75] {\scriptsize \emph{likelihood}$(t,k)$} (s1);
        \path[draw, -latex, fill=black!50,sloped] (s1) edge[bend left,draw=black!50] node[midway,above,color=black!75] {\scriptsize \emph{\ \ withinPrior}$(t,k)$} (fp2);
}
}
\caption{
\en{Factor graph of a history of events around a skill variable $s$ of an agent $i$ at a time step $t$.}
\es{Grafo de factorizaci\'on de una historia de eventos alrededor de la habilidad $s$ de un agente $i$ en el paso temporal $t$}
%
\en{The variables $ p_ {i_t} (j) $ represents the performance $ p $ that player $i$ had in their $j$-th game within the time step $ t $.}
\es{Las variables $p_{i_t}(j)$ representa el rendimiento $p$ que ese jugador tuvo en la $j$-\'esima partida al interior del paso temporal $t$.}
%
\en{The arrows represents messages computed by the sum-product algorithm.}
\es{Las flechas representan los mensajes computados por el algoritmo de sum-product.}
%
\en{The names were selected for the sole purpose of simplifying the notation.}
\es{Los nombres fueron elegidos solamente para simplificar la notaci\'on.}
}
\label{fig:history}
\end{figure}
%
\en{In the Figure~\ref{fig:history} we show a schematic representation of the factor graph of a history of events.}
\es{En la figura~\ref{fig:history} mostramos representaci\'on esquem\'atica del grapho de factorizaci\'on de un historia de eventos.}
%
\en{By the sum-product algorithm, we know that the marginal distribution of any variable is the product of the messages it receives from its neighbours.}
\es{Mediante el algoritmo de sum-product, sabemos que la distribuci\'on marginal de cualquier variable es el producto de los mensajes que esta recibe de sus vecinos.}
\en{Replaced the messages by the selected names, the distribution can be expressed as,}
\es{Remplazando los mensajes por los nombres seleccionados, esta distribuci\'on se puede expresar como,}
%
\begin{equation}
 p(s_{i_t}) = \emph{prior}_i(t) \cdot \emph{inversePrior}_i(t) \cdot \prod_{k=1}^{K_{i_t}} \emph{likelihood}_i(t,k)
\end{equation}
%
\en{The \emph{prior} and \emph{inversePrior} messages are the neighboring skill estimates, to which some uncertainty $\gamma$ is added due to the time step.}
\es{Los mensajes \emph{prior} e \emph{inversePrior} son las estimaciones de habilidad vecinas, a las que se le agrega cierta incertidumbre $\gamma$ por el paso temporal.}
%
\begin{equation*}
\emph{prior}_i(t) = \N(s_{i_t}|s_{i_{t-1}}, \gamma^2) \ \ \ \ \ \ \emph{inversePrior}_i(t) = \N(s_{i_t}|s_{i_{t+1}}, \gamma^2)
\end{equation*}
%
\en{And the likelihoods of events are computed following the section~\ref{sec:iterative_posterior}, using as prior all the information except that of the event of interest.}
\es{Y las verosimilituides se computan siguiendo la secci\'on~\ref{sec:iterative_posterior}, usando como prior toda la informaci\'on salvo la de la partida de inter\'es.}
%
 \begin{equation}
 \emph{withinPrior}_i(t,k) = \emph{prior}_i(t) \cdot \emph{inversePrior}_i(t) \cdot \prod_{\hfrac{q=1}{q\neq k}}^{K_{i_t}} \emph{likelihood}_i(t,q) = \frac{p(s_i^t) }{\emph{likelihood}_i(t,k)}
 \end{equation}
% 
\en{There is a mutual dependency between forward and backward messages that make imposible a one shot inference iteration.}
\es{Existe una dependencia muta entre los mensajes hacia adelante y hacia atras, haciendo imposible una inferencia de \'unica iteraci\'on.}
%
\en{The basic idea is to update repeatedly forward and backward until convergence, making sure that the effect of the previous update is removed before the new effect is added.}
\es{La idea b\'asica es actualizar repetidas veces los mensajes hacia adelante y hacia atras hasta la convergencia, asegur\'andonos que el efecto de la actualizaci\'on previa sea removida antes de que el nuevo efecto se agregue.}
%
\en{The messages that are not yet defined, for example the inverse prior in the first forward pass, are replace it by a neutral form like a unit scalar or a Gaussian distribution with infinite variance.}
\es{Los mensajes que no est\'an todav\'ia definidos, por ejemplo el prior inverso en la primer pasada hacia adelante, es remplazado por una forma neutral como un escalar unidad o una distribuci\'on gausiana con infinita varianza.}
%
\en{Finally, the messages that the variable $s_{i_t}$ sends to the past and the future are,}
\es{Finalmente, los mensajes que la variable $s_{i_t}$ env\'ia al pasado y al futuro son,}
%
\begin{equation*}
  \emph{posterior}_i(t-1) = \frac{p(s_{i_{t-1}})}{\emph{inversePrior}_i(t-1)} \ \ \ , \ \ \ \emph{inversePosterior}_i(t+1) = \frac{p(s_{i_{t+1}})}{\emph{prior}_i(t+1)}
\end{equation*}



%

\begin{lstlisting}[backgroundcolor=\color
{white},label=lst:posterior_s2_approx, caption=\relax, belowskip=-1.0 \baselineskip, aboveskip=-0 \baselineskip]
\end{lstlisting}
\begin{paracol}{3}
\begin{lstlisting}[backgroundcolor=\color{julia},belowskip=-0.77 \baselineskip]
prior2 = r2.N
ex = exclude(prior_tb,prior2)
\end{lstlisting}
\switchcolumn
\begin{lstlisting}[backgroundcolor=\color{python},belowskip=-0.77 \baselineskip]
prior2 = r2.N
ex = prior_tb.exclude(prior2)
\end{lstlisting}
\switchcolumn
\begin{lstlisting}[backgroundcolor=\color{r},belowskip=-0.77 \baselineskip]
prior2 = r2$N
ex = prior_tb$exclude(prior2)
\end{lstlisting}
\end{paracol}
\begin{lstlisting}[backgroundcolor=\color{all}]
team_b_without_s2 = ex
lhood_s2_approx = lhood_ta_approx - team_b_without_s2 
posterior_s2_approx = prior2 * lhood_s2_approx 
\end{lstlisting}

\en{At each forward pass we store each forward message, i.e. \emph{prior}$_i(t+1)$.}
\es{En cada pasada hacia adelante, guardamos cada mensaje hacia adelante, i.e. \emph{prior}$_i(t+1)$.}
%
\en{And at each backward pass we compute the backward message, i.e. \emph{inversePrior}$_i(t-1)$.}
\es{En cada pasada hacia atr\'as computamos el mensaje hacia atr\'as, i.e. \emph{inversePrior}$_i(t-1)$.}
%
\subsection{Data Structures} \label{sec:estructuras} 

% 
% 
% 
% \begin{leftbar}
% Note that around the \verb|{equation}| above there should be no spaces (avoided
% in the {\LaTeX} code by \verb|%| lines) so that ``normal'' spacing is used and
% not a new paragraph started.
% \end{leftbar}
% 
% 
% \begin{leftbar}
% As the synopsis above is a code listing that is not meant to be executed,
% one can use either the dedicated \verb|{Code}| environment or a simple
% \verb|{verbatim}| environment for this. Again, spaces before and after should be
% avoided.
% 
% Finally, there might be a reference to a \verb|{table}| such as
% Table~\ref{tab:overview}. Usually, these are placed at the top of the page
% (\verb|[t!]|), centered (\verb|\centering|), with a caption below the table,
% column headers and captions in sentence style, and if possible avoiding vertical
% lines.
% \end{leftbar}

% \begin{table}[t!]
% \centering
% \begin{tabular}{lllp{7.4cm}}
% \hline
% Type           & Distribution & Method   & Description \\ \hline
% GLM            & Poisson      & ML       & Poisson regression: classical GLM,
%                                            estimated by maximum likelihood (ML) \\
% \end{tabular}
% \caption{\label{tab:overview} Overview of various count regression models. The
% table is usually placed at the top of the page (\texttt{[t!]}), centered
% (\texttt{centering}), has a caption below the table, column headers and captions
% are in sentence style, and if possible vertical lines should be avoided.}
% \end{table}
%











































\section{Summary and discussion} \label{sec:summary}

\begin{leftbar}
As usual \dots
\end{leftbar}

























%% -- Optional special unnumbered sections -------------------------------------

\section*{Computational details}

\begin{leftbar}
If necessary or useful, information about certain computational details
such as version numbers, operating systems, or compilers could be included
in an unnumbered section. Also, auxiliary packages (say, for visualizations,
maps, tables, \dots) that are not cited in the main text can be credited here.
\end{leftbar}

The results in this paper were obtained using
\proglang{R}~3.4.1 with the
\pkg{MASS}~7.3.47 package. \proglang{R} itself
and all packages used are available from the Comprehensive
\proglang{R} Archive Network (CRAN) at
\url{https://CRAN.R-project.org/}.


\section*{Acknowledgments}

\begin{leftbar}
All acknowledgments (note the AE spelling) should be collected in this
unnumbered section before the references. It may contain the usual information
about funding and feedback from colleagues/reviewers/etc. Furthermore,
information such as relative contributions of the authors may be added here
(if any).
\end{leftbar}

 
\newpage
%% -- Bibliography -------------------------------------------------------------
%% - References need to be provided in a .bib BibTeX database.
%% - All references should be made with \cite, \citet, \citep, \citealp etc.
%%   (and never hard-coded). See the FAQ for details.
%% - JSS-specific markup (\proglang, \pkg, \code) should be used in the .bib.
%% - Titles in the .bib should be in title case.
%% - DOIs should be included where available.

\bibliography{../../bibliografia/journalsAbbr,../../bibliografia/Gaming/gaming}
% All references should be made with \verb|\cite|, \verb|\citet|, \verb|\citep|,
% \verb|\citealp|
% \begin{itemize}
%   \item JSS-specific markup (\verb|\proglang|, \verb|\pkg|, \verb|\code|) should
%     be used in the references.
%   \item Titles should be in title case.
%   \item Journal titles should not be abbreviated and in title case.
%   \item DOIs should be included where available.
%   \item Software should be properly cited as well. For \proglang{R} packages
%     \code{citation("pkgname")} typically provides a good starting point.
% \end{itemize}
% \end{leftbar}
% 

\newpage
\section{Appendix} \label{app:technical}
% Appendices can be included after the bibliography (with a page break). Each
% section within the appendix should have a proper section title (rather than
% just \emph{Appendix}).

% 
% \section{Propiedades de las funciones de densidad Normales}
% 
% \subsection{Multiplicacion de normales}\label{multiplicacion_normales}
% 
% Luego, el problema que tenemos que resolver es
% \begin{equation}
%  \int N(x;\mu_1,\sigma_1^2)N(x;\mu_2,\sigma_2^2) dx
% \end{equation}
% 
% Por defnici\'on,
% \begin{equation}
% \begin{split}
%  N(x;y,\beta^2)N(x;\mu,\sigma^2) & = \frac{1}{\sqrt{2\pi}\sigma_1}e^{-\frac{(x-\mu_1)^2}{2\sigma_1^2}} \frac{1}{\sqrt{2\pi}\sigma_2}e^{-\frac{(x-\mu_2)^2}{2\sigma_2^2}}  \\
%  & = \frac{1}{2\pi\sigma_1\sigma_2}\text{exp}\Bigg(-\underbrace{\left( \frac{(x-\mu_1)^2}{2\sigma_1^2} + \frac{(x-\mu_2)^2}{2\sigma_2^2} \right)}_{\theta} \Bigg)
% \end{split}
% \end{equation}
% 
% Luego,
% \begin{equation}
%  \theta = \frac{\sigma_2^2(x^2 + \mu_1^2 - 2x\mu_1) + \sigma_1^2(x^2 + \mu_2^2 - 2x\mu_2) }{2\sigma_1^2\sigma_2^2}
% \end{equation}
% 
% Expando y reordeno los factores por potencias de $x$
% \begin{equation}
%  \frac{(\sigma_1^2 + \sigma_2^2) x^2 - (2\mu_1\sigma_2^2 + 2\mu_2\sigma_1^2) x + (\mu_1^2\sigma_2^2 + \mu_2^2\sigma_1^2)}{2\sigma_1^2\sigma_2^2}
% \end{equation}
% 
% Divido al numerador y el denominador por el factor de $x^2$
% \begin{equation}
%  \frac{x^2 - 2\frac{(\mu_1\sigma_2^2 + \mu_2\sigma_1^2)}{(\sigma_1^2 + \sigma_2^2) } x + \frac{(\mu_1^2\sigma_2^2 + \mu_2^2\sigma_1^2)}{(\sigma_1^2 + \sigma_2^2) }}{2\frac{\sigma_1^2\sigma_2^2}{(\sigma_1^2 + \sigma_2^2)}}
% \end{equation}
% 
% Esta ecuaci\'on es cuadr\'atica en x, y por lo tanto es proporcional a una funci\'on de densidad gausiana con desv\'io
% \begin{equation}
% \sigma_{\times} = \sqrt{\frac{\sigma_1^2\sigma_2^2}{\sigma_1^2+\sigma_2^2}}
% \end{equation}
% 
% y media
% \begin{equation}
%  \mu_{\times} = \frac{(\mu_1\sigma_2^2 + \mu_2\sigma_1^2)}{(\sigma_1^2 + \sigma_2^2) }
% \end{equation}
% 
% Dado que un t\'ermino $\varepsilon = 0$ puede ser agregado para completar el cuadrado en $\theta$, esta prueba es suficiente cuando no se necesita una normalizaci\'on.
% Sea,
% \begin{equation}
%  \varepsilon = \frac{\mu_{\times}^2-\mu_{\times}^2}{2\sigma_{\times}^2} = 0
% \end{equation}
% 
% Al agregar este t\'ermino a $\theta$ tenemos
% \begin{equation}
%  \theta = \frac{x^2 - 2\mu_{\times}x + \mu_{\times}^2 }{2\sigma_{\times}^2} + \underbrace{\frac{ \frac{(\mu_1^2\sigma_2^2 + \mu_2^2\sigma_1^2)}{(\sigma_1^2 + \sigma_2^2) } - \mu_{\times}^2}{2\sigma_{\times}^2}}_{\varphi}
% \end{equation}
% 
% Reorganizando el t\'ermino $\varphi$
% \begin{equation}
% \begin{split}
% \varphi & = \frac{\frac{(\mu_1^2\sigma_2^2 + \mu_2^2\sigma_1^2)}{(\sigma_1^2 + \sigma_2^2) } - \left(\frac{(\mu_1\sigma_2^2 + \mu_2\sigma_1^2)}{(\sigma_1^2 + \sigma_2^2) }\right)^2 }{2\frac{\sigma_1^2\sigma_2^2}{\sigma_1^2+\sigma_2^2}}  \\
% & = \frac{(\sigma_1^2 + \sigma_2^2)(\mu_1^2\sigma_2^2 + \mu_2^2\sigma_1^2) - (\mu_1\sigma_2^2 + \mu_2\sigma_1^2)^2}{\sigma_1^2 + \sigma_2^2}\frac{1}{2\sigma_1^2\sigma_2^2} \\[0.3cm]
% & = \frac{(\mu_1^2\sigma_1^2\sigma_2^2 + \cancel{\mu_2^2\sigma_1^4} + \bcancel{\mu_1^2\sigma_2^4} + \mu_2^2\sigma_1^2\sigma_2^2) - (\bcancel{\mu_1^2\sigma_2^4} + 2\mu_1\mu_2\sigma_1^2\sigma_2^2 + \cancel{\mu_2^2\sigma_1^4} )}{\sigma_1^2 + \sigma_2^2}  \frac{1}{2\sigma_1^2\sigma_2^2} \\[0.3cm]
% & = \frac{(\sigma_1^2\sigma_2^2)(\mu_1^2 + \mu_2^2 - 2\mu_1\mu_2)}{\sigma_1^2 + \sigma_2^2}\frac{1}{2\sigma_1^2\sigma_2^2} = \frac{\mu_1^2 + \mu_2^2 - 2\mu_1\mu_2}{2(\sigma_1^2 + \sigma_2^2)} = \frac{(\mu_1 - \mu_2)^2}{2(\sigma_1^2 + \sigma_2^2)}
% \end{split}
% \end{equation}
% 
% Luego,
% \begin{equation}
%  \theta = \frac{(x-\mu_{\times})^2}{2\sigma_{\times}^2} + \frac{(\mu_1 - \mu_2)^2}{2(\sigma_1^2 + \sigma_2^2)}
% \end{equation}
% 
% Colocando esta forma de $\theta$ en su lugar
% \begin{equation}
% \begin{split}
%  N(x;y,\beta^2)N(x;\mu,\sigma^2) & = \frac{1}{2\pi\sigma_1\sigma_2}\text{exp}\Bigg(-\underbrace{\left( \frac{(x-\mu_{\times})^2}{2\sigma_{\times}^2} + \frac{(\mu_1 - \mu_2)^2}{2(\sigma_1^2 + \sigma_2^2)} \right)}_{\theta} \Bigg) \\
%  & = \frac{1}{2\pi\sigma_1\sigma_2}\text{exp}\left(  - \frac{(x-\mu_{\times})^2}{2\sigma_{\times}^2} \right) \text{exp} \left( - \frac{(\mu_1 - \mu_2)^2}{2(\sigma_1^2 + \sigma_2^2)} \right)
% \end{split}
% \end{equation}
% 
% Multiplicando por $\sigma_{\times}\sigma_{\times}^{-1}$
% \begin{equation}
% \overbrace{\frac{\cancel{\sigma_1\sigma_2}}{\sqrt{\sigma_1^2+\sigma_2^2}}}^{\sigma_{\times}} \frac{1}{\sigma_{\times}} \frac{1}{2\pi\cancel{\sigma_1\sigma_2}}\text{exp}\left(  - \frac{(x-\mu_{\times})^2}{2\sigma_{\times}^2} \right) \text{exp} \left( - \frac{(\mu_1 - \mu_2)^2}{2(\sigma_1^2 + \sigma_2^2)} \right)
% \end{equation}
% 
% Luego,
% \begin{equation}
%  \frac{1}{\sqrt{2\pi}\sigma_{\times}}\text{exp}\left(  - \frac{(x-\mu_{\times})^2}{2\sigma_{\times}^2} \right) \frac{1}{\sqrt{2\pi(\sigma_1^2+\sigma_2^2)}} \text{exp} \left( - \frac{(\mu_1 - \mu_2)^2}{2(\sigma_1^2 + \sigma_2^2)} \right)
% \end{equation}
% 
% Retonando a la integral
% \begin{equation}
% \begin{split}
% I & = \int N(x;\mu_{\times},\sigma_{\times}^2) \overbrace{N(\mu_1;\mu_2,\sigma_1^2 + \sigma_2^2)}^{\text{Escalar independiente de x}} dx \\[0.3cm]
% & = N(\mu_1;\mu_2,\sigma_1^2 + \sigma_2^2) \underbrace{\int N(x,\mu_{\times},\sigma_{\times}^2)  dx}_{\text{Integra 1}} \\
% & = N(\mu_1;\mu_2,\sigma_1^2 + \sigma_2^2)
% \end{split}
% \end{equation}
% 
% \subsection{Suma de n normales}\label{suma_normales_induccion}
% 
% Sabemos que
% 
% \begin{equation}
% t_n = \sum_{i=1}^n x_i \sim \int \dots \int \mathbb{I}(t_n= \sum_{i=1}^n x_i ) \left( \prod_{i=1}^n N(x_i;\mu_i,\sigma_i^2) \right) dx_1 \dots dx_n = N(t;\sum_{i=1}^n \mu_i,\sum_{i=1}^n \sigma_i^2 )
% \end{equation}
% 
% 
% Queremos probar por inducci\'on.
% \begin{equation}
%  P(n):= \int \dots \int \mathbb{I}(t_n= \sum_{i=1}^n x_i ) \left( \prod_{i=1}^n N(x_i;\mu_i,\sigma_i^2) \right) dx_1 \dots dx_n \overset{?}{=} N(t;\sum_{i=1}^n \mu_i,\sum_{i=1}^n \sigma_i^2 )
% \end{equation}
% 
% \paragraph{Casos base}
% 
% \begin{equation}
% \begin{split}
%  P(1) := \int \mathbb{I}(t_1 = x_1) N(x_1;\mu_1,\sigma_1^2) dx_1 = N(x;\mu_1,\sigma_1^2)
% \end{split}
% \end{equation}
% 
% Luego $P(1)$ es verdadera.
% 
% \begin{equation}
%  \begin{split}
% P(2) & := \iint \mathbb{I}(t_2 = x_1 + x_2) N(x_1|\mu_1, \sigma_1^2)N(x_2|\mu_2, \sigma_2^2) dx_1dx_2 \\
%  &= \int N(x_1|\mu_1, \sigma_1^2) N(t_2 - x_1|\mu_2, \sigma_2^2) dx_1   \\
%  & = \int N(x_1|\mu_1, \sigma_1^2) N(x_1|t_2 - \mu_2, \sigma_2^2) dx_1 \\
%  & \overset{*}{=} \int \underbrace{N(t_2|\mu_1+\mu_2,\sigma_1^2 + \sigma_2^2)}_{\text{const.}} \underbrace{N(x_1|\mu_{*},\sigma_{*}^2) dx_1}_{1} \\
%  & = N(t_2|\mu_1+\mu_2,\sigma_1^2 + \sigma_2^2)
%  \end{split}
%  \end{equation}
% 
%  Donde $\overset{*}{=}$ vale por la demostraci\'on de miltiplicaci\'on de normales en la secci\'on~\ref{multiplicacion_normales}.
%  Luego, vale $P(2)$.
% 
% 
% \paragraph{Paso inductivo} $P(n) \Rightarrow P(n+1)$
% 
% Sea,
% \begin{equation}
%  P(n) :=\int \dots \int \mathbb{I}(t_n= \sum_{i=1}^n x_i ) \left( \prod_{i=1}^n N(x_i;\mu_i,\sigma_i^2) \right) dx_1 \dots dx_n = N(t;\sum_{i=1}^n \mu_i,\sum_{i=1}^n \sigma_i^2 )
% \end{equation}
% 
% Queremos ver que vale $P(n+1)$
% 
% \begin{equation}
%  P(n+1) := \int \dots \int \mathbb{I}(t_{n+1}=  x_{n+1} + \sum_{i=1}^{n} x_i ) \left( \prod_{i=1}^{n} N(x_i;\mu_i,\sigma_i^2) \right) N(x_{n+1};\mu_{n+1},\sigma_{n+1}^2) dx_1 \dots dx_{n} dx_{n+1}
% \end{equation}
% 
% Por independencia
% \begin{equation}
%  \int N(x_{n+1};\mu_{n+1},\sigma_{n+1}^2) \left( \int \dots \int \mathbb{I}(t_{n+1}= x_{n+1} + \sum_{i=1}^{n} x_i ) \left( \prod_{i=1}^{n} N(x_i;\mu_i,\sigma_i^2) \right)  dx_1 \dots dx_{n}\right) dx_{n+1}
% \end{equation}
% 
% Por hip\'otesis inductiva
% \begin{equation}
%  \int N(x_{n+1};\mu_{n+1},\sigma_{n+1}^2) N(t-x_{n+1};\sum_{i=1}^n \mu_i,\sum_{i=1}^n \sigma_i^2) dx_{n+1}
% \end{equation}
% 
% Por demostraci\'on de la secci\'on~\ref{multiplicacion_normales},
% \begin{equation}
%   N(t;\mu_{n+1}+\sum_{i=1}^{n} \mu_i,\sigma_{n+1}^2 \sum_{i=1}^n \sigma_i^2) dx_{n+1}
% \end{equation}
% 
% Luego, vale $P(n+1)$.
% 
% \subsection{Normal por acumulada de Normal}
% 
% Queremos resolver la integral
% 
% \begin{equation}
%  f(x) = \int N(y;\mu_1,\sigma_1^2)\Phi(y+x;\mu_2,\sigma_2^2) dy
% \end{equation}
% 
% Para ello trabajamos con la drivada $\frac{\partial}{\partial x}f(x) = \theta(x)$,
% \begin{equation}
%  \theta(x) = \frac{\partial}{\partial x}\int N(y;\mu_1,\sigma_1^2)\Phi(y+x;\mu_2,\sigma_2^2) dy
% \end{equation}
% 
% Por ``Dominated convergence theorem, integrales y derivadas pueden intercambiar posiciones.
% \begin{equation}
%  \theta(x) = \int N(y;\mu_1,\sigma_1^2)\frac{\partial}{\partial x}\Phi(y+x;\mu_2,\sigma_2^2) dy
% \end{equation}
% 
% La derivada de $\Phi$ es justamente una normal,
% \begin{equation}
% \begin{split}
% \theta(x) & = \int N(y;\mu_1,\sigma_1^2)N(y+x;\mu_2,\sigma_2^2) dy \\
% & = \int N(y;\mu_1,\sigma_1^2)N(y;\mu_2-x,\sigma_2^2) dy
% \end{split}
% \end{equation}
% 
% Por la demostraci\'on de la secci\'on~\ref{multiplicacion_normales} sabemos
% \begin{equation}
%  \theta(x) = N(\mu_1; \mu_2 - x, \sigma_1^2 + \sigma_2^2)
% \end{equation}
% 
% Por simetr\'ia
% \begin{equation}
%  \theta(x) = N(x; \mu_2 - \mu_1, \sigma_1^2 + \sigma_2^2)
% \end{equation}
% 
% Retornando a $f(x)$
% \begin{equation}
%  f(x) = \Phi(x; \mu_2 - \mu_1, \sigma_1^2 + \sigma_2^2)
% \end{equation}
% 
% \subsection{Division de Normales}\label{sec:division_normales}
% 
% \begin{equation}
% \kappa = \frac{N(x;\mu_f,\sigma_f^2)}{N(x;\mu_g,\sigma_g^2)} = N(x;\mu_f,\sigma_f^2)N(x;\mu_g,\sigma_g^2)^{-1}
% \end{equation}
% 
% Por definici\'on
% \begin{equation}
% \begin{split}
% \kappa & = \frac{1}{\sqrt{2\pi}\sigma_f}e^{-\left(\frac{(x-\mu_f)^2}{2\sigma_f^2}\right)} \left( \frac{1}{\sqrt{2\pi}\sigma_g}e^{-\left(\frac{(x-\mu_g)^2}{2\sigma_g^2}\right)} \right)^{-1} \\[0.3cm]
% & = \frac{1}{\cancel{\sqrt{2\pi}}\sigma_f}e^{-\left(\frac{(x-\mu_f)^2}{2\sigma_f^2}\right)} \frac{\cancel{\sqrt{2\pi}}\sigma_g}{1} e^{\left(\frac{(x-\mu_g)^2}{2\sigma_g^2}\right)} \\[0.3cm]
% & = \frac{\sigma_g}{\sigma_f}\text{exp}\Bigg(-\underbrace{\Big(\frac{(x-\mu_f)^2}{2\sigma_f^2} - \frac{(x-\mu_g)^2}{2\sigma_g^2}\Big)}_{\theta}\Bigg)
% \end{split}
% \end{equation}
% 
% Reorganizando $\theta$
% \begin{equation}
% \begin{split}
%  \theta & = \frac{(x-\mu_f)^2}{2\sigma_f^2} - \frac{(x-\mu_g)^2}{2\sigma_g^2} = \frac{\sigma_g^2(x-\mu_f)^2 - \sigma_f^2(x-\mu_g)^2}{2\sigma_f^2\sigma_g^2} \\[0.3cm]
%  & = \frac{\sigma_g^2(x^2+\mu_f^2-2\mu_fx) - \sigma_f^2(x^2+\mu_g^2-2\mu_gx)}{2\sigma_f^2\sigma_g^2}
% \end{split}
% \end{equation}
% 
% Expandimos y ordenamos en base $x$,
% \begin{equation}
% \begin{split}
%  \theta & = \left((\sigma_g^2 - \sigma_f^2)x^2 - 2(\sigma_g^2\mu_f - \sigma_f^2\mu_g)x + (\sigma_g^2\mu_f^2 - \sigma_f^2\mu_g^2 )\right) \frac{1}{2\sigma_f^2\sigma_g^2} \\[0.3cm]
%  & = \left(x^2 - \frac{2(\sigma_g^2\mu_f - \sigma_f^2\mu_g)}{(\sigma_g^2 - \sigma_f^2)}x + \frac{(\sigma_g^2\mu_f^2 - \sigma_f^2\mu_g^2 )}{(\sigma_g^2 - \sigma_f^2)}\right) \frac{(\sigma_g^2 - \sigma_f^2)}{2\sigma_f^2\sigma_g^2}
% \end{split}
% \end{equation}
% 
% Esto es cuadr\'atico en x. Dado que un t\'ermino $\varepsilon=0$, independiente de $x$ puede ser agregado para completar el cuadrado en $\theta$, esta prueba es suficiente para dterminar la media y la varianza cuando no es necesario normalizar.
% 
% \begin{equation}
%  \sigma_{\div} = \sqrt{\frac{\sigma_f^2\sigma_g^2}{(\sigma_g^2 - \sigma_f^2)}}
% \end{equation}
% 
% \begin{equation}
%  \mu_{\div} = \frac{(\sigma_g^2\mu_f - \sigma_f^2\mu_g)}{(\sigma_g^2 - \sigma_f^2)}
% \end{equation}
% 
% agregado $\varepsilon = \frac{\mu_{\div}^2-\mu_{\div}^2}{2\sigma_{\div}^2}$
% 
% \begin{equation}
% \theta = \frac{x^2 - 2\mu_{\div} + \mu_{\div}^2 }{2\sigma_{\div}^2} + \underbrace{ \frac{ \frac{(\sigma_g^2\mu_f^2 - \sigma_f^2\mu_g^2)}{(\sigma_g^2 - \sigma_f^2)} - \mu_{\div}^2 }{2\sigma_{\div}^2} }_{\varphi}
% \end{equation}
% 
% Reorganizando $\varphi$
% \begin{equation}
% \begin{split}
%  \varphi & = \left( \frac{(\sigma_g^2\mu_f^2 - \sigma_f^2\mu_g^2)}{(\sigma_g^2 - \sigma_f^2)} - \left(\frac{(\sigma_g^2\mu_f - \sigma_f^2\mu_g)}{(\sigma_g^2 - \sigma_f^2)} \right)^2 \right) \frac{(\sigma_g^2 - \sigma_f^2)}{2\sigma_f^2\sigma_g^2} \\[0.3cm]
%  & = \left((\sigma_g^2\mu_f^2 - \sigma_f^2\mu_g^2)(\sigma_g^2 - \sigma_f^2) - \left((\sigma_g^2\mu_f - \sigma_f^2\mu_g) \right)^2 \right) \frac{1}{2\sigma_f^2\sigma_g^2(\sigma_g^2 - \sigma_f^2)} \\[0.3cm]
%  & =  \left( \cancel{\sigma_g^4\mu_f^2} - 2\sigma_f^2\sigma_g^2\mu_g^2 + \bcancel{\sigma_f^4\mu_g^2} - (\cancel{\sigma_g^4\mu_f^2} + \bcancel{\sigma_f^4\mu_g^2 } - 2\sigma_f^2\sigma_g^2\mu_f\mu_g)\right) \frac{1}{2\sigma_f^2\sigma_g^2(\sigma_g^2 - \sigma_f^2)}
%  \end{split}
% \end{equation}
% 
% Cancelando los $\sigma^2$
% \begin{equation}
%  \varphi = \frac{- \mu_g^2 - \mu_f^2 + 2\mu_f\mu_g}{2(\sigma_g^2 - \sigma_f^2)} = \frac{- (\mu_g - \mu_f)^2}{2(\sigma_g^2 - \sigma_f^2)}
% \end{equation}
% 
% Luego $\theta$
% \begin{equation}
%  \theta = \frac{(x - \mu_{\div})^2}{2\sigma_{\div}^2} - \frac{(\mu_g - \mu_f)^2)}{2(\sigma_g^2 - \sigma_f^2)}
% \end{equation}
% 
% Por lo tanto
% \begin{equation}
% \begin{split}
%  \kappa & = \frac{\sigma_g}{\sigma_f}  \, \text{exp}\left(- \frac{(x - \mu_{\div})^2}{2\sigma_{\div}^2} + \frac{(\mu_g - \mu_f)^2)}{2(\sigma_g^2 - \sigma_f^2)}  \right)\\[0.3cm]
%  & = \frac{\sigma_g}{\sigma_f} \, e^{-\frac{(x - \mu_{\div})^2}{2\sigma_{\div}^2}} \, e^{\frac{(\mu_g - \mu_f)^2)}{2(\sigma_g^2 - \sigma_f^2)}}
% \end{split}
% \end{equation}
% 
% Multiplicando por $\frac{\sqrt{2\pi}}{\sqrt{2\pi}}\frac{\sigma_{\div}}{\sigma_{\div}}\frac{\sqrt{\sigma_g^2 - \sigma_f^2}}{\sqrt{\sigma_g^2 - \sigma_f^2}}=1$,
% \begin{equation}
% \begin{split}
%  \kappa & =  \frac{1}{\sqrt{2\pi}\sigma_{\div}} \, e^{-\frac{(x - \mu_{\div})^2}{2\sigma_{\div}^2}} \, \left( \frac
%  {1}{\sqrt{2\pi(\sigma_g^2 - \sigma_f^2)} } e^{-\frac{(\mu_g - \mu_f)^2)}{2(\sigma_g^2 - \sigma_f^2)}} \right)^{-1} \, \frac{\sigma_{\div}}{\sqrt{\sigma_g^2 - \sigma_f^2}}\frac{\sigma_g}{\sigma_f}\\[0.3cm]
%  & = \frac{N\left(x; \mu_{\div},\sigma_{\div}\right)}{N\left(\mu_g;\mu_f,\sigma_g^2-\sigma_f^2\right)} \frac{\sigma_g^2}{\sigma_g^2 - \sigma_f^2}
% \end{split}
% \end{equation}
% 
% 
% 

%% -----------------------------------------------------------------------------

\newpage
\end{document}
